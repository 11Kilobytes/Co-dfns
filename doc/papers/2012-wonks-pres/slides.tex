\def\majorstop#1{\null\vfill \centerline{\subtitlef #1}\par\vfill}
\def\newslide{\par\vfill\break}
\def\heading#1{\centerline{\headingf #1}\smallskip\hrule\medskip}

\font\titlef = "APL385 Unicode" at 36pt
\font\subtitlef = "APL385 Unicode" at 24pt
\font\authorf = "APL385 Unicode" at 18pt
\font\emailf = "APL385 Unicode" at 16pt
\font\datef = "APL385 Unicode" at 16pt
\font\headingf = "APL385 Unicode" at 24pt
\font\rm = "APL385 Unicode" at 16pt

\hoffset = -0.5in
\voffset = -0.5in
\hsize = 7in
\vsize = 5in
\parindent = 0in
\parskip = 0.5in

\null\rm\newslide

\null\vfill
\centerline{\titlef Enhancing CnC}
\centerline{\subtitlef Or, an introduction that makes sense.}
\bigskip
\centerline{\authorf Aaron W. Hsu}
\smallskip
\centerline{\emailf awhsu@indiana.edu}
\bigskip\bigskip
\centerline{\datef 2 March 2012}

\newslide

\heading{Goals of the Talk}

Understand CnC

Prompt feedback on my language

%%%  C. My Research

\newslide

\heading{Current Research}

Implementing CnC extensions

Enriching the language

Combining APL and CnC

Bulding formalism around CnC

Streaming Optimizations

\newslide

\heading{Key ideas and motivations}

What is "My" programming language?

Emphasize the key ideas

{\parindent = 0.33in
Rigorously expressed

Given first priority\par}

Leveraging effective implicit parallelism

Avoid making parallelism the main thing

\newslide

\majorstop{CnC by Example: The Game of Life}

\newslide

\majorstop{Classic APL Example}

\newslide

\heading{Parallelizing}

Game of Life is basically a stencil

Should not need to know stencil optmization

Slicing/Data parallelism

\newslide

\heading{Using CnC to parallelize}

Very simple, but practical

Creation should be independent of tuning

CnC should orchestrate, not dominate

\newslide

\heading{Composing via a CnC Graph}

Let's draw the CnC Graph....

\newslide
\majorstop{Combining CnC with APL}
\newslide

\heading{Motivation}

CnC needed a more expressive graph

APL is great for concisely expressing indexing patterns

Inspired by the 3 layer cake

I wanted to push APL further

\newslide

\heading{Revisiting the Game of Life}

Graph is not just a pretty picture

What does the code look like?

\newslide

\heading{Pushing it further}

User extensions of analysis

Fully formalizing the language

\newslide
\majorstop{Conclusions and Future Work}
\newslide

\heading{Current Work}

Compiler in APL (Dyalog)

Runtime in UPC (Unified Parallel C)

Managing analysis of streaming

Talking with Dyalog to fund summer research

\newslide

\heading{Interesting questions}

How formal can we go?

How fast can it go?

Modularity/Hierarchy

User language extensions

User compiler extensions

\newslide

\heading{Other optimizations}

Locality of Data on nodes

Vectorization at the node level

Scheduling of Steps

Interlanguage overhead

Optimization by proof

\newslide

\heading{Conclusions}

CnC is actually easy to understand!

CnC and APL compose well to form a nice language

Interesting opportunities in a language that I like to use

\newslide

\majorstop{Thank you.}

\bye