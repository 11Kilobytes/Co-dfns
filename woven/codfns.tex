\documentclass{article}% ===> this file was generated automatically by noweave --- better not edit it
\usepackage[utf8]{inputenc}

\usepackage{noweb}
\pagestyle{noweb}
\noweboptions{longxref}

\usepackage{fontspec}
\usepackage{unicode-math}
\setmainfont[Ligatures=TeX]{Libre Baskerville}
\setsansfont[Ligatures=TeX]{Lucida Sans Unicode}
\setmonofont{APL385 Unicode}
\setmathfont{Cambria Math}

\usepackage{polyglossia}
\setdefaultlanguage[variant=american]{english}

\usepackage{hyperref}
\usepackage{booktabs}

\begin{document}

\nwfilename{codfns.nw}\nwbegindocs{1}\nwdocspar
\title{The Co-dfns Compiler}
\author{Aaron W. Hsu}

\maketitle

\vfill

\noindent
Co-dfns Compiler: High-performance, Parallel APL Compiler\\
Copyright $\copyright$ 2011-2022 Aaron W. Hsu <arcfide@sacrideo.us>
\medskip

\noindent
This program is free software: you can redistribute it and/or
modify it under the terms of the GNU Affero General Public License as
published by the Free Software Foundation, either version 3 of the
License, or (at your option) any later version.
\medskip

\noindent
This program is distributed in the hope that it will be useful,
but WITHOUT ANY WARRANTY; without even the implied warranty of
MERCHANTABILITY or FITNESS FOR A PARTICULAR PURPOSE.    See the
GNU Affero General Public License for more details.\medskip

\noindent
You should have received a copy of the GNU Affero General Public License
along with this program.
If not, see http://gnu.org/licenses.
\medskip

\noindent
\emph{This program is available under other license terms. Please contact
Aaron W. Hsu <arcfide@sacrideo.us> for more information.}


\clearpage

\tableofcontents

\clearpage

\section{Introduction}

\section{User's Guide}

\section{Co-dfns Architecture}

This section describes the ``big picture'' parts of the Co-dfns compiler.
The intent here is to try to show how all of the various moving parts
of the compiler fit together,
to provide a sort of road map that will give you a precise plan
for understanding how the various components affect one another.
One of the most important things to understand in any compiler
is the net effect a local change in the code
can have on the rest of the system,
so I hope that this section will help to clarify this.

The design of the Co-dfns compiler is one of austerity and minimalism.
My intent is, was, and hopefully shall remain that of producing an 
exceptionally clear design that avoids or eliminates unnecessary 
code and complexity within the design.
I attack this problem in many ways, but I primarily attempt to do 
this by both reducing the size of the code surface in total,
that is, write less code, as well as reducing the number of entry 
points and paths through that code. 
In other words, my ideal design is one in which you enter the 
compiler in some limited, but well defined and useful set of entry 
points, and then proceed in a linear fashion through the code as the
execution path, resulting finally in your result.
This is the ``ultimate'' in data flow, functionally oriented programming.

The ramifications of this design choice implies a few important things.
Firstly, it implies that I reduce and eliminate any code
that represents boilerplate or that does not actively contribute
to the ``big picture'' of the code.
This is required in an extreme degree if I am to reduce the overall
complexity of the design.
This also implies that there is very little intentional redundancy in 
the shape and style of the source,
making it very terse and compact.
Since there are intentionally very few entry and exit points through 
the control flow of the code, 
this reduces the number of dependencies for me to be aware of when 
dealing with a single piece of code,
but this also comes at the cost of not being able to see many examples
of the interfaces with that code. 
Often, there will be one, and only one place, in which a given piece 
of code is used, and I do not want the code to needlessly store 
excess information in its source that doesn't need to be there.

This all culminates in something that can be quite shocking at first:
making a change to the source is almost always a big deal.
If all the source code is meaningful and carefully constructed,
this also means that making changes to this code are almost always 
non-trivial, because if the code represented something trivial,
I would have tried to remove it from the code so that only the 
``big things'' were in the code itself.
Thus, anyone who wishes to view and read the compiler code should
take it upon themselves to appreciate the way in which the code flows
together,
and how the flow of the program runs, 
as doing so will be essential to understanding how to make changes to 
the source without breaking something.
Fortunately, this does come with the intended benefits of a very 
short and simple codebase that has clear flow through the system,
it just means that if you want to change something, 
make sure you realize that you are almost always likely to be working
at the ``architectural'' level, rather than at the small and trivial
level of details.

The compiler is designed to fit into a single Dyalog APL namespace,
and importantly, we do not define additional nested namespaces or 
other forms of name hiding. 
I intentionally want to restrict the namespace to a single global one.
This single global namespace should therefore contain the carefully 
curated names that matter, and any that do not matter should, ideally,
not be defined or used.
The namespace itself can be divided into three main groupings:
the public facing entry-points into the system,
the compiler logic itself,
and the utilities or other elements that serve to support the others.
This gives use the following code outline.

\nwenddocs{}\nwbegincode{2}\sublabel{NW2YR5B-1p0Y9w-1}\nwmargintag{{\nwtagstyle{}\subpageref{NW2YR5B-1p0Y9w-1}}}\moddef{*~{\nwtagstyle{}\subpageref{NW2YR5B-1p0Y9w-1}}}\endmoddef\nwstartdeflinemarkup\nwenddeflinemarkup
:Namespace \nwlinkedidentc{codfns}{NW2YR5B-1p0Y9w-1}

        \LA{}Global Settings~{\nwtagstyle{}\subpageref{NW2YR5B-2z4lmm-1}}\RA{}
        \LA{}The Fix API~{\nwtagstyle{}\subpageref{NW2YR5B-2o6hoR-1}}\RA{}
        \LA{}User-command API~{\nwtagstyle{}\subpageref{NW2YR5B-2YFL86-1}}\RA{}

        \LA{}Parser~{\nwtagstyle{}\subpageref{NW2YR5B-38DvvD-1}}\RA{}
        \LA{}Compiler~{\nwtagstyle{}\subpageref{NW2YR5B-1jC2QX-1}}\RA{}
        \LA{}Code Generator~{\nwtagstyle{}\subpageref{NW2YR5B-HCURD-1}}\RA{}
        \LA{}Interface to the backend C compiler~{\nwtagstyle{}\subpageref{NW2YR5B-2LnsgP-1}}\RA{}
        \LA{}Linking with Dyalog~{\nwtagstyle{}\subpageref{NW2YR5B-zH457-1}}\RA{}

        \LA{}Must Have APL Utilities~{\nwtagstyle{}\subpageref{NW2YR5B-AF6fz-1}}\RA{}
        \LA{}AST Record Structure~{\nwtagstyle{}\subpageref{NW2YR5B-1gMT0G-1}}\RA{}
        \LA{}Converters between parent and depth vectors~{\nwtagstyle{}\subpageref{NW2YR5B-3dpNce-1}}\RA{}
        \LA{}XML Rendering~{\nwtagstyle{}\subpageref{NW2YR5B-1evvdc-1}}\RA{}
        \LA{}Pretty-printing AST trees~{\nwtagstyle{}\subpageref{NW2YR5B-1VExi3-1}}\RA{}

:EndNamespace
\nwindexdefn{\nwixident{codfns}}{codfns}{NW2YR5B-1p0Y9w-1}\eatline
\nwnotused{*}\nwidentdefs{\\{{\nwixident{codfns}}{codfns}}}\nwendcode{}\nwbegindocs{3}\nwdocspar
This {\Tt{}\LA{}*~{\nwtagstyle{}\subpageref{NW2YR5B-1p0Y9w-1}}\RA{}\nwendquote} chunk is meant to be stored to a file. 
We have a build system for doing this that depends
on the contents of the {\Tt{}\LA{}Tangle Commands~{\nwtagstyle{}\subpageref{NW2YR5B-ufqpm-1}}\RA{}\nwendquote} chunk. 
Thus, we follow the convention here of updating the contents of 
the {\Tt{}\LA{}Tangle Commands~{\nwtagstyle{}\subpageref{NW2YR5B-ufqpm-1}}\RA{}\nwendquote} chunk each time that we initially define
a new chunk that is intended to be output to a file during the 
tangling process. 
See more about the build infrastructure later in this document.

\nwenddocs{}\nwbegincode{4}\sublabel{NW2YR5B-ufqpm-1}\nwmargintag{{\nwtagstyle{}\subpageref{NW2YR5B-ufqpm-1}}}\moddef{Tangle Commands~{\nwtagstyle{}\subpageref{NW2YR5B-ufqpm-1}}}\endmoddef\nwstartdeflinemarkup\nwusesondefline{\\{NW2YR5B-42EjwV-1}}\nwprevnextdefs{\relax}{NW2YR5B-ufqpm-2}\nwenddeflinemarkup
echo "Tangling \nwlinkedidentc{codfns}{NW2YR5B-1p0Y9w-1}\nwlinkedidentc{.apln}{NW2YR5B-ufqpm-1}..."
notangle \nwlinkedidentc{codfns}{NW2YR5B-1p0Y9w-1}.nw > \nwlinkedidentc{src}{NW2YR5B-3C6SQT-1}/\nwlinkedidentc{codfns}{NW2YR5B-1p0Y9w-1}\nwlinkedidentc{.apln}{NW2YR5B-ufqpm-1}
\nwindexdefn{\nwixident{codfns.apln}}{codfns.apln}{NW2YR5B-ufqpm-1}\eatline
\nwalsodefined{\\{NW2YR5B-ufqpm-2}\\{NW2YR5B-ufqpm-3}\\{NW2YR5B-ufqpm-4}\\{NW2YR5B-ufqpm-5}\\{NW2YR5B-ufqpm-6}\\{NW2YR5B-ufqpm-7}\\{NW2YR5B-ufqpm-8}\\{NW2YR5B-ufqpm-9}}\nwused{\\{NW2YR5B-42EjwV-1}}\nwidentdefs{\\{{\nwixident{codfns.apln}}{codfns.apln}}}\nwidentuses{\\{{\nwixident{codfns}}{codfns}}\\{{\nwixident{src}}{src}}}\nwindexuse{\nwixident{codfns}}{codfns}{NW2YR5B-ufqpm-1}\nwindexuse{\nwixident{src}}{src}{NW2YR5B-ufqpm-1}\nwendcode{}\nwbegindocs{5}\nwdocspar
The primary user-facing interfaces into the compiler are 
{\Tt{}\LA{}The Fix API~{\nwtagstyle{}\subpageref{NW2YR5B-2o6hoR-1}}\RA{}\nwendquote} and the {\Tt{}\LA{}User-command API~{\nwtagstyle{}\subpageref{NW2YR5B-2YFL86-1}}\RA{}\nwendquote}. 
These are the ways that you primarily drive the entire compiler.
I intentionally expose the rest of the compiler interfaces
without hiding them so that people who wish to leverage these 
other parts of the system without using the ``entire'' compiler 
pipeline are able to do so, but I do not consider this a public
interface.

This distinction matters because of our testing philosophy and our
version numbering. 
Generally speaking, our version numbering scheme only tracks a major
or minor change in the compiler when the externally facing interfaces
receive some fundamental changes.
Changes to the internal changes are \emph{not} considered for this
versioning scheme.
Moreover, since I intend for there to be great freedom in changing 
and altering the behavior of these internal pipeline interfaces,
these interfaces are not directly tested, 
and the test suite should \emph{not} include testing against these
internal interfaces.
We philosophically only test against the external interfaces,
and eschew internal unit tests.%
\footnote{You can read more of my opinions on this matter
in my article, 
\href{https://www.sacrideo.us/the-fallacy-of-unit-testing/}
{``The Fallacy of Unit Testing''}.}

The utility functions defined below the core compiler pipeline
represent functionality that is tangential to the main compiler
operation.
However, these utilities also tend to represent some specific 
insight into the design of the compiler.
Understanding the core AST structure and design as well as 
getting a grip on how to manipulate the core tree manipulation
structures are vital to understanding the rest of the code.
Therefore, this section spends more time on discussing these
topics before the upcoming sections dealing with a more detailed 
exposition of the compiler itself.
However, there are utilities that we consider more advanced, 
such as the pretty-printing functions and XML rendering
that are topics of interest to advanced users of the compiler,
but which are not part of the main compiler pipeline.
Even though these functions have intentionally general 
application and are likely to be useful not only to those 
working on the compiler itself but also to those who are using 
more advanced compiler features, 
these utilities are not critical to a deep understanding 
of the compiler, 
so these are not discussed in this section. 
Instead, we discuss those topics in the section on developer 
tooling and infrastructure concerns.

The remaining parts of this section will describe the external
facing interfaces to the compiler as well as the core underlying 
data structures and idioms that form the underlying skeleton and 
foundation for writing and working with any aspect of the compiler.
These are all feature and component agnostic elements of the system
that do not belong solely to only a single part, 
but that impact all other elements of the compiler source code,
and so it pays especially well to pay attention and understand
this code to a high degree.

\nwenddocs{}\nwbegindocs{6}\subsection{Global Settings}

There are some global options that we assume to exist throughout
the compiler.
These set the standard behaviors as well as serve as knobs that 
can be tweaked in some cases to identify what behaviors we want 
from the rest of the compiler.

First, we have a set of read-only global constants that are defined
to configure our APL environment.
These are the typical ones, and we try to stick to the defaults,
except that we are sane, and thus we use {\Tt{}\nwlinkedidentq{⎕IO}{NW2YR5B-2z4lmm-1}\nwendquote} set to {\Tt{}0\nwendquote}.

\nwenddocs{}\nwbegincode{7}\sublabel{NW2YR5B-2z4lmm-1}\nwmargintag{{\nwtagstyle{}\subpageref{NW2YR5B-2z4lmm-1}}}\moddef{Global Settings~{\nwtagstyle{}\subpageref{NW2YR5B-2z4lmm-1}}}\endmoddef\nwstartdeflinemarkup\nwusesondefline{\\{NW2YR5B-1p0Y9w-1}}\nwprevnextdefs{\relax}{NW2YR5B-2z4lmm-2}\nwenddeflinemarkup
\nwlinkedidentc{⎕IO}{NW2YR5B-2z4lmm-1} \nwlinkedidentc{⎕ML}{NW2YR5B-2z4lmm-1} \nwlinkedidentc{⎕WX}{NW2YR5B-2z4lmm-1}←0 1 3
\nwindexdefn{\nwixident{⎕IO}}{⎕IO}{NW2YR5B-2z4lmm-1}\nwindexdefn{\nwixident{⎕ML}}{⎕ML}{NW2YR5B-2z4lmm-1}\nwindexdefn{\nwixident{⎕WX}}{⎕WX}{NW2YR5B-2z4lmm-1}\eatline
\nwalsodefined{\\{NW2YR5B-2z4lmm-2}\\{NW2YR5B-2z4lmm-3}\\{NW2YR5B-2z4lmm-4}}\nwused{\\{NW2YR5B-1p0Y9w-1}}\nwidentdefs{\\{{\nwixident{⎕IO}}{⎕IO}}\\{{\nwixident{⎕ML}}{⎕ML}}\\{{\nwixident{⎕WX}}{⎕WX}}}\nwendcode{}\nwbegindocs{8}\nwdocspar
\noindent
Additionally, we set a {\Tt{}\nwlinkedidentq{VERSION}{NW2YR5B-2z4lmm-2}\nwendquote} constant to track changes to the 
system through the distributions.
We use semantic versioning%
\footnote{\href{https://semver.org/}{https://semver.org/}}
as our versioning scheme.
That being said, we also do not have particular qualms about changing
the public API at a rapid pace, provided that we document this.

\nwenddocs{}\nwbegincode{9}\sublabel{NW2YR5B-2z4lmm-2}\nwmargintag{{\nwtagstyle{}\subpageref{NW2YR5B-2z4lmm-2}}}\moddef{Global Settings~{\nwtagstyle{}\subpageref{NW2YR5B-2z4lmm-1}}}\plusendmoddef\nwstartdeflinemarkup\nwusesondefline{\\{NW2YR5B-1p0Y9w-1}}\nwprevnextdefs{NW2YR5B-2z4lmm-1}{NW2YR5B-2z4lmm-3}\nwenddeflinemarkup
\nwlinkedidentc{VERSION}{NW2YR5B-2z4lmm-2}←4 1 0
\nwindexdefn{\nwixident{VERSION}}{VERSION}{NW2YR5B-2z4lmm-2}\eatline
\nwused{\\{NW2YR5B-1p0Y9w-1}}\nwidentdefs{\\{{\nwixident{VERSION}}{VERSION}}}\nwendcode{}\nwbegindocs{10}\nwdocspar
\noindent
We depend on ArrayFire%
\footnote{\href{https://arrayfire.com/}{https://arrayfire.com/}}
for much of our GPU backend functionality.
This means we need to know two things,
where ArrayFire is installed
and which ArrayFire backend we should use when compiling.
We only really need to know where ArrayFire is installed on UNIX
style systems, as these systems seem to be much more variable in 
this regard, and there is an environment variable that we can use 
in Windows to find out where ArrayFire is installed more conveniently
on that platform.
We default to using {\Tt{}'cuda'\nwendquote} as our main option, but we also 
support the following options for {\Tt{}\nwlinkedidentq{AF∆LIB}{NW2YR5B-2z4lmm-3}\nwendquote}:

\begin{verbatim}
cuda opencl cpu
\end{verbatim}

\noindent
Using {\Tt{}''\nwendquote} for {\Tt{}\nwlinkedidentq{AF∆LIB}{NW2YR5B-2z4lmm-3}\nwendquote} will use ArrayFire's unified backend,
but we don't default to this because we have seen some issues on 
some platforms with reliability problems.
To avoid this, we choose to use {\Tt{}cuda\nwendquote} as the default,
which tends to either work or fail explicitly, 
which allows the user to respond rather than crashing 
ungracefully in the case of the unified backend.

The least reliable backend we have seen is the {\Tt{}opencl\nwendquote} one, 
which seems to be more hit or miss depending on the underlying
stability of the OpenCL drivers that are installed on the user's
system.
In particular, some Linux OpenCL installations seem to be 
particularly fragile.
In such cases, always make sure that a good, solid OpenCL library
is being used.

\nwenddocs{}\nwbegincode{11}\sublabel{NW2YR5B-2z4lmm-3}\nwmargintag{{\nwtagstyle{}\subpageref{NW2YR5B-2z4lmm-3}}}\moddef{Global Settings~{\nwtagstyle{}\subpageref{NW2YR5B-2z4lmm-1}}}\plusendmoddef\nwstartdeflinemarkup\nwusesondefline{\\{NW2YR5B-1p0Y9w-1}}\nwprevnextdefs{NW2YR5B-2z4lmm-2}{NW2YR5B-2z4lmm-4}\nwenddeflinemarkup
\nwlinkedidentc{AF∆PREFIX}{NW2YR5B-2z4lmm-3}←'/opt/arrayfire'
\nwlinkedidentc{AF∆LIB}{NW2YR5B-2z4lmm-3}←'cuda'
\nwindexdefn{\nwixident{AF∆PREFIX}}{AF∆PREFIX}{NW2YR5B-2z4lmm-3}\nwindexdefn{\nwixident{AF∆LIB}}{AF∆LIB}{NW2YR5B-2z4lmm-3}\eatline
\nwused{\\{NW2YR5B-1p0Y9w-1}}\nwidentdefs{\\{{\nwixident{AF∆LIB}}{AF∆LIB}}\\{{\nwixident{AF∆PREFIX}}{AF∆PREFIX}}}\nwendcode{}\nwbegindocs{12}\nwdocspar
\noindent
On Windows, we rely on the Visual Studio C/C++ compiler to build
our runtime and user code.
We have settled on trying to stay as up to date with this as 
possible. 
However, there are many different installation paths used by 
Visual Studio, which can make it difficult to know where to look
unless we hardcode each location.
Instead, we assume that Visual Studio will not be a primary 
interest to our users,
making it likely that they will be installing Visual Studio 
only as a dependency for using Co-dfns.
In this case, it is likely that they will be using the Community 
version.
Thus, we default to using the latest version of Visual Studio 
of which we are aware and using the Community version of this,
which Microsoft does not charge for.

If a different version of Visual Studio is installed, then it is 
important to figure out what the right path should be to locate 
the Visual Studio installation. 
The main thing we need to get from this path is access 
to the {\Tt{}vcvarsall.bat\nwendquote} batch file.
This file configures the {\Tt{}cmd.exe\nwendquote} environment to be able to 
find the Visual Studio compiler and work in the right way.
In the 2002 Community addition, and apparently most new versions 
of Visual Studio, this is located in the {\Tt{}VC{\nwbackslash}Auxiliary{\nwbackslash}Build{\nwbackslash}\nwendquote}
subdirectory of the main installation folder.
When changing this path, we want to make sure that the following
path points to the correct {\Tt{}vcvarsall.bat\nwendquote} file:

\begin{verbatim}
VS∆PATH,'\VC\Auxiliary\Build\vcvarsall.bat'
\end{verbatim}

\noindent
Most users will simply need to alter {\Tt{}Community\nwendquote} to match the 
edition of Visual Studio 2022 that they have installed on their 
system.

\nwenddocs{}\nwbegincode{13}\sublabel{NW2YR5B-2z4lmm-4}\nwmargintag{{\nwtagstyle{}\subpageref{NW2YR5B-2z4lmm-4}}}\moddef{Global Settings~{\nwtagstyle{}\subpageref{NW2YR5B-2z4lmm-1}}}\plusendmoddef\nwstartdeflinemarkup\nwusesondefline{\\{NW2YR5B-1p0Y9w-1}}\nwprevnextdefs{NW2YR5B-2z4lmm-3}{\relax}\nwenddeflinemarkup
\nwlinkedidentc{VS∆PATH}{NW2YR5B-2z4lmm-4}←'\\Program Files\\Microsoft Visual Studio'
\nwlinkedidentc{VS∆PATH}{NW2YR5B-2z4lmm-4},←'\\2022\\Community'
\nwindexdefn{\nwixident{VS∆PATH}}{VS∆PATH}{NW2YR5B-2z4lmm-4}\eatline
\nwused{\\{NW2YR5B-1p0Y9w-1}}\nwidentdefs{\\{{\nwixident{VS∆PATH}}{VS∆PATH}}}\nwendcode{}\nwbegindocs{14}\nwdocspar
\subsection{The Fix API}

One of the core entry points into the compiler is through the {\Tt{}\nwlinkedidentq{Fix}{NW2YR5B-2o6hoR-1}\nwendquote}
function.
This function is designed to mimic and more or less replace the
use of the {\Tt{}⎕FIX\nwendquote} function found in Dyalog APL.
Its design models that behavior, and it is important as an entry-point
because it exercises most of the core elements of the compiler.
In particular, the design of the compiler's pipeline is demonstrated
most fully in this function.

$$Parse → Compile → Generate → Backend → Link$$

\noindent
The interfaces to the {\Tt{}⎕FIX\nwendquote} function and the Co-dfns {\Tt{}\nwlinkedidentq{Fix}{NW2YR5B-2o6hoR-1}\nwendquote}
function differ in a few key ways.
The left argument to {\Tt{}\nwlinkedidentq{Fix}{NW2YR5B-2o6hoR-1}\nwendquote} is a character vector giving the name
to use when generating files and other artifacts.
This does \emph{not} affect the name of the resulting namespace,
since that is defined, if at all, in the file source itself.
The {\Tt{}⍺\nwendquote} argument only affects the name of the files and other
outputs that {\Tt{}\nwlinkedidentq{Fix}{NW2YR5B-2o6hoR-1}\nwendquote} generates.

We also print out which part of the compiler we are in when we
enter that ``phase''. Doing this helps to give us an intuitive sense
of how fast each phase is and whether one phase is taking an
abnormally long time or not.
It also helps in debugging.

\nwenddocs{}\nwbegincode{15}\sublabel{NW2YR5B-2o6hoR-1}\nwmargintag{{\nwtagstyle{}\subpageref{NW2YR5B-2o6hoR-1}}}\moddef{The Fix API~{\nwtagstyle{}\subpageref{NW2YR5B-2o6hoR-1}}}\endmoddef\nwstartdeflinemarkup\nwusesondefline{\\{NW2YR5B-1p0Y9w-1}}\nwenddeflinemarkup
\nwlinkedidentc{Fix}{NW2YR5B-2o6hoR-1}←\{
        _←a n s \nwlinkedidentc{src}{NW2YR5B-3C6SQT-1}←\nwlinkedidentc{PS}{NW2YR5B-38DvvD-1} ⍵⊣⍞←'P'
        _←          TT _⊣⍞←'C'
        _←          GC _⊣⍞←'G'
        _←        ⍺ CC _⊣⍞←'B'
                  n NS _⊣⍞←'L'
\}
\nwindexdefn{\nwixident{Fix}}{Fix}{NW2YR5B-2o6hoR-1}\eatline
\nwused{\\{NW2YR5B-1p0Y9w-1}}\nwidentdefs{\\{{\nwixident{Fix}}{Fix}}}\nwidentuses{\\{{\nwixident{PS}}{PS}}\\{{\nwixident{src}}{src}}}\nwindexuse{\nwixident{PS}}{PS}{NW2YR5B-2o6hoR-1}\nwindexuse{\nwixident{src}}{src}{NW2YR5B-2o6hoR-1}\nwendcode{}\nwbegindocs{16}\nwdocspar
The input requirements for {\Tt{}\nwlinkedidentq{Fix}{NW2YR5B-2o6hoR-1}\nwendquote} are not listed in the definition
itself, because both the parser {\Tt{}\nwlinkedidentq{PS}{NW2YR5B-38DvvD-1}\nwendquote} and the {\Tt{}\nwlinkedidentq{Fix}{NW2YR5B-2o6hoR-1}\nwendquote} function
need to use the same basic checks,
and since the {\Tt{}\nwlinkedidentq{Fix}{NW2YR5B-2o6hoR-1}\nwendquote} function calls the parser
as its first entry point,
it doesn't make much sense to
duplicate that work in both places.
The requirements are as follows:

\begin{itemize}
        \item Scalar/Vector
        \item Character type
        \item Simple or Vector of Vectors
\end{itemize}

\noindent
We generate a {\Tt{}DOMAIN\ ERROR\nwendquote} if the inputs are not well-formed.

\nwenddocs{}\nwbegincode{17}\sublabel{NW2YR5B-13WClx-1}\nwmargintag{{\nwtagstyle{}\subpageref{NW2YR5B-13WClx-1}}}\moddef{Verify source input \code{}⍵\edoc{}, set \code{}IN\edoc{}~{\nwtagstyle{}\subpageref{NW2YR5B-13WClx-1}}}\endmoddef\nwstartdeflinemarkup\nwusesondefline{\\{NW2YR5B-38DvvD-1}}\nwenddeflinemarkup
IN←⍵

err←'PARSER EXPECTS SCALAR OR VECTOR INPUT'
1<≢⍴IN:err ⎕\nwlinkedidentc{SIGNAL}{NW2YR5B-aELRs-1} 11

err←'PARSER EXPECTS SIMPLE OR VECTOR OF VECTOR INPUT'
2<|≡IN:err ⎕\nwlinkedidentc{SIGNAL}{NW2YR5B-aELRs-1} 11

\LA{}Normalize the input formatting~{\nwtagstyle{}\subpageref{NW2YR5B-32ArUy-1}}\RA{}

err←'PARSER EXPECTS CHARACTER ARRAY'
0≠10|⎕DR IN:err ⎕\nwlinkedidentc{SIGNAL}{NW2YR5B-aELRs-1} 11
\nwused{\\{NW2YR5B-38DvvD-1}}\nwidentuses{\\{{\nwixident{SIGNAL}}{SIGNAL}}}\nwindexuse{\nwixident{SIGNAL}}{SIGNAL}{NW2YR5B-13WClx-1}\nwendcode{}\nwbegindocs{18}\nwdocspar

The input formatting that is accepted means that newlines could be
denoted either with {\Tt{}LF\nwendquote}, {\Tt{}CR\nwendquote}, or {\Tt{}CRLF\nwendquote}
sequences inside of the vectors
themselves or they could be denoted by having separate vectors
for the various lines,
or even a mixture of both.
To simplify this situation we want to normalize them so that we are
always dealing with some combination of {\Tt{}LF\nwendquote}, {\Tt{}CR\nwendquote}, and {\Tt{}CRLF\nwendquote}
sequences
within the file itself, rather than dealing with the nested
situation.
This ensures that after verification of the input,
everything will work off of the same format.
We intentionally put a newline at the end of the file even if we
may not require one because it is possible that we are dealing
with a file that is missing its final newline.
By always adding one, we ensure that every line in the input
is always terminated by a line ending.
Life is also simpler if we just use LF as our line ending instead
of something else,
this means that future code must be aware that there could be mixed
line endings in the file.

\nwenddocs{}\nwbegincode{19}\sublabel{NW2YR5B-32ArUy-1}\nwmargintag{{\nwtagstyle{}\subpageref{NW2YR5B-32ArUy-1}}}\moddef{Normalize the input formatting~{\nwtagstyle{}\subpageref{NW2YR5B-32ArUy-1}}}\endmoddef\nwstartdeflinemarkup\nwusesondefline{\\{NW2YR5B-13WClx-1}}\nwenddeflinemarkup
IN←∊(⊆IN),¨⎕UCS 10
\nwused{\\{NW2YR5B-13WClx-1}}\nwendcode{}\nwbegindocs{20}\nwdocspar

\subsection{The User Command API}

\nwenddocs{}\nwbegincode{21}\sublabel{NW2YR5B-2YFL86-1}\nwmargintag{{\nwtagstyle{}\subpageref{NW2YR5B-2YFL86-1}}}\moddef{User-command API~{\nwtagstyle{}\subpageref{NW2YR5B-2YFL86-1}}}\endmoddef\nwstartdeflinemarkup\nwusesondefline{\\{NW2YR5B-1p0Y9w-1}}\nwenddeflinemarkup
∇Z←Help _
 Z←'Usage: <object> <target> [-af=\{cpu,opencl,cuda\}]'
∇

∇r←List
 r←⎕NS¨1⍴⊂⍬ ⋄ r.Name←,¨⊂'Compile' ⋄ r.Group←⊂'CODFNS'
 r[0].Desc←'Compile an object using Co-dfns'
 r.Parse←⊂'2S -af=cpu opencl cuda '
∇

∇ Run(C I);Convert;in;out
⍝ Parameters
⍝      \nwlinkedidentc{AF∆LIB}{NW2YR5B-2z4lmm-3}        ArrayFire backend to use
 Convert←\{⍺(⎕SE.SALT.Load'[SALT]/lib/NStoScript -noname').ntgennscode ⍵\}
 in out←I.Arguments ⋄ \nwlinkedidentc{AF∆LIB}{NW2YR5B-2z4lmm-3}←I.af''⊃⍨I.af≡0
 S←(⊂':Namespace ',out),2↓0 0 0 out Convert ##.THIS.⍎in
 →0⌿⍨'Compile'≢C
 \{##.THIS.⍎out,'←⍵'\}out \nwlinkedidentc{Fix}{NW2YR5B-2o6hoR-1} S⊣⎕EX'##.THIS.',out
∇
\nwused{\\{NW2YR5B-1p0Y9w-1}}\nwidentuses{\\{{\nwixident{AF∆LIB}}{AF∆LIB}}\\{{\nwixident{Fix}}{Fix}}}\nwindexuse{\nwixident{AF∆LIB}}{AF∆LIB}{NW2YR5B-2YFL86-1}\nwindexuse{\nwixident{Fix}}{Fix}{NW2YR5B-2YFL86-1}\nwendcode{}\nwbegindocs{22}\nwdocspar

\subsection{AST Record Structure}

\nwenddocs{}\nwbegincode{23}\sublabel{NW2YR5B-1gMT0G-1}\nwmargintag{{\nwtagstyle{}\subpageref{NW2YR5B-1gMT0G-1}}}\moddef{AST Record Structure~{\nwtagstyle{}\subpageref{NW2YR5B-1gMT0G-1}}}\endmoddef\nwstartdeflinemarkup\nwusesondefline{\\{NW2YR5B-1p0Y9w-1}}\nwenddeflinemarkup
f∆←'ptknfsrdx'
N∆←'ABCEFGKLMNOPSVZ'
A B C E F G K L M N O P S V Z←1+⍳15
\nwused{\\{NW2YR5B-1p0Y9w-1}}\nwendcode{}\nwbegindocs{24}\nwdocspar

\subsection{Converters between parent and depth vectors}

\nwenddocs{}\nwbegincode{25}\sublabel{NW2YR5B-3dpNce-1}\nwmargintag{{\nwtagstyle{}\subpageref{NW2YR5B-3dpNce-1}}}\moddef{Converters between parent and depth vectors~{\nwtagstyle{}\subpageref{NW2YR5B-3dpNce-1}}}\endmoddef\nwstartdeflinemarkup\nwusesondefline{\\{NW2YR5B-1p0Y9w-1}}\nwenddeflinemarkup
P2D←\{z←⍪⍳≢⍵ ⋄ d←⍵≠,z ⋄ _←\{p⊣d+←⍵≠p←⍺[z,←⍵]\}⍣≡⍨⍵ ⋄ d(⍋(-1+d)↑⍤0 1⊢⌽z)\}
D2P←\{0=≢⍵:⍬ ⋄ p⊣2\{p[⍵]←⍺[⍺⍸⍵]\}⌿⊢∘⊂⌸⍵⊣p←⍳≢⍵\}
\nwused{\\{NW2YR5B-1p0Y9w-1}}\nwendcode{}\nwbegindocs{26}\nwdocspar

\section{Testing}

We use the \href{https://github.com/Co-dfns/APLUnit}{APLUnit}
testing framework to facilitate our testing of the Co-dfns compiler.
The test harness is designed around a testing philosophy in which we
ever only write black-box tests that work on the whole compiler
using inputs that could be created or are expected to be creatable
by end-users.
That is, we do no ``unit testing'' of our source code,
but only whole program testing.

The testing framework is provided by the {\Tt{}ut.apln\nwendquote} file,
which is not part of this literate program and so is not included in
this document.
In order to make some of the testing more convenient,
we define the function {\Tt{}\nwlinkedidentq{TEST}{NW2YR5B-4gH67U-1}\nwendquote} to run the tests
that exist in the {\Tt{}tests{\nwbackslash}\nwendquote} subdirectory.
Each of these tests has a specific number which defines the test,
and we refer to the tests by number when running them.
Both of these testing functions assume that we are running inside
of the {\Tt{}tests{\nwbackslash}\nwendquote} directory or one configured identically to it.

The {\Tt{}\nwlinkedidentq{TEST}{NW2YR5B-4gH67U-1}\nwendquote} function takes either {\Tt{}'ALL'\nwendquote} as its input or a test
number in the form of an integer.
Given an integer, we call the test matching that number in the
current working directory.

The {\Tt{}'ALL'\nwendquote} option causes {\Tt{}\nwlinkedidentq{TEST}{NW2YR5B-4gH67U-1}\nwendquote} to run all of the tests that are
defined in the current working directory.
This command is a nicety, since we can technically do all of this
by iterating the {\Tt{}\nwlinkedidentq{TEST}{NW2YR5B-4gH67U-1}\nwendquote} function over the range of test numbers,
but this would not create the aggregate statistics that we would
like to see at the end of the testing report.
By using {\Tt{}'ALL'\nwendquote} we get to see a complete summary of the
results of testing all the code,
rather than just the individual testing results on a per testing
group/number basis.

\nwenddocs{}\nwbegincode{27}\sublabel{NW2YR5B-4gH67U-1}\nwmargintag{{\nwtagstyle{}\subpageref{NW2YR5B-4gH67U-1}}}\moddef{\code{}TEST\edoc{}~{\nwtagstyle{}\subpageref{NW2YR5B-4gH67U-1}}}\endmoddef\nwstartdeflinemarkup\nwenddeflinemarkup
\nwlinkedidentc{TEST}{NW2YR5B-4gH67U-1}←\{
        #.UT.(print_passed print_summary)←1
        'ALL'≡⍵:#.UT.run './'
        path←'./t',(1 0⍕(4⍴10)⊤⍵),'_*_tests.dyalog'
        #.UT.run ⊃⊃0⎕NINFO⍠1⊢path
\}
\nwindexdefn{\nwixident{TEST}}{TEST}{NW2YR5B-4gH67U-1}\eatline
\nwnotused{[[TEST]]}\nwidentdefs{\\{{\nwixident{TEST}}{TEST}}}\nwendcode{}\nwbegindocs{28}\nwdocspar
The {\Tt{}\nwlinkedidentq{TEST}{NW2YR5B-4gH67U-1}\nwendquote} function is part of the utilities that exist outside
of the {\Tt{}\nwlinkedidentq{codfns}{NW2YR5B-1p0Y9w-1}\nwendquote} namespace, 
so we define a file for it.

\nwenddocs{}\nwbegincode{29}\sublabel{NW2YR5B-ufqpm-2}\nwmargintag{{\nwtagstyle{}\subpageref{NW2YR5B-ufqpm-2}}}\moddef{Tangle Commands~{\nwtagstyle{}\subpageref{NW2YR5B-ufqpm-1}}}\plusendmoddef\nwstartdeflinemarkup\nwusesondefline{\\{NW2YR5B-42EjwV-1}}\nwprevnextdefs{NW2YR5B-ufqpm-1}{NW2YR5B-ufqpm-3}\nwenddeflinemarkup
echo "Tangling \nwlinkedidentc{src}{NW2YR5B-3C6SQT-1}/\nwlinkedidentc{TEST}{NW2YR5B-4gH67U-1}\nwlinkedidentc{.aplf}{NW2YR5B-ufqpm-2}..."
notangle -R'[[\nwlinkedidentc{TEST}{NW2YR5B-4gH67U-1}]]' \nwlinkedidentc{codfns}{NW2YR5B-1p0Y9w-1}.nw > \nwlinkedidentc{src}{NW2YR5B-3C6SQT-1}/\nwlinkedidentc{TEST}{NW2YR5B-4gH67U-1}\nwlinkedidentc{.aplf}{NW2YR5B-ufqpm-2}
\nwindexdefn{\nwixident{TEST.aplf}}{TEST.aplf}{NW2YR5B-ufqpm-2}\eatline
\nwused{\\{NW2YR5B-42EjwV-1}}\nwidentdefs{\\{{\nwixident{TEST.aplf}}{TEST.aplf}}}\nwidentuses{\\{{\nwixident{codfns}}{codfns}}\\{{\nwixident{src}}{src}}\\{{\nwixident{TEST}}{TEST}}}\nwindexuse{\nwixident{codfns}}{codfns}{NW2YR5B-ufqpm-2}\nwindexuse{\nwixident{src}}{src}{NW2YR5B-ufqpm-2}\nwindexuse{\nwixident{TEST}}{TEST}{NW2YR5B-ufqpm-2}\nwendcode{}\nwbegindocs{30}\nwdocspar
\section{Co-dfns Compiler}

\subsection{Parser}

The first, and in many ways, the most complex element of the 
compiler is the parser.
APL has a number of unique issues when it comes to adequately 
parsing the language,
but the most important is handling the context-sensitive 
nature of parsing variables: depending on the type of a variable,
the parse tree can look very different.
To manage this, we make use of a linear, multi-pass style of 
parser in which the parsing process consists of numerous small 
passes over the input, each time refining the input into something
more like the final result.
The parser should take some input that matches the input requirements
of the {\Tt{}\nwlinkedidentq{Fix}{NW2YR5B-2o6hoR-1}\nwendquote} function and produce a suitable output AST.

$$PS :: Source → AST × ExportTypes × SymbolTable × Source$$

\noindent
We can think of the parser as starting with a forest of trees, 
each of which contains a single root node that represents a single
character in from the input source,
with all trees arranged in the source order.
During each pass of the parser, we progressively combine these 
trees into more complex trees until we end up at the end with a 
single tree per parsed module.
In other words, we take a fully flat forest of single-node trees 
and progressively increase the depth while reducing the number of 
root-nodes until we have our desired AST structure.

We divide the parsing roughly into two main phases,
the tokenization phase and the parsing phase. 
Unlike most compilers, we don't have a strict division in these 
two phases, so, as they say, think of them more like guidelines
than actual rules%
\footnote{
        \href{https://www.youtube.com/watch?v=WJVBvvS57j0}
                {https://www.youtube.com/watch?v=WJVBvvS57j0}
}.

\nwenddocs{}\nwbegincode{31}\sublabel{NW2YR5B-38DvvD-1}\nwmargintag{{\nwtagstyle{}\subpageref{NW2YR5B-38DvvD-1}}}\moddef{Parser~{\nwtagstyle{}\subpageref{NW2YR5B-38DvvD-1}}}\endmoddef\nwstartdeflinemarkup\nwusesondefline{\\{NW2YR5B-1p0Y9w-1}}\nwenddeflinemarkup
\nwlinkedidentc{PS}{NW2YR5B-38DvvD-1}←\{
        \LA{}Verify source input \code{}⍵\edoc{}, set \code{}IN\edoc{}~{\nwtagstyle{}\subpageref{NW2YR5B-13WClx-1}}\RA{}
        
        \LA{}Parsing Constants~{\nwtagstyle{}\subpageref{NW2YR5B-3563XI-1}}\RA{}
        \LA{}Line and error reporting utilities~{\nwtagstyle{}\subpageref{NW2YR5B-aELRs-1}}\RA{}

        \LA{}Tokenize input~{\nwtagstyle{}\subpageref{NW2YR5B-1192i8-1}}\RA{}
        \LA{}Parse token stream~{\nwtagstyle{}\subpageref{NW2YR5B-4CIf4o-1}}\RA{}
        
        \LA{}Adjust AST for output~{\nwtagstyle{}\subpageref{NW2YR5B-1K2n9O-1}}\RA{}
\}
\nwindexdefn{\nwixident{PS}}{PS}{NW2YR5B-38DvvD-1}\eatline
\nwused{\\{NW2YR5B-1p0Y9w-1}}\nwidentdefs{\\{{\nwixident{PS}}{PS}}}\nwendcode{}\nwbegindocs{32}\nwdocspar
When parsing, it's very helpful to have names for line endings.

\nwenddocs{}\nwbegincode{33}\sublabel{NW2YR5B-3563XI-1}\nwmargintag{{\nwtagstyle{}\subpageref{NW2YR5B-3563XI-1}}}\moddef{Parsing Constants~{\nwtagstyle{}\subpageref{NW2YR5B-3563XI-1}}}\endmoddef\nwstartdeflinemarkup\nwusesondefline{\\{NW2YR5B-38DvvD-1}}\nwenddeflinemarkup
CR LF←⎕UCS 13 10
\nwused{\\{NW2YR5B-38DvvD-1}}\nwendcode{}\nwbegindocs{34}\nwdocspar

\subsubsection{Output of the Parser}

After we finish all of our parsing,
we need to take the resulting AST and convert that into something
that is suitable for output to the rest of the system.
We do this in a few ways. 

When we finish parsing, we expect the following fields:

\begin{center}
\begin{tabular}{cl}
\toprule
Field & Description\\
\midrule
{\Tt{}d\nwendquote} & Depth vector\\
{\Tt{}t\nwendquote} & Node type\\
{\Tt{}k\nwendquote} & Node sub-class or ``kind''\\
{\Tt{}n\nwendquote} & Name/value field\\
{\Tt{}pos\nwendquote} & Starting index for source position\\
{\Tt{}end\nwendquote} & Exclussive index for source end position\\
{\Tt{}\nwlinkedidentq{xn}{NW2YR5B-1K2n9O-1}\nwendquote} & Names of top-level exported bindings\\
{\Tt{}\nwlinkedidentq{xt}{NW2YR5B-1K2n9O-1}\nwendquote} & Types of top-level exported bindings\\
{\Tt{}sym\nwendquote} & Symbol Table\\
{\Tt{}IN\nwendquote} & Canonical source code\\
\bottomrule
\end{tabular}
\par\end{center}

The {\Tt{}\nwlinkedidentq{xn}{NW2YR5B-1K2n9O-1}\nwendquote} and {\Tt{}\nwlinkedidentq{xt}{NW2YR5B-1K2n9O-1}\nwendquote} fields are not part of the AST proper, 
but form an auxiliary analysis that is exceptionally useful,
and so we include this as a part of the output of the parser.
After parsing a module, we want to extract out the top-level 
bindings and what their types are,
which we can then use to feed into things like the linker 
and other areas that might need to know what names are available
in a given module.
Top-level bindings are identified as bindings that appear as a 
part of an initialization function, also known as {\Tt{}F0\nwendquote}.

\nwenddocs{}\nwbegincode{35}\sublabel{NW2YR5B-1K2n9O-1}\nwmargintag{{\nwtagstyle{}\subpageref{NW2YR5B-1K2n9O-1}}}\moddef{Adjust AST for output~{\nwtagstyle{}\subpageref{NW2YR5B-1K2n9O-1}}}\endmoddef\nwstartdeflinemarkup\nwusesondefline{\\{NW2YR5B-38DvvD-1}}\nwprevnextdefs{\relax}{NW2YR5B-1K2n9O-2}\nwenddeflinemarkup
msk←(t=B)∧k[I@\{t[⍵]≠F\}⍣≡⍨p]=0
\nwlinkedidentc{xn}{NW2YR5B-1K2n9O-1}←(0⍴⊂''),msk⌿n ⋄ \nwlinkedidentc{xt}{NW2YR5B-1K2n9O-1}←msk⌿k
\nwindexdefn{\nwixident{xn}}{xn}{NW2YR5B-1K2n9O-1}\nwindexdefn{\nwixident{xt}}{xt}{NW2YR5B-1K2n9O-1}\eatline
\nwalsodefined{\\{NW2YR5B-1K2n9O-2}\\{NW2YR5B-1K2n9O-3}\\{NW2YR5B-1K2n9O-4}}\nwused{\\{NW2YR5B-38DvvD-1}}\nwidentdefs{\\{{\nwixident{xn}}{xn}}\\{{\nwixident{xt}}{xt}}}\nwendcode{}\nwbegindocs{36}\nwdocspar
We also want to convert the AST to an order that follows a 
depth-first, preorder traversal order, 
so that we can switch from using the parent vector to the depth
vector.
We use this output as our main output because it is space efficient
for storage, and it works well as a canonical form to use.
Because applications may want to only use the parser and not the 
rest of the compiler, 
we want to choose an output format that is suitable for external 
as well as internal use.
This has some performance overheads,
but it is probably worth it regardless,
as reordering at this point to allow a depth vector enables some 
nice assumptions in the rest of the compiler.
We use the {\Tt{}P2D\nwendquote} utility to reorder all of our AST columns.
Note that things like the exported bindings and the symbol table 
are not strictly part of the AST structure, because they are of a 
different length and type than the other columns.

\nwenddocs{}\nwbegincode{37}\sublabel{NW2YR5B-1K2n9O-2}\nwmargintag{{\nwtagstyle{}\subpageref{NW2YR5B-1K2n9O-2}}}\moddef{Adjust AST for output~{\nwtagstyle{}\subpageref{NW2YR5B-1K2n9O-1}}}\plusendmoddef\nwstartdeflinemarkup\nwusesondefline{\\{NW2YR5B-38DvvD-1}}\nwprevnextdefs{NW2YR5B-1K2n9O-1}{NW2YR5B-1K2n9O-3}\nwenddeflinemarkup
d i←P2D p ⋄ d n t k pos end I∘⊢←⊂i
\nwused{\\{NW2YR5B-38DvvD-1}}\nwendcode{}\nwbegindocs{38}\nwdocspar

There is an inefficiency in the AST representation at this point,
where the {\Tt{}n\nwendquote} field contains character vectors.
This inefficiency was necessary while building up the AST because
we were not sure what symbols would be created
before we parsed them,
but at this point, we know the full set of symbols that we have in 
the AST.
This means that we can convert the {\Tt{}n\nwendquote} field to a symbol table 
representation.
In this case, we want the {\Tt{}n\nwendquote} field to pair with a {\Tt{}sym\nwendquote} list
that contains all the unique symbols in the source.
We want {\Tt{}old{\_}n≡sym[|new{\_}n]\nwendquote} to hold for this new {\Tt{}n\nwendquote} field.
In other words, we want the new {\Tt{}n\nwendquote} field to contain negative 
integers whose magnitudes are valid indices into the {\Tt{}sym\nwendquote} 
symbol table.
This means that there is only one character vector per unique symbol
or numeric literal in the source code, which can greatly reduce 
memory usage.
Moreover, it is much faster to compare symbols that are represented by 
numeric index rather than character vector.
Most of the work we expect to be done on the {\Tt{}n\nwendquote} field, so that we 
never have to pull in {\Tt{}sym\nwendquote} unless we want to know the actual value 
of the symbol.
This actually mimics the feature of symbols in other languages
like Scheme,
but it comes with an additional efficiency benefit in that we do not
require the use of a full generalized pointer to represent a symbol
if we have fewer symbols. 
This means that we are very likely only going to need a single byte 
or a couple of bytes per symbol to represent it in the {\Tt{}n\nwendquote} field.

The choice to make all of our symbols negative in value is somewhat 
strange, but we have a good reason for doing so.
The {\Tt{}n\nwendquote} field is a single field that we use to contain general 
data for every node, and as such, it represents a sort of union 
type of all sorts of different data.
In particular, we also want to be able to support using the {\Tt{}n\nwendquote} 
field to point to other nodes in the AST, which is a feature we 
rely heavily on in the compiler transformations.
However, this feature would conflict with using the {\Tt{}n\nwendquote} field as an 
index into the {\Tt{}sym\nwendquote} table, rather than as an index into the AST.
By making symbol pointers negative, we put them into a separate 
space than the positive AST node pointers, allowing us to store 
both pointers in the same field. This may seem like a little bit of 
a strange hack, but it actually makes reasoning about things a little 
easier, because we can tend to think of {\Tt{}n\nwendquote} as a name, even if that 
name is pointing to an AST or a symbol, and avoids needless space 
duplication or the need to remember to update multiple fields that are 
only relevant for some nodes.

We map the $0$th index to be a null or empty symbol. 
We also want to reserve the first four symbol slots $[1,4]$
so that they will \emph{always} refer to the same symbols, 
namely, {\Tt{}⍵\nwendquote}, {\Tt{}⍺\nwendquote}, {\Tt{}⍺⍺\nwendquote}, and {\Tt{}⍵⍵\nwendquote}. 

This gives us the following definitions for {\Tt{}sym\nwendquote} and {\Tt{}n\nwendquote}.

\nwenddocs{}\nwbegincode{39}\sublabel{NW2YR5B-1K2n9O-3}\nwmargintag{{\nwtagstyle{}\subpageref{NW2YR5B-1K2n9O-3}}}\moddef{Adjust AST for output~{\nwtagstyle{}\subpageref{NW2YR5B-1K2n9O-1}}}\plusendmoddef\nwstartdeflinemarkup\nwusesondefline{\\{NW2YR5B-38DvvD-1}}\nwprevnextdefs{NW2YR5B-1K2n9O-2}{NW2YR5B-1K2n9O-4}\nwenddeflinemarkup
sym←∪('')(,'⍵')(,'⍺')'⍺⍺' '⍵⍵',n
n←-sym⍳n
\nwused{\\{NW2YR5B-38DvvD-1}}\nwendcode{}\nwbegindocs{40}\nwdocspar

Finally, we want to return our AST structure in a meaningful way.
Logically, we have the AST proper, which consists of these fields:

\begin{verbatim}
d t k n pos end
\end{verbatim}

\noindent
The above fields are returned as an inverted table,
where each column is a vector of the same length.
We also want to return the variable environment,
which gives the names of our top-level bindings and their types,
also as an inverted table.
Finally, we must return a canonical representation of the source 
code that is suitable as an indexing target for the {\Tt{}pos\nwendquote} and {\Tt{}end\nwendquote}
fields, as well as the symbol table.
Thus, we have a four element vector as the return value:

\begin{verbatim}
AST TopBindingTypes SymbolTable InputSource
\end{verbatim}

\noindent
Which gives us the following return value.

\nwenddocs{}\nwbegincode{41}\sublabel{NW2YR5B-1K2n9O-4}\nwmargintag{{\nwtagstyle{}\subpageref{NW2YR5B-1K2n9O-4}}}\moddef{Adjust AST for output~{\nwtagstyle{}\subpageref{NW2YR5B-1K2n9O-1}}}\plusendmoddef\nwstartdeflinemarkup\nwusesondefline{\\{NW2YR5B-38DvvD-1}}\nwprevnextdefs{NW2YR5B-1K2n9O-3}{\relax}\nwenddeflinemarkup
(d t k n pos end)(\nwlinkedidentc{xn}{NW2YR5B-1K2n9O-1} \nwlinkedidentc{xt}{NW2YR5B-1K2n9O-1})sym IN
\nwused{\\{NW2YR5B-38DvvD-1}}\nwidentuses{\\{{\nwixident{xn}}{xn}}\\{{\nwixident{xt}}{xt}}}\nwindexuse{\nwixident{xn}}{xn}{NW2YR5B-1K2n9O-4}\nwindexuse{\nwixident{xt}}{xt}{NW2YR5B-1K2n9O-4}\nwendcode{}\nwbegindocs{42}\nwdocspar

\subsubsection{Handling Parsing Errors}

\nwenddocs{}\nwbegincode{43}\sublabel{NW2YR5B-aELRs-1}\nwmargintag{{\nwtagstyle{}\subpageref{NW2YR5B-aELRs-1}}}\moddef{Line and error reporting utilities~{\nwtagstyle{}\subpageref{NW2YR5B-aELRs-1}}}\endmoddef\nwstartdeflinemarkup\nwusesondefline{\\{NW2YR5B-38DvvD-1}}\nwenddeflinemarkup
\nwlinkedidentc{linestarts}{NW2YR5B-aELRs-1}←(⍸1⍪2>⌿IN∊CR LF)⍪≢IN
\nwlinkedidentc{mkdm}{NW2YR5B-aELRs-1}←\{⍺←2 ⋄ line←\nwlinkedidentc{linestarts}{NW2YR5B-aELRs-1}⍸⍵ ⋄ no←'[',(⍕1+line),'] '
        i←(~IN[i]∊CR LF)⌿i←beg+⍳\nwlinkedidentc{linestarts}{NW2YR5B-aELRs-1}[line+1]-beg←\nwlinkedidentc{linestarts}{NW2YR5B-aELRs-1}[line]
        (⎕EM ⍺)(no,IN[i])(' ^'[i∊⍵],⍨' '⍴⍨≢no)\}
\nwlinkedidentc{quotelines}{NW2YR5B-aELRs-1}←\{
        lines←∪\nwlinkedidentc{linestarts}{NW2YR5B-aELRs-1}⍸⍵
        nos←(1 0⍴⍨2×≢lines)⍀'[',(⍕⍪1+lines),⍤1⊢'] '
        beg←\nwlinkedidentc{linestarts}{NW2YR5B-aELRs-1}[lines] ⋄ end←\nwlinkedidentc{linestarts}{NW2YR5B-aELRs-1}[lines+1]
        m←∊∘⍵¨i←beg+⍳¨end-beg
        ¯1↓∊nos,(~∘CR LF¨⍪,(IN∘I¨i),⍪' ▔'∘I¨m),CR\}
\nwlinkedidentc{SIGNAL}{NW2YR5B-aELRs-1}←\{⍺←2 '' ⋄ en msg←⍺ ⋄ EN∘←en ⋄ DM∘←en \nwlinkedidentc{mkdm}{NW2YR5B-aELRs-1} ⊃⍵
        dmx←('EN' en)('Category' 'Compiler')('Vendor' 'Co-dfns')
        dmx,←⊂'Message'(msg,CR,\nwlinkedidentc{quotelines}{NW2YR5B-aELRs-1} ⍵)
        ⎕\nwlinkedidentc{SIGNAL}{NW2YR5B-aELRs-1}⊂dmx\}
\nwindexdefn{\nwixident{SIGNAL}}{SIGNAL}{NW2YR5B-aELRs-1}\nwindexdefn{\nwixident{quotelines}}{quotelines}{NW2YR5B-aELRs-1}\nwindexdefn{\nwixident{mkdm}}{mkdm}{NW2YR5B-aELRs-1}\nwindexdefn{\nwixident{linestarts}}{linestarts}{NW2YR5B-aELRs-1}\eatline
\nwused{\\{NW2YR5B-38DvvD-1}}\nwidentdefs{\\{{\nwixident{linestarts}}{linestarts}}\\{{\nwixident{mkdm}}{mkdm}}\\{{\nwixident{quotelines}}{quotelines}}\\{{\nwixident{SIGNAL}}{SIGNAL}}}\nwendcode{}\nwbegindocs{44}\nwdocspar
\subsubsection{Tokenizing the Input}

\nwenddocs{}\nwbegincode{45}\sublabel{NW2YR5B-1192i8-1}\nwmargintag{{\nwtagstyle{}\subpageref{NW2YR5B-1192i8-1}}}\moddef{Tokenize input~{\nwtagstyle{}\subpageref{NW2YR5B-1192i8-1}}}\endmoddef\nwstartdeflinemarkup\nwusesondefline{\\{NW2YR5B-38DvvD-1}}\nwenddeflinemarkup
⍝ Group input into lines as a nested vector
         pos←(⍳≢IN)⊆⍨~IN∊CR LF

⍝ Mask strings
         0≠≢lin←⍸⊃∘⌽¨msk←≠⍀¨''''=IN∘I¨pos:\{
                 EM←'SYNTAX ERROR: UNBALANCED STRING',('S'⌿⍨2≤≢lin),CR
                 EM,←\nwlinkedidentc{quotelines}{NW2YR5B-aELRs-1} ∊(msk⌿¨pos)[lin]
                 EM ⎕\nwlinkedidentc{SIGNAL}{NW2YR5B-aELRs-1} 2\}⍬

⍝ Remove comments
         pos msk⌿¨⍨←⊂∧⍀¨(~msk←msk∨¯1⌽¨msk)⍲'⍝'=IN∘I¨pos

⍝ Remove leading and trailing whitespace
         WS←⎕UCS 9 32 ⋄ pos msk⌿¨⍨←⊂~(∧⍀∨∧⍀U⌽)∘(WS∊⍨IN∘I)¨pos

⍝ Flatten and separate lines and ⋄ with Z type
         t←⊃0⍴⊂pos ⋄ t pos msk(∊,∘⍪⍨)←Z(⊃¨pos)0 ⋄ t[⍸'⋄'=IN[pos]]←Z

⍝ Tokenize Strings
         end←1+pos ⋄ t[i←⍸2<⌿0⍪msk]←C ⋄ end[i]←end[⍸2>⌿msk⍪0]
         t pos end⌿⍨←⊂(t≠0)∨~msk

⍝ Verify that all open characters are part of the valid character set
         alp←'ABCDEFGHIJKLMNOPQRSTUVWXYZ_abcdefghijklmnopqrstuvwxyz'
         alp,←'ÀÁÂÃÄÅÆÇÈÉÊËÌÍÎÏÐÑÒÓÔÕÖØÙÚÛÜÝßàáâãäåæçèéêëìíîïðñòóôõöøùúûüþ'
         alp,←'∆⍙ⒶⒷⒸⒹⒺⒻⒼⒽⒾⒿⓀⓁⓂⓃⓄⓅⓆⓇⓈⓉⓊⓋⓌⓍⓎⓏ'
         num←⎕D
         synb←'¯[]\{\}()'':⍺⍵⋄;'
         syna←'⍬⎕⍞#'
         prmfs←'+-×÷|⌈⌊*⍟○!?~∧∨⍲⍱<≤=>≥≠≡≢⍴,⍪⌽⊖⍉↑↓⊂⊆⊃∊⍷∩∪⍳⍸⌷⍋⍒⍎⍕⊥⊤⊣⊢⌹∇←→'
         prmdo←'∘.⍣⍠⌺⍤⍥@' ⋄ prmmo←'¨⍨&⌶⌸' ⋄ prmfo←'/⌿\\⍀'
         prms←prmfs,prmdo,prmmo,prmfo
         x←' '@\{t≠0\}IN[pos] ⍝ The spaces produce nice invariants
         ∨⌿msk←~x∊alp,num,syna,synb,prms,WS:\{
                 EM←'SYNTAX ERROR: INVALID CHARACTER(S) IN SOURCE',CR
                 EM,←\nwlinkedidentc{quotelines}{NW2YR5B-aELRs-1} ⍸msk
                 EM ⎕\nwlinkedidentc{SIGNAL}{NW2YR5B-aELRs-1} 2\}⍬

⍝ Tokenize numbers
         _←\{dm[⍵]←∧⍀dm[⍵]\}¨(dm∨x∊alp)⊆⍳≢dm←x∊num
         dm∨←('.'=x)∧(¯1⌽dm)∨1⌽dm
         dm∨←('¯'=x)∧1⌽dm
         dm∨←(x∊'EeJj')∧(¯1⌽dm)∧1⌽dm
         ∨⌿msk←(dm=0)∧x='¯':2'ORPHANED ¯'SIGNAL pos⌿⍨msk
         ∨⌿\{1<+⌿⍵='j'\}¨dp←⎕C¨dm⊆x:'MULTIPLE J IN NUMBER'⎕\nwlinkedidentc{SIGNAL}{NW2YR5B-aELRs-1} 2
         ∨⌿\{1<+⌿⍵='e'\}¨dp←⊃⍪/\{⍵⊆⍨⍵≠'j'\}¨dp:'MULTIPLE E IN NUMBER'⎕\nwlinkedidentc{SIGNAL}{NW2YR5B-aELRs-1} 2
         ∨⌿'e'=⊃¨dp:'MISSING MANTISSA'⎕\nwlinkedidentc{SIGNAL}{NW2YR5B-aELRs-1} 2
         ∨⌿'e'=⊃∘⌽¨dp:'MISSING EXPONENT'⎕\nwlinkedidentc{SIGNAL}{NW2YR5B-aELRs-1} 2
         mn ex←↓⍉↑\{2↑(⍵⊆⍨⍵≠'e'),⊂''\}¨dp
         ∨⌿\{1<+⌿'.'=⍵\}¨mn,ex:'MULTIPLE . IN NUMBER'⎕\nwlinkedidentc{SIGNAL}{NW2YR5B-aELRs-1} 2
         ∨⌿'.'∊¨ex:'REAL NUMBER IN EXPONENT'⎕\nwlinkedidentc{SIGNAL}{NW2YR5B-aELRs-1} 2
         ∨⌿\{∨⌿1↓⍵∊'¯'\}¨mn,ex:'MISPLACED ¯'⎕\nwlinkedidentc{SIGNAL}{NW2YR5B-aELRs-1} 2
         t[i←⍸2<⌿0⍪dm]←N ⋄ end[i]←end⌿⍨2>⌿dm⍪0

⍝ Tokenize Variables
         t[i←⍸2<⌿0⍪vm←(~dm)∧x∊alp,num]←V ⋄ end[i]←end⌿⍨2>⌿vm⍪0

⍝ Tokenize ⍺, ⍵ formals
         fm←\{mm←⌽⊃(>∘⊃,⊢)⌿⌽m←⍺=' ',⍵ ⋄ 1↓¨(mm∧~m1)(mm∧m1←1⌽m)\}
         am aam←'⍺'fm x ⋄ wm wwm←'⍵'fm x
         ((am∨wm)⌿t)←A ⋄ ((aam∨wwm)⌿t)←P ⋄ ((aam∨wwm)⌿end)←end⌿⍨¯1⌽aam∨wwm

⍝ Tokenize Primitives, Atoms
         t[⍸(~dm)∧x∊prms]←P ⋄ t[⍸x∊syna]←A

⍝ Compute dfns regions and type, include \} as a child
         t[⍸'\{'=x]←F ⋄ 0≠⊃d←¯1⌽+⍀1 ¯1 0['\{\}'⍳x]:'UNBALANCED DFNS'⎕\nwlinkedidentc{SIGNAL}{NW2YR5B-aELRs-1} 2

⍝ Check for out of context dfns formals
         ∨⌿(d=0)∧(t=P)∧IN[pos]∊'⍺⍵':'DFN FORMAL REFERENCED OUTSIDE DFNS'⎕\nwlinkedidentc{SIGNAL}{NW2YR5B-aELRs-1} 2

⍝ Compute trad-fns regions
         ∨⌿Z≠t⌿⍨1⌽msk←(d=0)∧'∇'=x:'TRAD-FNS START/END LINES MUST BEGIN WITH ∇'⎕\nwlinkedidentc{SIGNAL}{NW2YR5B-aELRs-1} 2
         0≠⊃tm←¯1⌽≠⍀(d=0)∧'∇'=x:'UNBALANCED TRAD-FNS'⎕\nwlinkedidentc{SIGNAL}{NW2YR5B-aELRs-1} 2
         ∨⌿Z≠t⌿⍨⊃1 ¯1∨.⌽⊂(2>⌿tm)⍪0:'TRAD-FNS END LINE MUST CONTAIN ∇ ALONE'⎕\nwlinkedidentc{SIGNAL}{NW2YR5B-aELRs-1} 2

⍝ Identify Label colons versus others
         t[⍸tm∧(d=0)∧∊((~⊃)∧(<⍀∨⍀))¨':'=(t=Z)⊂IN[pos]]←L

⍝ Tokenize Keywords
         ki←⍸(t=0)∧(d=0)∧(':'=IN[pos])∧1⌽t=V
         t[ki]←K ⋄ end[ki]←end[ki+1] ⋄ t[ki+1]←0
         ERR←'EMPTY COLON IN NON-DFNS CONTEXT, EXPECTED LABEL OR KEYWORD'
         ∨⌿(t=0)∧(d=0)∧':'=IN[pos]:ERR ⎕\nwlinkedidentc{SIGNAL}{NW2YR5B-aELRs-1} 2

⍝ Tokenize System Variables
         si←⍸('⎕'=IN[pos])∧1⌽t=V
         t[si]←S ⋄ end[si]←end[si+1] ⋄ t[si+1]←0

⍝ Delete all characters we no longer need from the tree
         d tm t pos end(⌿⍨)←⊂(t≠0)∨x∊'()[]\{\}:;'

⍝ Tokenize Labels
         ERR←'LABEL MUST CONSIST OF A SINGLE NAME'
         ∨⌿(Z≠t[li-1])∨(V≠t[li←⍸1⌽msk←t=L]):ERR ⎕\nwlinkedidentc{SIGNAL}{NW2YR5B-aELRs-1} 2
         t[li]←L ⋄ end[li]←end[li+1]
         d tm t pos end(⌿⍨)←⊂~msk
\nwused{\\{NW2YR5B-38DvvD-1}}\nwidentuses{\\{{\nwixident{quotelines}}{quotelines}}\\{{\nwixident{SIGNAL}}{SIGNAL}}}\nwindexuse{\nwixident{quotelines}}{quotelines}{NW2YR5B-1192i8-1}\nwindexuse{\nwixident{SIGNAL}}{SIGNAL}{NW2YR5B-1192i8-1}\nwendcode{}\nwbegindocs{46}\nwdocspar




\subsubsection{Parsing Token Stream}

\nwenddocs{}\nwbegincode{47}\sublabel{NW2YR5B-4CIf4o-1}\nwmargintag{{\nwtagstyle{}\subpageref{NW2YR5B-4CIf4o-1}}}\moddef{Parse token stream~{\nwtagstyle{}\subpageref{NW2YR5B-4CIf4o-1}}}\endmoddef\nwstartdeflinemarkup\nwusesondefline{\\{NW2YR5B-38DvvD-1}}\nwenddeflinemarkup
⍝ Now that all compound data is tokenized, reify n field before tree-building
         n←\{1↓⍎¨'0',⍵\}@\{t=N\}(⊂'')@\{t∊Z F\}1 ⎕C@\{t∊K S\}IN∘I¨pos+⍳¨end-pos

⍝ Verify that keywords are defined and scoped correctly
         KW←'NAMESPACE' 'ENDNAMESPACE' 'END' 'IF' 'ELSEIF' 'ANDIF' 'ORIF' 'ENDIF'
         KW,←'WHILE' 'ENDWHILE' 'UNTIL' 'REPEAT' 'ENDREPEAT' 'LEAVE' 'FOR' 'ENDFOR'
         KW,←'IN' 'INEACH' 'SELECT' 'ENDSELECT' 'CASE' 'CASELIST' 'ELSE' 'WITH'
         KW,←'ENDWITH' 'HOLD' 'ENDHOLD' 'TRAP' 'ENDTRAP' 'GOTO' 'RETURN' 'CONTINUE'
         KW,←'SECTION' 'ENDSECTION' 'DISPOSABLE' 'ENDDISPOSABLE'
         KW,¨⍨←':'
         msk←~KW∊⍨kws←n⌿⍨km←t=K
         ∨⌿msk:('UNRECOGNIZED KEYWORD ',kws⊃⍨⊃⍸msk)⎕\nwlinkedidentc{SIGNAL}{NW2YR5B-aELRs-1} 2
         msk←kws∊':NAMESPACE' ':ENDNAMESPACE'
         ∨⌿msk∧km⌿tm:'NAMESPACE SCRIPTS MUST APPEAR AT THE TOP LEVEL'⎕\nwlinkedidentc{SIGNAL}{NW2YR5B-aELRs-1} 2
         msk←kws∊KW~':NAMESPACE' ':ENDNAMESPACE' ':SECTION' ':ENDSECTION'
         ∨⌿msk←msk∧~km⌿tm:\{msg←2'STRUCTURED STATEMENTS MUST APPEAR WITHIN TRAD-FNS'
                 msg \nwlinkedidentc{SIGNAL}{NW2YR5B-aELRs-1} ∊\{x+⍳end[⍵]-x←pos[⍵]\}¨⍸km⍀msk\}⍬

⍝ Verify system variables are valid
         SYSV←,¨'Á' 'A' 'AI' 'AN' 'AV' 'AVU' 'BASE' 'CT' 'D' 'DCT' 'DIV' 'DM'
         SYSV,←,¨'DMX' 'EXCEPTION' 'FAVAIL' 'FNAMES' 'FNUMS' 'FR' 'IO' 'LC' 'LX'
         SYSV,←,¨'ML' 'NNAMES' 'NNUMS' 'NSI' 'NULL' 'PATH' 'PP' 'PW' 'RL' 'RSI'
         SYSV,←,¨'RTL' 'SD' 'SE' 'SI' 'SM' 'STACK' 'TC' 'THIS' 'TID' 'TNAME' 'TNUMS'
         SYSV,←,¨'TPOOL' 'TRACE' 'TRAP' 'TS' 'USING' 'WA' 'WSID' 'WX' 'XSI'
         SYSF←,¨'ARBIN' 'ARBOUT' 'AT' 'C' 'CLASS' 'CLEAR' 'CMD' 'CONV' 'CR' 'CS' 'CSV'
         SYSF,←,¨'CY' 'DF' 'DL' 'DQ' 'DR' 'DT' 'ED' 'EM' 'EN' 'EX' 'EXPORT'
         SYSF,←,¨'FAPPEND' 'FCHK' 'FCOPY' 'FCREATE' 'FDROP' 'FERASE' 'FFT' 'IFFT'
         SYSF,←,¨'FHIST' 'FHOLD' 'FIX' 'FLIB' 'FMT' 'FPROPS' 'FRDAC' 'FRDCI' 'FREAD'
         SYSF,←,¨'FRENAME' 'FREPLACE' 'FRESIZE' 'FSIZE' 'FSTAC' 'FSTIE' 'FTIE'
         SYSF,←,¨'FUNTIE' 'FX' 'INSTANCES' 'JSON' 'KL' 'LOAD' 'LOCK' 'MAP' 'MKDIR'
         SYSF,←,¨'MONITOR' 'NA' 'NAPPEND' 'NC' 'NCOPY' 'NCREATE' 'NDELETE' 'NERASE'
         SYSF,←,¨'NEW' 'NEXISTS' 'NGET' 'NINFO' 'NL' 'NLOCK' 'NMOVE' 'NPARTS'
         SYSF,←,¨'NPUT' 'NQ' 'NR' 'NREAD' 'NRENAME' 'NREPLACE' 'NRESIZE' 'NS'
         SYSF,←,¨'NSIZE' 'NTIE' 'NUNTIE' 'NXLATE' 'OFF' 'OR' 'PFKEY' 'PROFILE'
         SYSF,←,¨'REFS' 'SAVE' 'SH' 'SHADOW' 'SIGNAL' 'SIZE' 'SR' 'SRC' 'STATE'
         SYSF,←,¨'STOP' 'SVC' 'SVO' 'SVQ' 'SVR' 'SVS' 'TCNUMS' 'TGET' 'TKILL' 'TPUT'
         SYSF,←,¨'TREQ' 'TSYNC' 'UCS' 'VR' 'VFI' 'WC' 'WG' 'WN' 'WS' 'XML' 'XT'
         SYSD←,¨'OPT' 'R' 'S'
         ∨⌿msk←(t=S)∧~n∊'⎕',¨SYSV,SYSF,SYSD:\{
                 ERR←2'INVALID SYSTEM VARIABLE, FUNCTION, OR OPERATOR'
                 ERR \nwlinkedidentc{SIGNAL}{NW2YR5B-aELRs-1}∊pos[⍵]\{⍺+⍳⍵-⍺\}¨end[⍵]
         \}⍸msk

⍝ Compute parent vector from d
         p←D2P d

⍝ Compute nameclass of dfns
         k←2×t∊F ⋄ k[∪p⌿⍨(t=P)∧n∊⊂'⍺⍺']←3 ⋄ k[∪p⌿⍨(t=P)∧n∊⊂'⍵⍵']←4

⍝ We will often wrap a set of nodes as children under a Z node
         gz←\{z←⍵↑⍨-0≠≢⍵ ⋄ ks←¯1↓⍵
                 t[z]←Z ⋄ p[ks]←⊃z ⋄ pos[z]←pos[⊃⍵] ⋄ end[z]←end[⊃⌽z,ks] ⋄ z\}

⍝ Nest top-level root lines as Z nodes
         _←(gz 1⌽⊢)¨(t[i]=Z)⊂i←⍸d=0
         'Non-Z top-level node'assert t[⍸p=⍳≢p]=Z:

⍝ Nest all dfns expression bodies as Z nodes
         _←p[i]\{end[⍺]←end[⊃⌽⍵] ⋄ gz¨⍵⊂⍨1,¯1↓t[⍵]=Z\}⌸i←⍸t[p]=F
         'Non-Z dfns body node'assert t[⍸t[p]=F]=Z:

⍝ Drop/eliminate any Z nodes that are empty or blank
         _←p[i]\{msk[⍺,⍵]←~∧⌿IN[pos[⍵]]∊WS\}⌸i←⍸(t[p]=Z)∧p≠⍳≢p⊣msk←t≠Z
         tm n t k pos end(⌿⍨)←⊂msk ⋄ p←(⍸~msk)(⊢-1+⍸)msk⌿p

⍝ Parse Keyword structures
         nss←n∊⊂':NAMESPACE' ⋄ nse←n∊⊂':ENDNAMESPACE'
         ERR←':NAMESPACE KEYWORD MAY ONLY APPEAR AT BEGINNING OF A LINE'
         Z∨.≠t⌿⍨1⌽nss:ERR ⎕\nwlinkedidentc{SIGNAL}{NW2YR5B-aELRs-1} 2
         ERR←'NAMESPACE DECLARATION MAY HAVE ONLY A NAME OR BE EMPTY'
         ∨⌿(Z≠t⌿⍨¯1⌽nss)∧(V≠t⌿⍨¯1⌽nss)∨Z≠t⌿⍨¯2⌽nss:ERR ⎕\nwlinkedidentc{SIGNAL}{NW2YR5B-aELRs-1} 2
         ERR←':ENDNAMESPACE KEYWORD MUST APPEAR ALONE ON A LINE'
         ∨⌿Z≠t⌿⍨⊃1 ¯1∨.⌽⊂nse:ERR ⎕\nwlinkedidentc{SIGNAL}{NW2YR5B-aELRs-1} 2
         t[nsi←⍸1⌽nss]←M ⋄ t[nei←⍸1⌽nse]←-M
         n[i]←n[1+i←⍸(t=M)∧V=1⌽t] ⋄ end[nsi]←end[nei]
         x←⍸p=⍳≢p ⋄ d←+⍀(t[x]=M)+-t[x]=-M
         0≠⊃⌽d:':NAMESPACE KEYWORD MISSING :ENDNAMESPACE PAIR'⎕\nwlinkedidentc{SIGNAL}{NW2YR5B-aELRs-1} 2
         p[x]←x[D2P ¯1⌽d]

⍝ Delete unnecessary namespace nodes from the tree, leave only M's
         msk←~nss∨((¯1⌽nss)∧t=V)∨nse∨1⌽nse
         t k n pos end⌿⍨←⊂msk ⋄ p←(⍸~msk)(⊢-1+⍸)msk⌿p

⍝ PARSE LABELS ∘∘∘

⍝ Map guard statements to (G (Z ...) (Z ...))
         _←p[i]\{
                 0=+⌿m←':'=IN[pos[⍵]]:⍬
                 ⊃m:'EMPTY GUARD \nwlinkedidentc{TEST}{NW2YR5B-4gH67U-1} EXPRESSION'⎕\nwlinkedidentc{SIGNAL}{NW2YR5B-aELRs-1} 2
                 1<+⌿m:'TOO MANY GUARDS'⎕\nwlinkedidentc{SIGNAL}{NW2YR5B-aELRs-1} 2
                 t[⍺]←G ⋄ p[ti←gz⊃tx cq←2↑(⊂⍬)⍪⍨⍵⊂⍨1,¯1↓m]←⍺ ⋄ k[ti]←1
                 ci←≢p ⋄ p,←⍺ ⋄ t k pos end⍪←0 ⋄ n,←⊂'' ⋄ k[gz cq,ci]←1
         0\}⌸i←⍸t[p[p]]=F

⍝ Parse brackets and parentheses into ¯1 and Z nodes
         _←p[i]\{
                 x←IN[pos[⍵]] ⋄ bd←+⍀bm←(bo←'['=x)+-bc←']'=x ⋄ pd←+⍀pm←(po←'('=x)+-pc←')'=x
                 0≠⊃⌽bd:2'UNBALANCED BRACKETS'SIGNAL pos[⍵]\{x+⍳(⌈⌿⍵)-x←⌊⌿⍺\}⍥\{⍵⌿⍨0≠bd\}end[⍵]
                 0≠⊃⌽pd:2'UNBALANCED PARENTHESES'SIGNAL pos[⍵]\{x+⍳(⌈⌿⍵)-x←⌊⌿⍺\}⍥\{⍵⌿⍨0≠pd\}end[⍵]
                 (po⌿bd)∨.≠⌽pc⌿bd:'OVERLAPPING BRACKETS AND PARENTHESES'⎕\nwlinkedidentc{SIGNAL}{NW2YR5B-aELRs-1} 2
                 p[⍵]←(⍺,⍵)[1+¯1@\{⍵=⍳≢⍵\}D2P +⍀¯1⌽bm+pm] ⋄ t[bo⌿⍵]←¯1 ⋄ t[po⌿⍵]←Z
                 end[po⌿⍵]←end[⌽pc⌿⍵] ⋄ end[bo⌿⍵]←end[⌽bc⌿⍵]
         0\}⌸i←⍸(t[p]=Z)∧p≠⍳≢p
         t k n pos end⌿⍨←⊂msk←~IN[pos]∊')' ⋄ p←(⍸~msk)(⊢-1+⍸)msk⌿p

⍝ Convert semi-colon indexing into Z nodes in the ¯1 nodes
         _←p[i]\{k[z←⊃⍪⌿gz¨g←⍵⊂⍨¯1⌽IN[pos[⍵]]∊';]']←1 ⋄ t[z]←Z P[1=≢¨g]\}⌸i←⍸t[p]=¯1

⍝ Mark bindable nodes
         bm←(t=V)∨(t=A)∧n∊,¨'⎕⍞'
         bm←\{bm⊣p[i]\{bm[⍺]←(V ¯1≡t[⍵])∨∧⌿bm[⍵]\}⌸i←⍸(~bm[p])∧t[p]=Z\}⍣≡bm

⍝ Binding nodes
         _←p[i]\{
                 t[⍵⌿⍨(n[⍵]∊⊂,'←')∧0,¯1↓bm[⍵]]←B
                 b v←\{(⊃¨x)(1↓¨x←⍵⌿⍨\{t[⊃⍵]=B\}¨⍵)\}¯1⌽¨⍵⊂⍨1,¯1↓t[⍵]∊P B
                 ∨⌿~bm[∊v]:'CANNOT BIND ASSIGNMENT VALUE'⎕\nwlinkedidentc{SIGNAL}{NW2YR5B-aELRs-1} 2
                 p[⍵]←(⍺,b)[0,¯1↓+⍀t[⍵]=B]
                 n[b]←n[∊v] ⋄ t[∊v]←¯7 ⋄ pos[b]←pos[∊v] ⋄ end[b]←end[⊃⌽⍵]
         0\}⌸i←⍸(t[p]=Z)∧p≠⍳≢p
         t k n pos end⌿⍨←⊂msk←t≠¯7 ⋄ p←(⍸~msk)(⊢-1+⍸)msk⌿p

⍝ Mark unambiguous primitive kinds
         k[⍸(t=S)∧n∊'⎕',¨SYSV]←1 ⋄ k[⍸(t=S)∧n∊'⎕',¨SYSF]←2 ⋄ k[⍸(t=S)∧n∊'⎕',¨SYSD]←4
         t[⍸t=S]←P
         k[⍸t∊A C N]←1 ⋄ k[⍸n∊,¨prmfs]←2 ⋄ k[⍸n∊,¨prmmo]←3 ⋄ k[⍸n∊,¨prmdo]←4
         k[⍸n∊,¨prmfo]←5
         k[i←⍸msk←(n∊⊂,'∘')∧1⌽n∊⊂,'.']←3 ⋄ end[i]←end[i+1] ⋄ n[i]←⊂,'∘.'
         t k n pos end⌿⍨←⊂msk←~¯1⌽msk ⋄ p←(⍸~msk)(⊢-1+⍸)msk⌿p

⍝ Anchor variables to earliest binding in matching frame
         rf←¯1@\{~t[⍵]∊F G M\}p[rz←I@\{~(t[⍵]=Z)∧(t[p[⍵]]∊F G M)∨p[⍵]=⍵\}⍣≡⍨p]
         rf[i]←p[i←⍸t=G] ⋄ rz[i]←i ⋄ rf←rf I@\{rz∊p[i]⊢∘⊃⌸i←⍸t[p]=G\}rf
         mk←\{⍺[⍵],⍪n[⍵]\}
         fr←rf mk⊢fb←fb[⍳⍨rf mk⊢fb←fb I∘(⍳⍨)U⊖rz mk⊢fb←⍸t=B] ⋄ fb,←¯1
         vb←fb[fr⍳rf mk i]@(i←⍸t=V)⊢¯1⍴⍨≢p
         vb[i⌿⍨(rz[i]<rz[b])∨(rz[i]=rz[b])∧i≥b←vb[i←i⌿⍨vb[i]≠¯1]]←¯1
         _←\{z/⍨¯1=vb[1⌷z]←fb[fr⍳⍉n I@1⊢z←rf I@0⊢⍵]\}⍣≡⍉\{rf[⍵],⍪⍵\}⍸(t=V)∧vb=¯1
         ∨⌿msk←(t=V)∧vb=¯1:\{
                 6'ALL VARIABLES MUST REFERENCE A BINDING'SIGNAL∊pos[⍵]\{⍺+⍳⍵-⍺\}¨end[⍵]
         \}⍸msk

⍝ ⍺/⍵ → V ; M → F0 ; ⍺⍺/⍵⍵ → P2
         t←V@(i←⍸(t=A)∧n∊,¨'⍺⍵')⊢F@\{t=M\}t ⋄ vb[i]←i ⋄ k[⍸(t=P)∧n∊'⍺⍺' '⍵⍵']←2

⍝ Infer types of bindings, groups, and variables
         z x←↓⍉p[i]\{⍺⍵\}⌸i←⍸(t[p]∊B Z)∧p≠⍳≢p
         x←\{⍵⌿⍨~∧⍀t[⍵]=¯1\}U⌽¨x
         0∨.=≢¨x:'BRACKET SYNTAX REQUIRES FUNCTION OR ARRAY TO ITS LEFT'⎕\nwlinkedidentc{SIGNAL}{NW2YR5B-aELRs-1} 2
         _←\{
                 k[msk⌿z]←k[x⌿⍨msk←(k[⊃¨x]≠0)∧1=≢¨x] ⋄ z x⌿⍨←⊂~msk
                 k[z⌿⍨msk←k[⊃¨x]=4]←3 ⋄ z x⌿⍨←⊂~msk
                 k[z⌿⍨msk←\{(2 3 5∊⍨k[⊃⍵])∨4=(⍵,≢k)[0⍳⍨∧⍀k[⍵]=1]⌷k,0\}∘⌽¨x]←2 ⋄ z x⌿⍨←⊂~msk
                 k[z⌿⍨msk←k[⊃∘⌽¨x]=1]←1 ⋄ z x⌿⍨←⊂~msk
                 k[i]←k[vb[i←⍸t=V]]
         ≢z\}⍣(=∨0=⊣)≢z
         'FAILED TO INFER ALL BINDING TYPES'assert 0=≢z:

⍝ Strand arrays into atoms
         i←|i⊣km←0<i←i[⍋|(i,⍨←-∪p[i]),p[i←⍸t[p]∊B Z]]
         msk←(t[i]∊C N)∨msk∧⊃1 ¯1∨.⌽⊂msk←km∧(t[i]∊A C N V Z)∧k[i]=1
         np←(≢p)+⍳≢ai←i⌿⍨am←2>⌿msk⍪0 ⋄ p←(np@ai⍳≢p)[p] ⋄ p,←ai ⋄ km←2<⌿0⍪msk
         t k n pos end(⊣,I)←⊂ai ⋄ k[ai]←1 6[∨⌿¨msk⊆t[i]≠N]
         t n pos(⊣@ai⍨)←A(⊂'')(pos[km⌿i]) ⋄ p[msk⌿i]←ai[(msk←msk∧~am)⌿¯1++⍀km]
         i←⍸(t[p]=A)∧(k[p]=6)∧t=N
         p,←i ⋄ t k n pos end(⊣,I)←⊂i ⋄ t k n(⊣@i⍨)←A 1(⊂'')

⍝ PARSE B←D...
⍝ PARSE B←...D

⍝ Rationalize F[X]
         _←p[i]\{
                 ⊃m←t[⍵]=¯1:'SYNTAX ERROR:NOTHING TO INDEX'⎕\nwlinkedidentc{SIGNAL}{NW2YR5B-aELRs-1} 2
                 k[⍵⌿⍨m∧¯1⌽(k[⍵]∊2 3 5)∨¯1⌽k[⍵]=4]←4
         0\}⌸i←⍸(t[p]∊B Z)∧(p≠⍳≢p)∧k[p]∊1 2
         i←⍸(t=¯1)∧k=4 ⋄ j←⍸(t[p]=¯1)∧k[p]=4
         (≢i)≠≢j:\{
                 2'AXIS REQUIRES SINGLE AXIS EXPRESSION'SIGNAL ∊pos[⍵]+⍳¨end[⍵]-pos[⍵]
         \}⊃⍪⌿\{⊂⍺⌿⍨1<≢⍵\}⌸p[j]
         ∨⌿msk←t[j]≠Z:\{
                 2'AXIS REQUIRES NON-EMPTY AXIS EXPRESSION'SIGNAL ∊pos[⍵]+⍳¨end[⍵]-pos[⍵]
         \}msk⌿p[j]
         p[j]←p[i] ⋄ t[i]←P ⋄ end[i]←1+pos[i]

⍝ Group function and value expressions
         i km←⍪⌿p[i]\{(⍺⍪⍵)(0,1∨⍵)\}⌸i←⍸(t[p]∊B Z)∧(p≠⍳≢p)∧k[p]∊1 2

⍝ Mask and verify dyadic operator right operands
         (dm←¯1⌽(k[i]=4)∧t[i]∊F P V Z)∨.∧(~km)∨k[i]∊0 3 4:\{
                 'MISSING RIGHT OPERAND'⎕\nwlinkedidentc{SIGNAL}{NW2YR5B-aELRs-1} 2
         \}⍬

⍝ Refine schizophrenic types
         k[i⌿⍨(k[i]=5)∧dm∨¯1⌽(~km)∨(~dm)∧k[i]∊1 6]←2 ⋄ k[i⌿⍨k[i]=5]←3

⍝ Rationalize ∘.
         jm←(t[i]=P)∧n[i]∊⊂,'∘.'
         jm∨.∧1⌽(~km)∨k[i]∊3 4:'MISSING OPERAND TO ∘.'⎕\nwlinkedidentc{SIGNAL}{NW2YR5B-aELRs-1} 2
         p←((ji←jm⌿i)@(jj←i⌿⍨¯1⌽jm)⍳≢p)[p] ⋄ t[ji,jj]←t[jj,ji] ⋄ k[ji,jj]←k[jj,ji]
         n[ji,jj]←n[jj,ji] ⋄ pos[ji,jj]←pos[ji,ji] ⋄ end[ji,jj]←end[jj,jj]

⍝ Mask and verify monadic and dyadic operator left operands
         ∨⌿msk←(dm∧¯2⌽~km)∨(¯1⌽~km)∧mm←(k[i]=3)∧t[i]∊F P V Z:\{
                 2'MISSING LEFT OPERAND'SIGNAL ∊pos[⍵]+⍳¨end[⍵]-pos[⍵]
         \}i⌿⍨msk
         msk←dm∨mm

⍝ Parse function expressions
         np←(≢p)+⍳xc←≢oi←msk⌿i ⋄ p←(np@oi⍳≢p)[p] ⋄ p,←oi ⋄ t k n pos end(⊣,I)←⊂oi
         p[g⌿i]←oi[(g←(~msk)∧(1⌽msk)∨2⌽dm)⌿xc-⌽+⍀⌽msk]
         p[g⌿oi]←(g←msk⌿(1⌽mm)∨2⌽dm)⌿1⌽oi ⋄ t[oi]←O ⋄ n[oi]←⊂''
         pos[oi]←pos[g⌿i][msk⌿¯1++⍀g←(~msk)∧(1⌽mm)∨2⌽dm]
         ol←1+(k[i⌿⍨(2⌽mm)∨3⌽dm]=4)∨k[i⌿⍨(1⌽mm)∨2⌽dm]∊2 3
         or←(msk⌿dm)⍀1+k[dm⌿i]=2
         k[oi]←3 3⊥↑or ol

⍝ Wrap all assignment values as Z nodes
         i km←⍪⌿p[i]\{(⍺⍪⍵)(0,1∨⍵)\}⌸i←⍸(t[p]∊B Z)∧(p≠⍳≢p)∧k[p]∊1
         j←i⌿⍨msk←(t[i]=P)∧n[i]∊⊂,'←' ⋄ nz←(≢p)+⍳zc←+⌿msk
         p,←nz ⋄ t k n,←zc⍴¨Z 1(⊂'') ⋄ pos,←1+pos[j] ⋄ end,←end[p[j]]
         zm←¯1⌽msk ⋄ p[km⌿i]←(zpm⌿(i×~km)+zm⍀nz)[km⌿¯1++⍀zpm←zm∨~km]

⍝ This is the definition of a function value at this point
         isfn←\{(t[⍵]∊O F)∨(t[⍵]∊B P V Z)∧k[⍵]=2\}

⍝ Parse modified assignment to E4(V, F, Z)
         j←i⌿⍨m←msk∧(¯1⌽isfn i)∧¯2⌽(t[i]=V)∧k[i]=1 ⋄ p[zi←nz⌿⍨msk⌿m]←j
         p[i⌿⍨(1⌽m)∨2⌽m]←2⌿j ⋄ t k(⊣@j⍨)←E 4 ⋄ pos end n\{⍺[⍵]@j⊢⍺\}←vi zi,⊂vi←i⌿⍨2⌽m

⍝ Parse bracket modified assignment to E4(E6, O2(F, P3(←)), Z)
         j←i⌿⍨m←msk∧(¯1⌽isfn i)∧(¯2⌽t[i]=¯1)∧¯3⌽(t[i]=V)∧k[i]=1
         p[zi←nz⌿⍨msk⌿m]←ei←i⌿⍨3⌽m ⋄ t k end(⊣@ei⍨)←E 4(end[zi])
         p t k n(⊣@(i⌿⍨2⌽m)⍨)←ei E 6(⊂'')
         p,←j ⋄ t,←P⍴⍨≢j ⋄ k,←3⍴⍨≢j ⋄ n,←(≢j)⍴⊂,'←' ⋄ pos,←pos[j] ⋄ end,←end[j]
         p t k n pos(⊣@j⍨)←ei O 2(⊂'')(pos[fi←i⌿⍨1⌽m]) ⋄ p[fi]←j

⍝ Parse bracket assignment to E4(E6, P2(←), Z)
         j←i⌿⍨m←msk∧(¯1⌽t[i]=¯1)∧¯2⌽(t[i]=V)∧k[i]=1 ⋄ p[zi←nz⌿⍨msk⌿m]←ei←i⌿⍨2⌽m
         t k end(⊣@ei⍨)←E 4(end[zi]) ⋄ p t k n(⊣@(i⌿⍨1⌽m)⍨)←ei E 6(⊂'')
         p t k(⊣@j⍨)←ei P 2

⍝ Parse modified strand assignment
⍝ Parse strand assignment

⍝ SELECTIVE MODIFIED ASSIGNMENT
⍝ SELECTIVE ASSIGNMENT

⍝ Enclose V[X;...] for expression parsing
         i←i[⍋p[i←⍸(t[p]∊B Z)∧(k[p]=1)∧p≠⍳≢p]] ⋄ j←i⌿⍨jm←t[i]=¯1
         t[j]←A ⋄ k[j]←¯1 ⋄ p[i⌿⍨1⌽jm]←j

⍝ TRAINS

⍝ Parse expression sequences
         i km←⍪⌿p[i]\{(⍺⍪⍵)(0,(2≤≢⍵)∧1∨⍵)\}⌸i←⍸(t[p]∊B Z)∧(k[p]=1)∧p≠⍳≢p
         msk←m2∨fm∧~¯1⌽m2←km∧(1⌽km)∧~fm←(t[i]=O)∨(t[i]≠A)∧k[i]=2
         t,←E⍴⍨xc←+⌿msk ⋄ k,←msk⌿msk+m2 ⋄ n,←xc⍴⊂''
         pos,←pos[msk⌿i] ⋄ end,←end[p[msk⌿i]]
         p,←msk⌿¯1⌽(i×~km)+km×x←¯1+(≢p)++⍀msk ⋄ p[km⌿i]←km⌿x

⍝ Rationalize V[X;...]
         i←i[⍋p[i←⍸(t[p]=A)∧k[p]=¯1]] ⋄ msk←~2≠⌿¯1,ip←p[i] ⋄ ip←∪ip ⋄ nc←2×≢ip
         t[ip]←E ⋄ k[ip]←2 ⋄ n[ip]←⊂'' ⋄ p[msk⌿i]←msk⌿(≢p)+1+2ׯ1++⍀~msk
         p,←2⌿ip ⋄ t,←nc⍴P E ⋄ k,←nc⍴2 6 ⋄ n,←nc⍴,¨'[' ''
         pos,←2⌿pos[ip] ⋄ end,←∊(1+pos[ip]),⍪end[ip] ⋄ pos[ip]←pos[i⌿⍨~msk]

⍝ Sanity check
         ERR←'INVARIANT ERROR: Z node with multiple children'
         ERR \nwlinkedidentc{assert}{NW2YR5B-AF6fz-1}(+⌿(t[p]=Z)∧p≠⍳≢p)=+⌿t=Z:

⍝ Count parentheses in source information
         ip←p[i←⍸(t[p]=Z)∧n[p]∊⊂,'('] ⋄ pos[i]←pos[ip] ⋄ end[i]←end[ip]

⍝ VERIFY Z/B NODE TYPES MATCH ACTUAL TYPE

⍝ Eliminate Z nodes from the tree
         zi←p I@\{t[p[⍵]]=Z\}⍣≡ki←⍸msk←(t[p]=Z)∧t≠Z
         p←(zi@ki⍳≢p)[p] ⋄ t k n pos end(⊣@zi⍨)←t k n pos end I¨⊂ki
         t k n pos end⌿⍨←⊂msk←~msk∨t=Z ⋄ p←(⍸~msk)(⊢-1+⍸)msk⌿p
\nwused{\\{NW2YR5B-38DvvD-1}}\nwidentuses{\\{{\nwixident{assert}}{assert}}\\{{\nwixident{SIGNAL}}{SIGNAL}}\\{{\nwixident{TEST}}{TEST}}}\nwindexuse{\nwixident{assert}}{assert}{NW2YR5B-4CIf4o-1}\nwindexuse{\nwixident{SIGNAL}}{SIGNAL}{NW2YR5B-4CIf4o-1}\nwindexuse{\nwixident{TEST}}{TEST}{NW2YR5B-4CIf4o-1}\nwendcode{}\nwbegindocs{48}\nwdocspar

\subsection{Compiler Transformations}

\nwenddocs{}\nwbegincode{49}\sublabel{NW2YR5B-1jC2QX-1}\nwmargintag{{\nwtagstyle{}\subpageref{NW2YR5B-1jC2QX-1}}}\moddef{Compiler~{\nwtagstyle{}\subpageref{NW2YR5B-1jC2QX-1}}}\endmoddef\nwstartdeflinemarkup\nwusesondefline{\\{NW2YR5B-1p0Y9w-1}}\nwenddeflinemarkup
 TT←\{((d t k n ss se)exp sym \nwlinkedidentc{src}{NW2YR5B-3C6SQT-1})←⍵ ⋄ I←\{(⊂⍵)⌷⍺\}
         A B C E F G K L M N O P S V Z←1+⍳15

⍝ Compute parent vector and reference scope
         r←I@\{t[⍵]≠F\}⍣≡⍨p⊣2\{p[⍵]←⍺[⍺⍸⍵]\}⌿⊢∘⊂⌸d⊣p←⍳≢d

⍝ Lift Functions to top-level
         p,←n[i]←(≢p)+⍳≢i←⍸(t=F)∧p≠⍳≢p ⋄ t k n r(⊣,I)←⊂i ⋄ p r I⍨←⊂n[i]@i⊢⍳≢p
         t[i]←C

⍝ Wrap expressions as binding or return statements
         i←(⍸(~t∊F G)∧t[p]=F),\{⍵⌿⍨2|⍳≢⍵\}⍸t[p]=G ⋄ p t k n r⌿⍨←⊂m←2@i⊢1⍴⍨≢p
         p r i I⍨←⊂j←(+⍀m)-1 ⋄ n←j I@(0≤⊢)n ⋄ p[i]←j←i-1
         k[j]←-(k[r[j]]=0)∨0@(\{⊃⌽⍵\}⌸p[j])⊢(t[j]=B)∨(t[j]=E)∧k[j]=4 ⋄ t[j]←E

⍝ Lift guard tests
         p[i]←p[x←¯1+i←\{⍵⌿⍨~2|⍳≢⍵\}⍸t[p]=G] ⋄ t[i,x]←t[x,i] ⋄ k[i,x]←k[x,i]
         n[x]←n[i] ⋄ p←((x,i)@(i,x)⊢⍳≢p)[p]

⍝ Count strand and indexing children
         n[⍸(t∊A E)∧k=6]←0 ⋄ n[p⌿⍨(t[p]∊A E)∧k[p]=6]+←1

⍝ Lift and flatten expressions
         p[i]←p[x←p I@\{~t[p[⍵]]∊F G\}⍣≡i←⍸t∊G A B C E O P V] ⋄ j←(⌽i)[⍋⌽x]
         p t k n r\{⍺[⍵]@i⊢⍺\}←⊂j ⋄ p←(i@j⊢⍳≢p)[p]

⍝ Compute slots for each frame
         s←¯1,⍨∊⍳¨n[∪x]←⊢∘≢⌸x←0⌷⍉e←∪I∘⍋⍨rn←r[b],⍪n[b←⍸t=B]

⍝ Compute frame depths
         d←(≢p)↑d ⋄ d[i←⍸t=F]←0 ⋄ _←\{z⊣d[i]+←⍵≠z←r[⍵]\}⍣≡i ⋄ f←d[0⌷⍉e],¯1

⍝ Record exported top-level bindings
         xi←⍸(t=B)∧k[r]=0

         p t k n f s r d xi sym\}
\nwused{\\{NW2YR5B-1p0Y9w-1}}\nwidentuses{\\{{\nwixident{src}}{src}}}\nwindexuse{\nwixident{src}}{src}{NW2YR5B-1jC2QX-1}\nwendcode{}\nwbegindocs{50}\nwdocspar

\subsection{Code Generator}

\nwenddocs{}\nwbegincode{51}\sublabel{NW2YR5B-HCURD-1}\nwmargintag{{\nwtagstyle{}\subpageref{NW2YR5B-HCURD-1}}}\moddef{Code Generator~{\nwtagstyle{}\subpageref{NW2YR5B-HCURD-1}}}\endmoddef\nwstartdeflinemarkup\nwusesondefline{\\{NW2YR5B-1p0Y9w-1}}\nwenddeflinemarkup
GC←\{
        p t k n fr sl rf fd xi sym←⍵ ⋄ A B C E F G K L M N O P S V Z←1+⍳15
        I←\{(⊂⍵)⌷⍺\} ⋄ com←\{⊃\{⍺,',',⍵\}/⍵\}
        ks←\{⍵⊂[0]⍨(⊃⍵)=⍵[;0]\} ⋄ nam←\{'∆'⎕R'__'∘⍕¨sym[|⍵]\}

        syms ←,¨'+'           '-'            '×'            '÷'            '*'             '⍟'           '|'              '○'           '⌊'          '⌈'           '!'
        nams ←        'add'    'sub'  'mul' 'div' 'exp' 'log' 'res'    'cir'   'min'  'max' 'fac'
        syms,←,¨'<'           '≤'          '='             '≥'           '>'             '≠'           '~'              '∧'           '∨'          '⍲'           '⍱'
        nams,←        'lth'    'lte'  'eql' 'gte' 'gth' 'neq' 'not'    'and'   'lor'  'nan' 'nor'
        syms,←,¨'⌷'                 '['            '⍳'           '⍴'           ','             '⍪'           '⌽'            '⍉'           '⊖'          '∊'           '⊃'
        nams,←        'sqd'    'brk'  'iot' 'rho' 'cat' 'ctf' 'rot'    'trn'   'rtf'  'mem' 'dis'
        syms,←,¨'≡'                 '≢'          '⊢'           '⊣'           '⊤'           '⊥'           '/'              '⌿'           '\\'            '⍀'           '?'
        nams,←        'eqv'    'nqv'  'rgt' 'lft' 'enc' 'dec' 'red'    'rdf'   'scn'  'scf' 'rol'
        syms,←,¨'↑'                 '↓'          '¨'            '⍨'           '.'             '⍤'           '⍣'            '∘'           '∪'          '∩'           '←'
        nams,←        'tke'    'drp'  'map' 'com' 'dot' 'rnk' 'pow'    'jot'   'unq'  'int' 'get'
        syms,←,¨'⍋'                 '⍒'          '∘.'  '⍷'           '⊂'           '⌹'           '⎕FFT' '⎕IFFT' '%s'         '⊆'           '⎕CONV'
        nams,←        'gdu'    'gdd'  'oup' 'fnd' 'par' 'mdv' 'fft'    'ift'   'scl'  'nst' 'conv'
        syms,←,¨'∇'                 ';'            '⍺'           '⍵'           '⍺⍺'        '⍵⍵'        '%u'
        nams,←        'this' 'span' 'l'               'r'             'aa'    'ww'    ''

        gck← (A 1)(A 6)
        gcv← 'Aa' 'As'
        gck,←(B 1)(B 2)(B 3)(B 4)
        gcv,←'Bv' 'Bf' 'Bo' 'Bo'
        gck,←(C 1)(C 2)
        gcv,←'Ca' 'Cf'
        gck,←(E ¯2)(E ¯1)(E 0)(E 1)(E 2)(E 4)(E 6)
        gcv,←'Ec'      'Ek'    'Er' 'Em' 'Ed' 'Eb' 'Ei'
        gck,←(F 0)(F 2)(F 3)(F 4)
        gcv,←'Fz' 'Fn' 'Fm' 'Fd'
        gck,←(G 0)(N 1)
        gcv,←'Gd' 'Na'
        gck,←(O 1)(O 2)(O 4) (O 5) (O 7) (O 8)
        gcv,←'Ov' 'Of' 'Ovv' 'Ofv' 'Ovf' 'Off'
        gck,←(P 0)(P 1)(P 2)(P 3)(P 4)
        gcv,←'Pv' 'Pv' 'Pf' 'Po' 'Po'
        gck,←(V 0)(V 1)(V 2)(V 3)(V 4)
        gcv,←'Va' 'Va' 'Vf' 'Vo' 'Vo'
        gcv,←⊂'\{''/* Unhandled '',(⍕⍺),'' */'',NL\}'
        NL←⎕UCS 13 10

        pref ←⊂'#include "\nwlinkedidentc{codfns}{NW2YR5B-1p0Y9w-1}.h"'
        pref,←⊂''
        pref,←⊂'EXPORT int'
        pref,←⊂'DyalogGetInterpreterFunctions(void *p)'
        pref,←⊂'\{'
        pref,←⊂'    return set_dwafns(p);'
        pref,←⊂'\}'
        pref,←⊂''

        Bf←\{id←sym⊃⍨|4⊃⍺
                z ←⊂id,' = retain_cell(stkhd[-1]);'
        z\}

        Cf←\{id←⍕4⊃⍺
                z ←⊂'mk_closure((struct closure **)stkhd++, fn',id,', 0);'
        z\}

        Ek←\{
                z ←⊂'release_cell(*--stkhd);'
                z,←⊂''
        z\}

        Em←\{
                z ←⊂'c = *--stkhd;'
                z,←⊂'w = *--stkhd;'
                z,←⊂'(c->fn)((struct array **)stkhd++, NULL, w, c->fv);'
                z,←⊂'release_cell(c);'
                z,←⊂'release_cell(w);'
        z\}

        Er←\{
                z ←⊂'*z = *--stkhd;'
                z,←⊂'goto cleanup;'
                z,←⊂''
        z\}

        Fn←\{id←⍕5⊃⍺ ⋄ x←⍉⊃⍪⌿⍵ ⋄ t←2⌷x ⋄ k←3⌷x
                hsw←(t=O)∨(t=E)∧k∊1 2 ⋄ hsa←((t=E)∧k=2)∨(t=O)∧k∊4 5 7 8
                z ←⊂'int'
                z,←⊂'fn',id,'(struct array **z, struct array *l, struct array *r, void *fv[])'
                z,←⊂'\{'
                z,←⊂'       void    *stk[128];'
                z,←⊂'       void    **stkhd;'
                z,←hsw⌿⊂' void    *w;'
                z,←hsa⌿⊂' void    *a;'
                z,←hsw⌿⊂' struct  closure *c;'
                z,←⊂''
                z,←⊂'       stkhd = &stk[0];'
                z,←⊂''
                z,← ' ',¨⊃,⌿dis¨⍵
                z,←⊂'       *z = NULL;'
                z,←⊂''
                z,←⊂'cleanup:'
                z,←⊂'       return 0;'
                z,←⊂'\}'
                z,←⊂''
        z\}

        Fz←\{id←⍕5⊃⍺ ⋄ awc←∨⌿(3⌷x)\{(⍵∊A O)∨(⍵=E)∧⍺>0\}2⌷x←⍉⊃⍪⌿⍵
                z ←⊂'int init',id,' = 0;'
                z,←⊂''
                z,←⊂'EXPORT int'
                z,←⊂'init(void)'
                z,←⊂'\{'
                z,←⊂' return fn',id,'(NULL, NULL, NULL, NULL);'
                z,←⊂'\}'
                z,←⊂''
                z,←⊂'int'
                z,←⊂'fn',id,'(struct array **z, struct array *l, struct array *r, void *fv[])'
                z,←⊂'\{'
                z,←⊂'       void    *stk[128];'
                z,←⊂'       void    **stkhd;'
                z,← awc⌿⊂'        void    *a, *w;'
                z,← awc⌿⊂'        struct  closure *c;'
                z,←⊂''
                z,←⊂'       if (init',id,')'
                z,←⊂'               return 0;'
                z,←⊂''
                z,←⊂'       stkhd = &stk[0];'
                z,←⊂'       init',id,' = 1;'
                z,←⊂'       cdf_init();'
                z,←⊂''
                z,← ' ',¨⊃,⌿dis¨⍵
                z,←⊂'       return 0;'
                z,←⊂'\}'
                z,←⊂''
        z\}

        Pf←\{id←(syms⍳sym[|4⊃⍺])⊃nams
                z ←⊂'*stkhd++ = retain_cell(',id,');'
        z\}

        Va←\{id←(|4⊃⍺)⊃'' 'r' 'l' 'aa' 'ww',5↓sym
                z ←⊂'*stkhd++ = retain_cell(',id,');'
        z\}

        Zp←\{n←'fn',⍕⍵
                k[⍵]∊0 2:\{
                        z ←⊂'int'
                        z,←⊂n,'(struct array **z, struct array *l, struct array *r, void *fv[]);'
                        z,←⊂''
                z\}⍵
                'UNKNOWN FUNCTION TYPE'⎕\nwlinkedidentc{SIGNAL}{NW2YR5B-aELRs-1} 16
        \}

        Zx←\{n←sym⊃⍨|n[⍵] ⋄ rid←⍕rf[⍵]
                k[⍵]=0:⊂''
                k[⍵]=1:\{
                        z ←⊂'struct array *',n,';'
                z\}⍵
                k[⍵]=2:\{
                        z ←⊂'struct closure *',n,';'
                        z,←⊂''
                        z,←⊂'EXPORT int'
                        z,←⊂n,'_dwa(struct localp *zp, struct localp *lp, struct localp *rp)'
                        z,←⊂'\{'
                        z,←⊂'       struct array *z, *l, *r;'
                        z,←⊂'       int err;'
                        z,←⊂''
                        z,←⊂'       l = NULL;'
                        z,←⊂'       r = NULL;'
                        z,←⊂''
                        z,←⊂'       fn',rid,'(NULL, NULL, NULL, NULL);'
                        z,←⊂''
                        z,←⊂'       err = 0;'
                        z,←⊂''
                        z,←⊂'       if (lp)'
                        z,←⊂'               err = dwa2array(&l, lp->pocket);'
                        z,←⊂''
                        z,←⊂'       if (err)'
                        z,←⊂'               dwa_error(err);;'
                        z,←⊂''
                        z,←⊂'       if (rp)'
                        z,←⊂'               dwa2array(&r, rp->pocket);'
                        z,←⊂''
                        z,←⊂'       if (err) \{'
                        z,←⊂'               release_array(l);'
                        z,←⊂'               dwa_error(err);'
                        z,←⊂'       \}'
                        z,←⊂''
                        z,←⊂'       err = (',n,'->fn)(&z, l, r, ',n,'->fv);'
                        z,←⊂''
                        z,←⊂'       release_array(l);'
                        z,←⊂'       release_array(r);'
                        z,←⊂''
                        z,←⊂'       if (err)'
                        z,←⊂'               dwa_error(err);'
                        z,←⊂''
                        z,←⊂'       err = array2dwa(NULL, z, zp);'
                        z,←⊂'       release_array(z);'
                        z,←⊂''
                        z,←⊂'       if (err)'
                        z,←⊂'               dwa_error(err);'
                        z,←⊂''
                        z,←⊂'       return 0;'
                        z,←⊂'\}'
                        z,←⊂''
                z\}⍵
                ⍎'''UNKNOWN EXPORT TYPE''⎕\nwlinkedidentc{SIGNAL}{NW2YR5B-aELRs-1} 16'
        \}

        d i←P2D p ⋄ ast←(⍉↑d p t k n(⍳≢p)fr sl fd)[i;]
        NOTFOUND←\{('[GC] UNSUPPORTED NODE TYPE ',N∆[⊃⍵],⍕⊃⌽⍵)⎕\nwlinkedidentc{SIGNAL}{NW2YR5B-aELRs-1} 16\}
        dis←\{0=2⊃h←,1↑⍵:'' ⋄ (≢gck)=i←gck⍳⊂h[2 3]:NOTFOUND h[2 3] ⋄ h(⍎i⊃gcv)ks 1↓⍵\}
        z←∊,∘NL¨pref,⊃,⌿(,⌿Zp¨⍸t=F),(,⌿Zx¨xi),(⊂⊂''),dis¨ks ast
        z\}
\nwused{\\{NW2YR5B-1p0Y9w-1}}\nwidentuses{\\{{\nwixident{codfns}}{codfns}}\\{{\nwixident{SIGNAL}}{SIGNAL}}}\nwindexuse{\nwixident{codfns}}{codfns}{NW2YR5B-HCURD-1}\nwindexuse{\nwixident{SIGNAL}}{SIGNAL}{NW2YR5B-HCURD-1}\nwendcode{}\nwbegindocs{52}\nwdocspar

\subsection{Backend C Compiler Interface}

\nwenddocs{}\nwbegincode{53}\sublabel{NW2YR5B-2LnsgP-1}\nwmargintag{{\nwtagstyle{}\subpageref{NW2YR5B-2LnsgP-1}}}\moddef{Interface to the backend C compiler~{\nwtagstyle{}\subpageref{NW2YR5B-2LnsgP-1}}}\endmoddef\nwstartdeflinemarkup\nwusesondefline{\\{NW2YR5B-1p0Y9w-1}}\nwenddeflinemarkup
CC←\{
        \nwlinkedidentc{vsbat}{NW2YR5B-2pNilx-1}←\nwlinkedidentc{VS∆PATH}{NW2YR5B-2z4lmm-4},'\\VC\\Auxiliary\\Build\\vcvarsall.bat'
        \nwlinkedidentc{tie}{NW2YR5B-XAz19-1}←\{0::⎕\nwlinkedidentc{SIGNAL}{NW2YR5B-aELRs-1} ⎕EN ⋄ 22::⍵ ⎕NCREATE 0 ⋄ 0 ⎕NRESIZE ⍵ ⎕NTIE 0\}
        \nwlinkedidentc{put}{NW2YR5B-XAz19-1}←\{s←(¯128+256|128+'UTF-8'⎕UCS ⍺)⎕NAPPEND(t←\nwlinkedidentc{tie}{NW2YR5B-XAz19-1} ⍵)83 ⋄ 1:r←s⊣⎕NUNTIE t\}
        \LA{}The \code{}opsys\edoc{} utility~{\nwtagstyle{}\subpageref{NW2YR5B-1ejEd9-1}}\RA{}
        soext←\{opsys'.dll' '.so' '.dylib'\}
        ccf←\{' -o ''',⍵,'.',⍺,''' ''',⍵,'.c'' -laf',\nwlinkedidentc{AF∆LIB}{NW2YR5B-2z4lmm-3},' > ',⍵,'.log 2>&1'\}
        cci←\{'-I''',\nwlinkedidentc{AF∆PREFIX}{NW2YR5B-2z4lmm-3},'/include'' -L''',\nwlinkedidentc{AF∆PREFIX}{NW2YR5B-2z4lmm-3},opsys''' ' '/lib64'' ' '/lib'' '\}
        cco←'-std=c99 -Ofast -g -Wall -fPIC -shared -Wno-parentheses '
        cco,←'-Wno-misleading-indentation '
        ucc←\{⍵⍵(⎕SH ⍺⍺,' ',cco,cci,ccf)⍵\}
        gcc←'gcc'ucc'so'
        clang←'clang'ucc'dylib'
        vsco←\{z←'/W3 /wd4102 /wd4275 /O2 /Zc:inline /Zi /FS /Fd"',⍵,'.pdb" '
                z,←'/WX /MD /EHsc /nologo '
                z,'/I"%AF_PATH%\\include" /D "NOMINMAX" /D "AF_DEBUG" '\}
        vslo←\{z←'/link /DLL /OPT:REF /INCREMENTAL:NO /SUBSYSTEM:WINDOWS '
                z,←'/LIBPATH:"%AF_PATH%\\lib" /OPT:ICF /ERRORREPORT:PROMPT /TLBID:1 '
                z,'/DYNAMICBASE "af', \nwlinkedidentc{AF∆LIB}{NW2YR5B-2z4lmm-3}, '.lib" "\nwlinkedidentc{codfns}{NW2YR5B-1p0Y9w-1}.lib" '\}
        vsc0←\{~⎕NEXISTS \nwlinkedidentc{vsbat}{NW2YR5B-2pNilx-1}:'VISUAL C?'⎕\nwlinkedidentc{SIGNAL}{NW2YR5B-aELRs-1} 99 ⋄ '""',\nwlinkedidentc{vsbat}{NW2YR5B-2pNilx-1},'" amd64'\}
        vsc1←\{' && cd "',(⊃⎕CMD'echo %CD%'),'" && cl ',(vsco ⍵),' "',⍵,'.c" '\}
        vsc2←\{(vslo ⍵),'/OUT:"',⍵,'.dll" > "',⍵,'.log""'\}
        \nwlinkedidentc{vsc}{NW2YR5B-2pNilx-1}←\{⎕CMD ('%comspec% /C ',vsc0,vsc1,vsc2)⍵\}
        _←(⍎opsys'vsc' 'gcc' 'clang')⍺⊣⍵ \nwlinkedidentc{put}{NW2YR5B-XAz19-1} ⍺,'.c'⊣1 ⎕NDELETE f←⍺,soext⍬
        ⎕←⍪⊃⎕NGET(⍺,'.log')1
        ⎕NEXISTS f:f ⋄ 'COMPILE ERROR' ⎕\nwlinkedidentc{SIGNAL}{NW2YR5B-aELRs-1} 22\}
\nwused{\\{NW2YR5B-1p0Y9w-1}}\nwidentuses{\\{{\nwixident{AF∆LIB}}{AF∆LIB}}\\{{\nwixident{AF∆PREFIX}}{AF∆PREFIX}}\\{{\nwixident{codfns}}{codfns}}\\{{\nwixident{put}}{put}}\\{{\nwixident{SIGNAL}}{SIGNAL}}\\{{\nwixident{tie}}{tie}}\\{{\nwixident{vsbat}}{vsbat}}\\{{\nwixident{vsc}}{vsc}}\\{{\nwixident{VS∆PATH}}{VS∆PATH}}}\nwindexuse{\nwixident{AF∆LIB}}{AF∆LIB}{NW2YR5B-2LnsgP-1}\nwindexuse{\nwixident{AF∆PREFIX}}{AF∆PREFIX}{NW2YR5B-2LnsgP-1}\nwindexuse{\nwixident{codfns}}{codfns}{NW2YR5B-2LnsgP-1}\nwindexuse{\nwixident{put}}{put}{NW2YR5B-2LnsgP-1}\nwindexuse{\nwixident{SIGNAL}}{SIGNAL}{NW2YR5B-2LnsgP-1}\nwindexuse{\nwixident{tie}}{tie}{NW2YR5B-2LnsgP-1}\nwindexuse{\nwixident{vsbat}}{vsbat}{NW2YR5B-2LnsgP-1}\nwindexuse{\nwixident{vsc}}{vsc}{NW2YR5B-2LnsgP-1}\nwindexuse{\nwixident{VS∆PATH}}{VS∆PATH}{NW2YR5B-2LnsgP-1}\nwendcode{}\nwbegindocs{54}\nwdocspar

\subsection{Linking with Dyalog}

\nwenddocs{}\nwbegincode{55}\sublabel{NW2YR5B-zH457-1}\nwmargintag{{\nwtagstyle{}\subpageref{NW2YR5B-zH457-1}}}\moddef{Linking with Dyalog~{\nwtagstyle{}\subpageref{NW2YR5B-zH457-1}}}\endmoddef\nwstartdeflinemarkup\nwusesondefline{\\{NW2YR5B-1p0Y9w-1}}\nwenddeflinemarkup
 NS←\{
         MKA←\{mka⊂⍵\} ⋄ EXA←\{exa ⍬ ⍵\}
         Display←\{⍺←'Co-dfns' ⋄ W←w_new⊂⍺ ⋄ 777::w_del W
                 w_del W⊣W ⍺⍺\{w_close ⍺:⍎'⎕\nwlinkedidentc{SIGNAL}{NW2YR5B-aELRs-1} 777' ⋄ ⍺ ⍺⍺ ⍵\}⍣⍵⍵⊢⍵\}
         LoadImage←\{⍺←1 ⋄ ~⎕NEXISTS ⍵:⎕\nwlinkedidentc{SIGNAL}{NW2YR5B-aELRs-1} 22 ⋄ loadimg ⍬ ⍵ ⍺\}
         SaveImage←\{⍺←'image.png' ⋄ saveimg ⍵ ⍺\}
         Image←\{~2 3∨.=≢⍴⍵:⎕\nwlinkedidentc{SIGNAL}{NW2YR5B-aELRs-1} 4 ⋄ (3≠⊃⍴⍵)∧3=≢⍴⍵:⎕\nwlinkedidentc{SIGNAL}{NW2YR5B-aELRs-1} 5 ⋄ ⍵⊣w_img ⍵ ⍺\}
         Plot←\{2≠≢⍴⍵:⎕\nwlinkedidentc{SIGNAL}{NW2YR5B-aELRs-1} 4 ⋄ ~2 3∨.=1⊃⍴⍵:⎕\nwlinkedidentc{SIGNAL}{NW2YR5B-aELRs-1} 5 ⋄ ⍵⊣w_plot (⍉⍵) ⍺\}
         Histogram←\{⍵⊣w_hist ⍵,⍺\}
         Rtm∆Init←\{
                 _←'w_new'    ⎕NA'P ',⍵,'|w_new           <C[]'
                 _←'w_close'⎕NA'I ',⍵,'|w_close P'
                 _←'w_del'    ⎕NA                   ⍵,'|w_del             P'
                 _←'w_img'    ⎕NA                   ⍵,'|w_img             <\nwlinkedidentc{PP}{NW2YR5B-39J7A7-1} P'
                 _←'w_plot' ⎕NA                     ⍵,'|w_plot    <\nwlinkedidentc{PP}{NW2YR5B-39J7A7-1} P'
                 _←'w_hist' ⎕NA                     ⍵,'|w_hist    <\nwlinkedidentc{PP}{NW2YR5B-39J7A7-1} F8   F8 P'
                 _←'loadimg'⎕NA                     ⍵,'|loadimg >\nwlinkedidentc{PP}{NW2YR5B-39J7A7-1} <C[] I'
                 _←'saveimg'⎕NA                     ⍵,'|saveimg <\nwlinkedidentc{PP}{NW2YR5B-39J7A7-1} <C[]'
                 _←'exa'              ⎕NA                   ⍵,'|exarray >\nwlinkedidentc{PP}{NW2YR5B-39J7A7-1} P'
                 _←'mka'              ⎕NA'P ',⍵,'|mkarray <PP'
                 _←'FREA'             ⎕NA                   ⍵,'|frea              P'
                 _←'Sync'             ⎕NA                   ⍵,'|cd_sync'
                 0 0 ⍴ ⍬\}
         mkna←\{⍺,'|',('∆'⎕R'__'⊢⍵),'_cdf P P P'\}
         mkf←\{fn←⍺,'|',('∆'⎕R'__'⊢⍵),'_dwa ' ⋄ mon dya←⍵∘,¨'_mon' '_dya'
                 z←('Z←\{A\}',⍵,' W')(':If 0=⎕NC''⍙.',mon,'''')
                 z,←(mon dya\{'''',⍺,'''⍙.⎕NA''',fn,⍵,' <PP'''\}¨'>\nwlinkedidentc{PP}{NW2YR5B-39J7A7-1} P' '>\nwlinkedidentc{PP}{NW2YR5B-39J7A7-1} <PP'),⊂':EndIf'
                 z,':If 0=⎕NC''A'''('Z←⍙.',mon,' 0 0 W')':Else'('Z←⍙.',dya,' 0 A W')':EndIf'\}
         ns←#.⎕NS⍬ ⋄ _←'∆⍙'ns.⎕NS¨⊂⍬ ⋄ ∆ ⍙←ns.(∆ ⍙) ⋄ ∆.names←(0⍴⊂''),(2=1⊃⍺)⌿0⊃⍺
         fns←'Rtm∆Init' 'MKA' 'EXA' 'Display' 'LoadImage' 'SaveImage' 'Image' 'Plot'
         fns,←'Histogram' 'soext' 'opsys' 'mkna'
         _←∆.⎕FX∘⎕CR¨fns ⋄ ∆.(decls←⍵∘mkna¨names) ⋄ _←ns.⎕FX¨(⊂''),⍵∘mkf¨∆.names
         _←∆.⎕FX'Z←Init'('Z←Rtm∆Init ''',⍵,'''')'→0⌿⍨0=≢names' 'names ##.⍙.⎕NA¨decls'
         ns\}
\nwused{\\{NW2YR5B-1p0Y9w-1}}\nwidentuses{\\{{\nwixident{PP}}{PP}}\\{{\nwixident{SIGNAL}}{SIGNAL}}}\nwindexuse{\nwixident{PP}}{PP}{NW2YR5B-zH457-1}\nwindexuse{\nwixident{SIGNAL}}{SIGNAL}{NW2YR5B-zH457-1}\nwendcode{}\nwbegindocs{56}\nwdocspar

\section{Co-dfns Runtime}

\nwenddocs{}\nwbegincode{57}\sublabel{NW2YR5B-941K7-1}\nwmargintag{{\nwtagstyle{}\subpageref{NW2YR5B-941K7-1}}}\moddef{Implementation of APL Primitives~{\nwtagstyle{}\subpageref{NW2YR5B-941K7-1}}}\endmoddef\nwstartdeflinemarkup\nwenddeflinemarkup
⍝ TBW
\nwnotused{Implementation of APL Primitives}\nwendcode{}\nwbegindocs{58}\nwdocspar

\nwenddocs{}\nwbegincode{59}\sublabel{NW2YR5B-3ZoAJL-1}\nwmargintag{{\nwtagstyle{}\subpageref{NW2YR5B-3ZoAJL-1}}}\moddef{C Runtime Support~{\nwtagstyle{}\subpageref{NW2YR5B-3ZoAJL-1}}}\endmoddef\nwstartdeflinemarkup\nwenddeflinemarkup
/* TBW */
\nwnotused{C Runtime Support}\nwendcode{}\nwbegindocs{60}\nwdocspar

\nwenddocs{}\nwbegincode{61}\sublabel{NW2YR5B-2mMS19-1}\nwmargintag{{\nwtagstyle{}\subpageref{NW2YR5B-2mMS19-1}}}\moddef{C Runtime Header~{\nwtagstyle{}\subpageref{NW2YR5B-2mMS19-1}}}\endmoddef\nwstartdeflinemarkup\nwenddeflinemarkup
/* TBW */
\nwnotused{C Runtime Header}\nwendcode{}\nwbegindocs{62}\nwdocspar

\section{Utilities}

\subsection{Must haves}

There are some APL functions that are so critical as to be worthy 
of primitive status.

\begin{itemize}
        \item Indexing
        \item Under
        \item Assert
\end{itemize}

\nwenddocs{}\nwbegincode{63}\sublabel{NW2YR5B-AF6fz-1}\nwmargintag{{\nwtagstyle{}\subpageref{NW2YR5B-AF6fz-1}}}\moddef{Must Have APL Utilities~{\nwtagstyle{}\subpageref{NW2YR5B-AF6fz-1}}}\endmoddef\nwstartdeflinemarkup\nwusesondefline{\\{NW2YR5B-1p0Y9w-1}}\nwenddeflinemarkup
I←\{(⊂⍵)⌷⍺\}
U←\{⍺←⊢ ⋄ ⍵⍵⍣¯1⊢⍺ ⍺⍺⍥⍵⍵ ⍵\}
\nwlinkedidentc{assert}{NW2YR5B-AF6fz-1}←\{⍺←'assertion failure' ⋄ 0∊⍵:⍎'⍺ ⎕\nwlinkedidentc{SIGNAL}{NW2YR5B-aELRs-1} 8' ⋄ shy←0\}
\nwindexdefn{\nwixident{assert}}{assert}{NW2YR5B-AF6fz-1}\eatline
\nwused{\\{NW2YR5B-1p0Y9w-1}}\nwidentdefs{\\{{\nwixident{assert}}{assert}}}\nwidentuses{\\{{\nwixident{SIGNAL}}{SIGNAL}}}\nwindexuse{\nwixident{SIGNAL}}{SIGNAL}{NW2YR5B-AF6fz-1}\nwendcode{}\nwbegindocs{64}\nwdocspar
\subsection{AST Pretty-printing}

\nwenddocs{}\nwbegincode{65}\sublabel{NW2YR5B-1VExi3-1}\nwmargintag{{\nwtagstyle{}\subpageref{NW2YR5B-1VExi3-1}}}\moddef{Pretty-printing AST trees~{\nwtagstyle{}\subpageref{NW2YR5B-1VExi3-1}}}\endmoddef\nwstartdeflinemarkup\nwusesondefline{\\{NW2YR5B-1p0Y9w-1}}\nwenddeflinemarkup
\nwlinkedidentc{dct}{NW2YR5B-1VExi3-1}←\{⍺[(2×2≠/n,0)+(1↑⍨≢m)+m+n←⌽∨\\⌽m←' '≠⍺⍺ ⍵]⍵⍵ ⍵\}
\nwlinkedidentc{dlk}{NW2YR5B-1VExi3-1}←\{((x⌷⍴⍵)↑[x←2|1+⍵⍵]⍺),[⍵⍵]⍺⍺@(⊂0 0)⍣('┌'=⊃⍵)⊢⍵\}
\nwlinkedidentc{dwh}{NW2YR5B-1VExi3-1}←\{⍵('┬'dlk 1)' │├┌└─'(0⌷⍉)\nwlinkedidentc{dct}{NW2YR5B-1VExi3-1},⊃⍪/((≢¨⍺),¨⊂⌈/≢∘⍉¨⍺)↑¨⍺\}
\nwlinkedidentc{dwv}{NW2YR5B-1VExi3-1}←\{⍵('├'dlk 0)' ─┬┌┐│'(0⌷⊢)\nwlinkedidentc{dct}{NW2YR5B-1VExi3-1}(⊣⍪1↓⊢)⊃\{⍺,' ',⍵\}/(1+⌈/≢¨⍺)\{⍺↑⍵⍪⍨'│'↑⍨≢⍉⍵\}¨⍺\}

\nwlinkedidentc{pp3}{NW2YR5B-1VExi3-1}←\{⍺←'○' ⋄ d←(⍳≢⍵)≠⍵ ⋄ _←\{z⊣d+←⍵≠z←⍺[⍵]\}⍣≡⍨⍵ ⋄ lbl←⍺⍴⍨≢⍵
        lyr←\{i←⍸⍺=d ⋄ k v←↓⍉⍵⍵[i],∘⊂⌸i ⋄ (⍵∘\{⍺[⍵]\}¨v)⍺⍺¨@k⊢⍵\}⍵
        (⍵=⍳≢⍵)⌿⊃⍺⍺ lyr⌿(1+⍳⌈/d),⊂⍉∘⍪∘⍕¨lbl\}

\nwlinkedidentc{lb3}{NW2YR5B-1VExi3-1}←\{⍺←⍳≢⊃⍵
        '(',¨')',¨⍨\{⍺,';',⍵\}⌿⍕¨(N∆\{⍺[⍵]\}@2⊢(2⊃⍵)\{⍺[|⍵]\}@\{0>⍵\}@4↑⊃⍵)[⍺;]\}
\nwindexdefn{\nwixident{dct}}{dct}{NW2YR5B-1VExi3-1}\nwindexdefn{\nwixident{dlk}}{dlk}{NW2YR5B-1VExi3-1}\nwindexdefn{\nwixident{dwh}}{dwh}{NW2YR5B-1VExi3-1}\nwindexdefn{\nwixident{dwv}}{dwv}{NW2YR5B-1VExi3-1}\nwindexdefn{\nwixident{pp3}}{pp3}{NW2YR5B-1VExi3-1}\nwindexdefn{\nwixident{lb3}}{lb3}{NW2YR5B-1VExi3-1}\eatline
\nwused{\\{NW2YR5B-1p0Y9w-1}}\nwidentdefs{\\{{\nwixident{dct}}{dct}}\\{{\nwixident{dlk}}{dlk}}\\{{\nwixident{dwh}}{dwh}}\\{{\nwixident{dwv}}{dwv}}\\{{\nwixident{lb3}}{lb3}}\\{{\nwixident{pp3}}{pp3}}}\nwendcode{}\nwbegindocs{66}\nwdocspar
\subsection{Debugging utilities}

The following utilities help to improve quality of life when working
with the Co-dfns source code.

The {\Tt{}\nwlinkedidentq{DISPLAY}{NW2YR5B-2qLMFZ-1}\nwendquote} function is taken from \url{https://dfns.dyalog.com}
and helps to make debugging easier by allowing us to thread
{\Tt{}\nwlinkedidentq{DISPLAY}{NW2YR5B-2qLMFZ-1}\nwendquote} calls into expressions. I prefer to do something like
this:

\begin{verbatim}
... {⍵⊣⎕←#.DISPLAY ⍵} ...
\end{verbatim}

\noindent
The function itself returns the character rendering of the code,
so the above little expression is one that I use to insert and do
debugging within an expression.

\nwenddocs{}\nwbegincode{67}\sublabel{NW2YR5B-2qLMFZ-1}\nwmargintag{{\nwtagstyle{}\subpageref{NW2YR5B-2qLMFZ-1}}}\moddef{\code{}DISPLAY\edoc{} Utility~{\nwtagstyle{}\subpageref{NW2YR5B-2qLMFZ-1}}}\endmoddef\nwstartdeflinemarkup\nwenddeflinemarkup
\nwlinkedidentc{DISPLAY}{NW2YR5B-2qLMFZ-1}←\{\nwlinkedidentc{⎕IO}{NW2YR5B-2z4lmm-1} \nwlinkedidentc{⎕ML}{NW2YR5B-2z4lmm-1}←0                                                                                      ⍝ Boxed display of array.

        ⍺←1 ⋄ chars←⍺⊃'..''''|-' '┌┐└┘│─'                       ⍝ ⍺: 0-clunky, 1-smooth.

        tl tr bl br vt hz←chars                                                               ⍝ Top left, top right, ...

        box←\{                                                                                                                                 ⍝ Box with type and axes.
                vrt hrz←(¯1+⍴⍵)⍴¨vt hz                                                        ⍝ Vert. and horiz. lines.
                top←(hz,'⊖→')[¯1↑⍺],hrz                                                      ⍝ Upper border with axis.
                bot←(⊃⍺),hrz                                                                                              ⍝ Lower border with type.
                rgt←tr,vt,vrt,br                                                                              ⍝ Right side with corners.
                lax←(vt,'⌽↓')[¯1↓1↓⍺],¨⊂vrt                                     ⍝ Left side(s) with axes,
                lft←⍉tl,(↑lax),bl                                                                         ⍝ ... and corners.
                lft,(top⍪⍵⍪bot),rgt                                                                       ⍝ Fully boxed array.
        \}

        deco←\{⍺←type open ⍵ ⋄ ⍺,axes ⍵\}                           ⍝ Type and axes vector.
        axes←\{(-2⌈⍴⍴⍵)↑1+×⍴⍵\}                                                                  ⍝ Array axis types.
        open←\{(1⌈⍴⍵)⍴⍵\}                                                                                             ⍝ Expose null axes.
        trim←\{(~1 1⍷∧⌿⍵=' ')/⍵\}                                                             ⍝ Remove extra blank cols.
        type←\{\{(1=⍴⍵)⊃'+'⍵\}∪,char¨⍵\}                                     ⍝ Simple array type.
        char←\{⍬≡⍴⍵:hz ⋄ (⊃⍵∊'¯',⎕D)⊃'#~'\}∘⍕          ⍝ Simple scalar type.
        line←\{(6≠10|⎕DR' '⍵)⊃' -'\}                                            ⍝ underline for atom.

        \{                                                                                                                                                       ⍝ Recursively box arrays:
                0=≡⍵:' '⍪(open ⎕FMT ⍵)⍪line ⍵                             ⍝ Simple scalar.
                1 ⍬≡(≡⍵)(⍴⍵):'∇' 0 0 box ⎕FMT ⍵                       ⍝ Object rep: ⎕OR.
                1=≡⍵:(deco ⍵)box open ⎕FMT open ⍵             ⍝ Simple array.
                ('∊'deco ⍵)box trim ⎕FMT ∇¨open ⍵            ⍝ Nested array.
        \}⍵
\}
\nwindexdefn{\nwixident{DISPLAY}}{DISPLAY}{NW2YR5B-2qLMFZ-1}\eatline
\nwnotused{[[DISPLAY]] Utility}\nwidentdefs{\\{{\nwixident{DISPLAY}}{DISPLAY}}}\nwidentuses{\\{{\nwixident{⎕IO}}{⎕IO}}\\{{\nwixident{⎕ML}}{⎕ML}}}\nwindexuse{\nwixident{⎕IO}}{⎕IO}{NW2YR5B-2qLMFZ-1}\nwindexuse{\nwixident{⎕ML}}{⎕ML}{NW2YR5B-2qLMFZ-1}\nwendcode{}\nwbegindocs{68}\nwdocspar
I also define a function {\Tt{}\nwlinkedidentq{PP}{NW2YR5B-39J7A7-1}\nwendquote} that encapsulates the above usage
pattern that I like to use, making the whole thing less verbose and
a little more convenient.

\nwenddocs{}\nwbegincode{69}\sublabel{NW2YR5B-39J7A7-1}\nwmargintag{{\nwtagstyle{}\subpageref{NW2YR5B-39J7A7-1}}}\moddef{\code{}PP\edoc{} Utility~{\nwtagstyle{}\subpageref{NW2YR5B-39J7A7-1}}}\endmoddef\nwstartdeflinemarkup\nwenddeflinemarkup
\nwlinkedidentc{PP}{NW2YR5B-39J7A7-1}←\{⍵⊣⎕←#.\nwlinkedidentc{DISPLAY}{NW2YR5B-2qLMFZ-1} ⍵\}
\nwindexdefn{\nwixident{PP}}{PP}{NW2YR5B-39J7A7-1}\eatline
\nwnotused{[[PP]] Utility}\nwidentdefs{\\{{\nwixident{PP}}{PP}}}\nwidentuses{\\{{\nwixident{DISPLAY}}{DISPLAY}}}\nwindexuse{\nwixident{DISPLAY}}{DISPLAY}{NW2YR5B-39J7A7-1}\nwendcode{}\nwbegindocs{70}\nwdocspar
Both of these function exist outside of the {\Tt{}\nwlinkedidentq{codfns}{NW2YR5B-1p0Y9w-1}\nwendquote} namespace 
and so they get their own files inside of the {\Tt{}\nwlinkedidentq{src}{NW2YR5B-3C6SQT-1}{\nwbackslash}\nwendquote} directory.

\nwenddocs{}\nwbegincode{71}\sublabel{NW2YR5B-ufqpm-3}\nwmargintag{{\nwtagstyle{}\subpageref{NW2YR5B-ufqpm-3}}}\moddef{Tangle Commands~{\nwtagstyle{}\subpageref{NW2YR5B-ufqpm-1}}}\plusendmoddef\nwstartdeflinemarkup\nwusesondefline{\\{NW2YR5B-42EjwV-1}}\nwprevnextdefs{NW2YR5B-ufqpm-2}{NW2YR5B-ufqpm-4}\nwenddeflinemarkup
echo "Tangling \nwlinkedidentc{src}{NW2YR5B-3C6SQT-1}/\nwlinkedidentc{DISPLAY}{NW2YR5B-2qLMFZ-1}\nwlinkedidentc{.aplf}{NW2YR5B-ufqpm-3}..."
notangle -R'[[\nwlinkedidentc{DISPLAY}{NW2YR5B-2qLMFZ-1}]] Utility' \nwlinkedidentc{codfns}{NW2YR5B-1p0Y9w-1}.nw > \nwlinkedidentc{src}{NW2YR5B-3C6SQT-1}/\nwlinkedidentc{DISPLAY}{NW2YR5B-2qLMFZ-1}\nwlinkedidentc{.aplf}{NW2YR5B-ufqpm-3}

echo "Tangling \nwlinkedidentc{src}{NW2YR5B-3C6SQT-1}/\nwlinkedidentc{PP}{NW2YR5B-39J7A7-1}\nwlinkedidentc{.aplf}{NW2YR5B-ufqpm-3}..."
notangle -R'[[\nwlinkedidentc{PP}{NW2YR5B-39J7A7-1}]] Utility' \nwlinkedidentc{codfns}{NW2YR5B-1p0Y9w-1}.nw > \nwlinkedidentc{src}{NW2YR5B-3C6SQT-1}/\nwlinkedidentc{PP}{NW2YR5B-39J7A7-1}\nwlinkedidentc{.aplf}{NW2YR5B-ufqpm-3}
\nwindexdefn{\nwixident{DISPLAY.aplf}}{DISPLAY.aplf}{NW2YR5B-ufqpm-3}\nwindexdefn{\nwixident{PP.aplf}}{PP.aplf}{NW2YR5B-ufqpm-3}\eatline
\nwused{\\{NW2YR5B-42EjwV-1}}\nwidentdefs{\\{{\nwixident{DISPLAY.aplf}}{DISPLAY.aplf}}\\{{\nwixident{PP.aplf}}{PP.aplf}}}\nwidentuses{\\{{\nwixident{codfns}}{codfns}}\\{{\nwixident{DISPLAY}}{DISPLAY}}\\{{\nwixident{PP}}{PP}}\\{{\nwixident{src}}{src}}}\nwindexuse{\nwixident{codfns}}{codfns}{NW2YR5B-ufqpm-3}\nwindexuse{\nwixident{DISPLAY}}{DISPLAY}{NW2YR5B-ufqpm-3}\nwindexuse{\nwixident{PP}}{PP}{NW2YR5B-ufqpm-3}\nwindexuse{\nwixident{src}}{src}{NW2YR5B-ufqpm-3}\nwendcode{}\nwbegindocs{72}\nwdocspar
\subsection{Reading and Writing Files}

It is helpful to be able to easily write files to disk, and the
following {\Tt{}\nwlinkedidentq{put}{NW2YR5B-XAz19-1}\nwendquote} and {\Tt{}\nwlinkedidentq{tie}{NW2YR5B-XAz19-1}\nwendquote} utilities help us to do so when we
want to.
These are pretty standard, but they could maybe be replaced by
{\Tt{}⎕NPUT\nwendquote} or something like that.

\nwenddocs{}\nwbegincode{73}\sublabel{NW2YR5B-XAz19-1}\nwmargintag{{\nwtagstyle{}\subpageref{NW2YR5B-XAz19-1}}}\moddef{Basic \code{}tie\edoc{} and \code{}put\edoc{} utilities~{\nwtagstyle{}\subpageref{NW2YR5B-XAz19-1}}}\endmoddef\nwstartdeflinemarkup\nwusesondefline{\\{NW2YR5B-66tOQ-1}}\nwenddeflinemarkup
\nwlinkedidentc{tie}{NW2YR5B-XAz19-1}←\{
        0::⎕\nwlinkedidentc{SIGNAL}{NW2YR5B-aELRs-1} ⎕EN
        22::⍵ ⎕NCREATE 0
        0 ⎕NRESIZE ⍵ ⎕NTIE 0
\}

\nwlinkedidentc{put}{NW2YR5B-XAz19-1}←\{
        s←(¯128+256|128+'UTF-8'⎕UCS ⍵)⎕NAPPEND(t←\nwlinkedidentc{tie}{NW2YR5B-XAz19-1} ⍺)83
        1:r←s⊣⎕NUNTIE t
\}
\nwindexdefn{\nwixident{tie}}{tie}{NW2YR5B-XAz19-1}\nwindexdefn{\nwixident{put}}{put}{NW2YR5B-XAz19-1}\eatline
\nwused{\\{NW2YR5B-66tOQ-1}}\nwidentdefs{\\{{\nwixident{put}}{put}}\\{{\nwixident{tie}}{tie}}}\nwidentuses{\\{{\nwixident{SIGNAL}}{SIGNAL}}}\nwindexuse{\nwixident{SIGNAL}}{SIGNAL}{NW2YR5B-XAz19-1}\nwendcode{}\nwbegindocs{74}\nwdocspar
\subsection{XML Rendering}

\nwenddocs{}\nwbegincode{75}\sublabel{NW2YR5B-1evvdc-1}\nwmargintag{{\nwtagstyle{}\subpageref{NW2YR5B-1evvdc-1}}}\moddef{XML Rendering~{\nwtagstyle{}\subpageref{NW2YR5B-1evvdc-1}}}\endmoddef\nwstartdeflinemarkup\nwusesondefline{\\{NW2YR5B-1p0Y9w-1}}\nwenddeflinemarkup
\nwlinkedidentc{Xml}{NW2YR5B-1evvdc-1}←\{⍺←0 ⋄ ast←⍺\{d i←P2D⊃⍵ ⋄ i∘\{⍵[⍺]\}¨(⊂d),1↓⍺↓⍵\}⍣(0≠⍺)⊢⍵ ⋄ d t k n←4↑ast
        cls←N∆[t],¨('-..'[1+×k]),¨⍕¨|k ⋄ fld←\{((≢⍵)↑3↓f∆),⍪⍵\}¨↓⍉↑3↓ast
        ⎕XML⍉↑d cls(⊂'')fld\}
\nwindexdefn{\nwixident{Xml}}{Xml}{NW2YR5B-1evvdc-1}\eatline
\nwused{\\{NW2YR5B-1p0Y9w-1}}\nwidentdefs{\\{{\nwixident{Xml}}{Xml}}}\nwendcode{}\nwbegindocs{76}\nwdocspar
\subsection{Detecting the Operating System}

It is quite helpful to be able to easily detect the operating system
that we are on.
This turns out to be helpful in more areas than just the compiler.

\nwenddocs{}\nwbegincode{77}\sublabel{NW2YR5B-1ejEd9-1}\nwmargintag{{\nwtagstyle{}\subpageref{NW2YR5B-1ejEd9-1}}}\moddef{The \code{}opsys\edoc{} utility~{\nwtagstyle{}\subpageref{NW2YR5B-1ejEd9-1}}}\endmoddef\nwstartdeflinemarkup\nwusesondefline{\\{NW2YR5B-2LnsgP-1}\\{NW2YR5B-4OPRT1-1}\\{NW2YR5B-2CaUHx-1}}\nwenddeflinemarkup
\nwlinkedidentc{opsys}{NW2YR5B-1ejEd9-1}←\{⍵⊃⍨'Win' 'Lin' 'Mac'⍳⊂3↑⊃'.'⎕WG'APLVersion'\}
\nwindexdefn{\nwixident{opsys}}{opsys}{NW2YR5B-1ejEd9-1}\eatline
\nwused{\\{NW2YR5B-2LnsgP-1}\\{NW2YR5B-4OPRT1-1}\\{NW2YR5B-2CaUHx-1}}\nwidentdefs{\\{{\nwixident{opsys}}{opsys}}}\nwendcode{}\nwbegindocs{78}\nwdocspar
\section{Developer Infrastructure}

\subsection{Building the Compiler}

The Co-dfns compiler is written, developed, and distributed as a
literate program.
For more information about literate programming,
see the resources available at \url{http://literateprogramming.com/}.
We use \href{https://www.cs.tufts.edu/~nr/noweb/}{noweb} as our
preferred literate programming tool because it is eminently simple,
while still handling the majority of our needs and producing high
quality output in \LaTeX\ format with all the important elements of
literate programming, including live hyperlinking and cross-references.

\subsubsection{Tangling the Source}

The process of tangling produces the executable source code 
for the compiler.
Importantly, the tangled output is \emph{not} meant to be used 
as the primary means of reading or debugging the source.
Instead, it is meant primarily as the machine readable version
of the code only.

With noweb, we need to invoke {\Tt{}notangle\nwendquote} once for each of the 
chunks that we wish to use to produce an output file.
To make this easy, we build up a script to do this work for us.

For Linux and Mac, the following bash script creates these files. 
We use a separate chunk that we build up incrementally 
throughout the rest of this document as a record of all the chunks
that we should create.
Notice that we explicitly tangle the {\Tt{}\nwlinkedidentq{TANGLE}{NW2YR5B-4OPRT1-1}\nwlinkedidentq{.sh}{NW2YR5B-42EjwV-1}\nwendquote} file as the last
thing that we do;
this helps to ensure that we are reliably executing the rest of the 
script before changing the contents of the file,
as some systems will be affected and change execution behavior 
in strange ways if we change the {\Tt{}\nwlinkedidentq{TANGLE}{NW2YR5B-4OPRT1-1}\nwlinkedidentq{.sh}{NW2YR5B-42EjwV-1}\nwendquote} file early on in the 
execution of the file.

\nwenddocs{}\nwbegincode{79}\sublabel{NW2YR5B-42EjwV-1}\nwmargintag{{\nwtagstyle{}\subpageref{NW2YR5B-42EjwV-1}}}\moddef{\code{}TANGLE.sh\edoc{}~{\nwtagstyle{}\subpageref{NW2YR5B-42EjwV-1}}}\endmoddef\nwstartdeflinemarkup\nwenddeflinemarkup
#!/bin/bash

\LA{}Tangle Commands~{\nwtagstyle{}\subpageref{NW2YR5B-ufqpm-1}}\RA{}

echo "Tangling \nwlinkedidentc{TANGLE}{NW2YR5B-4OPRT1-1}\nwlinkedidentc{.sh}{NW2YR5B-42EjwV-1}..."
notangle -R'[[\nwlinkedidentc{TANGLE}{NW2YR5B-4OPRT1-1}\nwlinkedidentc{.sh}{NW2YR5B-42EjwV-1}]]' \nwlinkedidentc{codfns}{NW2YR5B-1p0Y9w-1}.nw > \nwlinkedidentc{TANGLE}{NW2YR5B-4OPRT1-1}\nwlinkedidentc{.sh}{NW2YR5B-42EjwV-1}
\nwindexdefn{\nwixident{TANGLE.sh}}{TANGLE.sh}{NW2YR5B-42EjwV-1}\eatline
\nwnotused{[[TANGLE.sh]]}\nwidentdefs{\\{{\nwixident{TANGLE.sh}}{TANGLE.sh}}}\nwidentuses{\\{{\nwixident{codfns}}{codfns}}\\{{\nwixident{TANGLE}}{TANGLE}}}\nwindexuse{\nwixident{codfns}}{codfns}{NW2YR5B-42EjwV-1}\nwindexuse{\nwixident{TANGLE}}{TANGLE}{NW2YR5B-42EjwV-1}\nwendcode{}\nwbegindocs{80}\nwdocspar
On Windows, the best way that we have found to do this is
by installing noweb using the
\href{https://www.cygwin.com/}{Cygwin project}
and then calling {\Tt{}\nwlinkedidentq{TANGLE}{NW2YR5B-4OPRT1-1}\nwlinkedidentq{.sh}{NW2YR5B-42EjwV-1}\nwendquote} from a local {\Tt{}\nwlinkedidentq{TANGLE}{NW2YR5B-4OPRT1-1}\nwlinkedidentq{.bat}{NW2YR5B-YYfto-1}\nwendquote} file.
This document assumes that you have already successfully built and
installed via Cygwin a working Icon-driven noweb installation.

Users who prefer to work in a UNIX fashion via Cygwin or some other
subsystem on Windows can follow the build scripts directly.
For developers who prefer to work in a primarily Windows environment,
the following {\Tt{}\nwlinkedidentq{TANGLE}{NW2YR5B-4OPRT1-1}\nwlinkedidentq{.bat}{NW2YR5B-YYfto-1}\nwendquote} build script assists 
in handling the calls into Cygwin
so that you do not need to have a Cygwin terminal open all the time.

\nwenddocs{}\nwbegincode{81}\sublabel{NW2YR5B-YYfto-1}\nwmargintag{{\nwtagstyle{}\subpageref{NW2YR5B-YYfto-1}}}\moddef{\code{}TANGLE.bat\edoc{}~{\nwtagstyle{}\subpageref{NW2YR5B-YYfto-1}}}\endmoddef\nwstartdeflinemarkup\nwenddeflinemarkup
set SH=C:\\cygwin64\\bin\\bash.exe -l -c
%SH% "cd $OLDPWD && ./\nwlinkedidentc{TANGLE}{NW2YR5B-4OPRT1-1}\nwlinkedidentc{.sh}{NW2YR5B-42EjwV-1}"
\nwindexdefn{\nwixident{TANGLE.bat}}{TANGLE.bat}{NW2YR5B-YYfto-1}\eatline
\nwnotused{[[TANGLE.bat]]}\nwidentdefs{\\{{\nwixident{TANGLE.bat}}{TANGLE.bat}}}\nwidentuses{\\{{\nwixident{TANGLE}}{TANGLE}}\\{{\nwixident{TANGLE.sh}}{TANGLE.sh}}}\nwindexuse{\nwixident{TANGLE}}{TANGLE}{NW2YR5B-YYfto-1}\nwindexuse{\nwixident{TANGLE.sh}}{TANGLE.sh}{NW2YR5B-YYfto-1}\nwendcode{}\nwbegindocs{82}\nwdocspar
\nwenddocs{}\nwbegincode{83}\sublabel{NW2YR5B-ufqpm-4}\nwmargintag{{\nwtagstyle{}\subpageref{NW2YR5B-ufqpm-4}}}\moddef{Tangle Commands~{\nwtagstyle{}\subpageref{NW2YR5B-ufqpm-1}}}\plusendmoddef\nwstartdeflinemarkup\nwusesondefline{\\{NW2YR5B-42EjwV-1}}\nwprevnextdefs{NW2YR5B-ufqpm-3}{NW2YR5B-ufqpm-5}\nwenddeflinemarkup
echo "Tangling \nwlinkedidentc{TANGLE}{NW2YR5B-4OPRT1-1}\nwlinkedidentc{.bat}{NW2YR5B-YYfto-1}..."
notangle -R'[[\nwlinkedidentc{TANGLE}{NW2YR5B-4OPRT1-1}\nwlinkedidentc{.bat}{NW2YR5B-YYfto-1}]]' \nwlinkedidentc{codfns}{NW2YR5B-1p0Y9w-1}.nw > \nwlinkedidentc{TANGLE}{NW2YR5B-4OPRT1-1}\nwlinkedidentc{.bat}{NW2YR5B-YYfto-1}
\nwused{\\{NW2YR5B-42EjwV-1}}\nwidentuses{\\{{\nwixident{codfns}}{codfns}}\\{{\nwixident{TANGLE}}{TANGLE}}\\{{\nwixident{TANGLE.bat}}{TANGLE.bat}}}\nwindexuse{\nwixident{codfns}}{codfns}{NW2YR5B-ufqpm-4}\nwindexuse{\nwixident{TANGLE}}{TANGLE}{NW2YR5B-ufqpm-4}\nwindexuse{\nwixident{TANGLE.bat}}{TANGLE.bat}{NW2YR5B-ufqpm-4}\nwendcode{}\nwbegindocs{84}\nwdocspar

When tangled to the {\Tt{}\nwlinkedidentq{TANGLE}{NW2YR5B-4OPRT1-1}\nwlinkedidentq{.aplf}{NW2YR5B-ufqpm-5}\nwendquote} file,
the following script enables the
user to simply type {\Tt{}\nwlinkedidentq{TANGLE}{NW2YR5B-4OPRT1-1}\nwendquote} within a Dyalog APL session
to update the code tree from within Dyalog itself.
This is much more convenient than keeping a Cygwin Terminal
session open along with a Dyalog APL session while programming.

\emph{Note: this command expects to be run from within the root of
the repository, not from, say, within the {\Tt{}testing\nwendquote} directory.}

\nwenddocs{}\nwbegincode{85}\sublabel{NW2YR5B-4OPRT1-1}\nwmargintag{{\nwtagstyle{}\subpageref{NW2YR5B-4OPRT1-1}}}\moddef{\code{}TANGLE\edoc{}~{\nwtagstyle{}\subpageref{NW2YR5B-4OPRT1-1}}}\endmoddef\nwstartdeflinemarkup\nwenddeflinemarkup
\nwlinkedidentc{TANGLE}{NW2YR5B-4OPRT1-1};\nwlinkedidentc{opsys}{NW2YR5B-1ejEd9-1}
\LA{}The \code{}opsys\edoc{} utility~{\nwtagstyle{}\subpageref{NW2YR5B-1ejEd9-1}}\RA{}
⎕CMD \nwlinkedidentc{opsys}{NW2YR5B-1ejEd9-1} '.\\\nwlinkedidentc{TANGLE}{NW2YR5B-4OPRT1-1}.bat' './\nwlinkedidentc{TANGLE}{NW2YR5B-4OPRT1-1}.sh' './\nwlinkedidentc{TANGLE}{NW2YR5B-4OPRT1-1}.sh'
\nwindexdefn{\nwixident{TANGLE}}{TANGLE}{NW2YR5B-4OPRT1-1}\eatline
\nwnotused{[[TANGLE]]}\nwidentdefs{\\{{\nwixident{TANGLE}}{TANGLE}}}\nwidentuses{\\{{\nwixident{opsys}}{opsys}}}\nwindexuse{\nwixident{opsys}}{opsys}{NW2YR5B-4OPRT1-1}\nwendcode{}\nwbegindocs{86}\nwdocspar
\nwenddocs{}\nwbegincode{87}\sublabel{NW2YR5B-ufqpm-5}\nwmargintag{{\nwtagstyle{}\subpageref{NW2YR5B-ufqpm-5}}}\moddef{Tangle Commands~{\nwtagstyle{}\subpageref{NW2YR5B-ufqpm-1}}}\plusendmoddef\nwstartdeflinemarkup\nwusesondefline{\\{NW2YR5B-42EjwV-1}}\nwprevnextdefs{NW2YR5B-ufqpm-4}{NW2YR5B-ufqpm-6}\nwenddeflinemarkup
echo "Tangling \nwlinkedidentc{TANGLE}{NW2YR5B-4OPRT1-1}\nwlinkedidentc{.aplf}{NW2YR5B-ufqpm-5}..."
notangle -R'[[\nwlinkedidentc{TANGLE}{NW2YR5B-4OPRT1-1}]]' \nwlinkedidentc{codfns}{NW2YR5B-1p0Y9w-1}.nw > \nwlinkedidentc{src}{NW2YR5B-3C6SQT-1}/\nwlinkedidentc{TANGLE}{NW2YR5B-4OPRT1-1}\nwlinkedidentc{.aplf}{NW2YR5B-ufqpm-5}
\nwindexdefn{\nwixident{TANGLE.aplf}}{TANGLE.aplf}{NW2YR5B-ufqpm-5}\eatline
\nwused{\\{NW2YR5B-42EjwV-1}}\nwidentdefs{\\{{\nwixident{TANGLE.aplf}}{TANGLE.aplf}}}\nwidentuses{\\{{\nwixident{codfns}}{codfns}}\\{{\nwixident{src}}{src}}\\{{\nwixident{TANGLE}}{TANGLE}}}\nwindexuse{\nwixident{codfns}}{codfns}{NW2YR5B-ufqpm-5}\nwindexuse{\nwixident{src}}{src}{NW2YR5B-ufqpm-5}\nwindexuse{\nwixident{TANGLE}}{TANGLE}{NW2YR5B-ufqpm-5}\nwendcode{}\nwbegindocs{88}\nwdocspar
\subsubsection{Weaving the Source}

Weaving is the process by which we produce the final printed output
of this document,
intended for reading and general human consumption.
We rely on the \LaTeX\ typesetting system to do this.
Moreover, because we make heavy use of UTF-8 and prefer to have our
own fonts installed and used,
it is necessary to use the {\Tt{}xelatex\nwendquote} system instead of the typical
\LaTeX\ engine.
In order to get the indexing right, we must run the engine twice.
The first run will update the indexing files that will be picked
up on the second run and incorporated into the final document.
Note, we have tried to use the {\Tt{}lualatex\nwendquote} engine, which in theory
should work just as well as the {\Tt{}xelatex\nwendquote} engine, but we get a
strange error relating to noweb's style file, so we stick with
{\Tt{}xelatex\nwendquote} for now.

Running this script also depends on having the appropriate fonts
installed.
In this case, please ensure that the following fonts are installed
in your Windows font system so that they can be picked up by the \TeX\
engine.

\begin{itemize}
        \item Libre Baskerville (Regular, Italic, Bold)
        \item APL385 Unicode
        \item Lucida Sans Unicode
        \item Cambria Math
\end{itemize}

\noindent
If you do not wish to use these fonts, then see the top of the
{\Tt{}\nwlinkedidentq{codfns}{NW2YR5B-1p0Y9w-1}.nw\nwendquote} file and edit the font specifications to the fonts that
you do wish to use.

Note the use of {\Tt{}-delay\ -index\nwendquote} for options. We want to generate
indexing, but we also need to make sure that we can use some of our
own packages in the system,

\emph{Note: this command expects to be run from within the root of
the repository, not from, say, within the {\Tt{}testing\nwendquote} directory.}

\nwenddocs{}\nwbegincode{89}\sublabel{NW2YR5B-40bMLy-1}\nwmargintag{{\nwtagstyle{}\subpageref{NW2YR5B-40bMLy-1}}}\moddef{\code{}WEAVE.sh\edoc{}~{\nwtagstyle{}\subpageref{NW2YR5B-40bMLy-1}}}\endmoddef\nwstartdeflinemarkup\nwenddeflinemarkup
#!/bin/bash
mkdir -p woven
noweave -delay -index \nwlinkedidentc{codfns}{NW2YR5B-1p0Y9w-1}.nw > woven/\nwlinkedidentc{codfns}{NW2YR5B-1p0Y9w-1}.tex
cd woven
xelatex \nwlinkedidentc{codfns}{NW2YR5B-1p0Y9w-1}
xelatex \nwlinkedidentc{codfns}{NW2YR5B-1p0Y9w-1}
\nwindexdefn{\nwixident{WEAVE.sh}}{WEAVE.sh}{NW2YR5B-40bMLy-1}\eatline
\nwnotused{[[WEAVE.sh]]}\nwidentdefs{\\{{\nwixident{WEAVE.sh}}{WEAVE.sh}}}\nwidentuses{\\{{\nwixident{codfns}}{codfns}}}\nwindexuse{\nwixident{codfns}}{codfns}{NW2YR5B-40bMLy-1}\nwendcode{}\nwbegindocs{90}\nwdocspar
\nwenddocs{}\nwbegincode{91}\sublabel{NW2YR5B-ufqpm-6}\nwmargintag{{\nwtagstyle{}\subpageref{NW2YR5B-ufqpm-6}}}\moddef{Tangle Commands~{\nwtagstyle{}\subpageref{NW2YR5B-ufqpm-1}}}\plusendmoddef\nwstartdeflinemarkup\nwusesondefline{\\{NW2YR5B-42EjwV-1}}\nwprevnextdefs{NW2YR5B-ufqpm-5}{NW2YR5B-ufqpm-7}\nwenddeflinemarkup
echo "Tangling \nwlinkedidentc{WEAVE}{NW2YR5B-2CaUHx-1}\nwlinkedidentc{.sh}{NW2YR5B-40bMLy-1}..."
notangle -R'[[\nwlinkedidentc{WEAVE}{NW2YR5B-2CaUHx-1}\nwlinkedidentc{.sh}{NW2YR5B-40bMLy-1}]]' \nwlinkedidentc{codfns}{NW2YR5B-1p0Y9w-1}.nw > \nwlinkedidentc{WEAVE}{NW2YR5B-2CaUHx-1}\nwlinkedidentc{.sh}{NW2YR5B-40bMLy-1}
\nwused{\\{NW2YR5B-42EjwV-1}}\nwidentuses{\\{{\nwixident{codfns}}{codfns}}\\{{\nwixident{WEAVE}}{WEAVE}}\\{{\nwixident{WEAVE.sh}}{WEAVE.sh}}}\nwindexuse{\nwixident{codfns}}{codfns}{NW2YR5B-ufqpm-6}\nwindexuse{\nwixident{WEAVE}}{WEAVE}{NW2YR5B-ufqpm-6}\nwindexuse{\nwixident{WEAVE.sh}}{WEAVE.sh}{NW2YR5B-ufqpm-6}\nwendcode{}\nwbegindocs{92}\nwdocspar

\noindent
And just like the tangling code, we want to define a {\Tt{}\nwlinkedidentq{TANGLE}{NW2YR5B-4OPRT1-1}\nwlinkedidentq{.bat}{NW2YR5B-YYfto-1}\nwendquote}
batch file to call the Cygwin environment from Windows.

\nwenddocs{}\nwbegincode{93}\sublabel{NW2YR5B-3fkVpu-1}\nwmargintag{{\nwtagstyle{}\subpageref{NW2YR5B-3fkVpu-1}}}\moddef{\code{}WEAVE.bat\edoc{}~{\nwtagstyle{}\subpageref{NW2YR5B-3fkVpu-1}}}\endmoddef\nwstartdeflinemarkup\nwenddeflinemarkup
set SH=C:\\cygwin64\\bin\\bash.exe -l -c
%SH% "cd $OLDPWD && ./\nwlinkedidentc{WEAVE}{NW2YR5B-2CaUHx-1}\nwlinkedidentc{.sh}{NW2YR5B-40bMLy-1}"
\nwindexdefn{\nwixident{WEAVE.bat}}{WEAVE.bat}{NW2YR5B-3fkVpu-1}\eatline
\nwnotused{[[WEAVE.bat]]}\nwidentdefs{\\{{\nwixident{WEAVE.bat}}{WEAVE.bat}}}\nwidentuses{\\{{\nwixident{WEAVE}}{WEAVE}}\\{{\nwixident{WEAVE.sh}}{WEAVE.sh}}}\nwindexuse{\nwixident{WEAVE}}{WEAVE}{NW2YR5B-3fkVpu-1}\nwindexuse{\nwixident{WEAVE.sh}}{WEAVE.sh}{NW2YR5B-3fkVpu-1}\nwendcode{}\nwbegindocs{94}\nwdocspar
\nwenddocs{}\nwbegincode{95}\sublabel{NW2YR5B-ufqpm-7}\nwmargintag{{\nwtagstyle{}\subpageref{NW2YR5B-ufqpm-7}}}\moddef{Tangle Commands~{\nwtagstyle{}\subpageref{NW2YR5B-ufqpm-1}}}\plusendmoddef\nwstartdeflinemarkup\nwusesondefline{\\{NW2YR5B-42EjwV-1}}\nwprevnextdefs{NW2YR5B-ufqpm-6}{NW2YR5B-ufqpm-8}\nwenddeflinemarkup
echo "Tangling \nwlinkedidentc{WEAVE}{NW2YR5B-2CaUHx-1}\nwlinkedidentc{.bat}{NW2YR5B-3fkVpu-1}..."
notangle -R'[[\nwlinkedidentc{WEAVE}{NW2YR5B-2CaUHx-1}\nwlinkedidentc{.bat}{NW2YR5B-3fkVpu-1}]]' \nwlinkedidentc{codfns}{NW2YR5B-1p0Y9w-1}.nw > \nwlinkedidentc{WEAVE}{NW2YR5B-2CaUHx-1}\nwlinkedidentc{.bat}{NW2YR5B-3fkVpu-1}
\nwused{\\{NW2YR5B-42EjwV-1}}\nwidentuses{\\{{\nwixident{codfns}}{codfns}}\\{{\nwixident{WEAVE}}{WEAVE}}\\{{\nwixident{WEAVE.bat}}{WEAVE.bat}}}\nwindexuse{\nwixident{codfns}}{codfns}{NW2YR5B-ufqpm-7}\nwindexuse{\nwixident{WEAVE}}{WEAVE}{NW2YR5B-ufqpm-7}\nwindexuse{\nwixident{WEAVE.bat}}{WEAVE.bat}{NW2YR5B-ufqpm-7}\nwendcode{}\nwbegindocs{96}\nwdocspar

Like the {\Tt{}\LA{}\code{}TANGLE\edoc{} Command~{\nwtagstyle{}\subpageref{nw@notdef}}\RA{}\nwendquote}, the following command,
when tangled to the {\Tt{}\nwlinkedidentq{WEAVE}{NW2YR5B-2CaUHx-1}\nwlinkedidentq{.aplf}{NW2YR5B-ufqpm-8}\nwendquote}
file enables weaving in a the Dyalog APL session 
by executing the {\Tt{}\nwlinkedidentq{WEAVE}{NW2YR5B-2CaUHx-1}\nwendquote} command.

\nwenddocs{}\nwbegincode{97}\sublabel{NW2YR5B-2CaUHx-1}\nwmargintag{{\nwtagstyle{}\subpageref{NW2YR5B-2CaUHx-1}}}\moddef{\code{}WEAVE\edoc{}~{\nwtagstyle{}\subpageref{NW2YR5B-2CaUHx-1}}}\endmoddef\nwstartdeflinemarkup\nwenddeflinemarkup
\nwlinkedidentc{WEAVE}{NW2YR5B-2CaUHx-1};\nwlinkedidentc{opsys}{NW2YR5B-1ejEd9-1}
\LA{}The \code{}opsys\edoc{} utility~{\nwtagstyle{}\subpageref{NW2YR5B-1ejEd9-1}}\RA{}
⎕CMD \nwlinkedidentc{opsys}{NW2YR5B-1ejEd9-1} '.\\\nwlinkedidentc{WEAVE}{NW2YR5B-2CaUHx-1}.bat' './\nwlinkedidentc{WEAVE}{NW2YR5B-2CaUHx-1}.sh' './\nwlinkedidentc{WEAVE}{NW2YR5B-2CaUHx-1}.sh'
\nwindexdefn{\nwixident{WEAVE}}{WEAVE}{NW2YR5B-2CaUHx-1}\eatline
\nwnotused{[[WEAVE]]}\nwidentdefs{\\{{\nwixident{WEAVE}}{WEAVE}}}\nwidentuses{\\{{\nwixident{opsys}}{opsys}}}\nwindexuse{\nwixident{opsys}}{opsys}{NW2YR5B-2CaUHx-1}\nwendcode{}\nwbegindocs{98}\nwdocspar
\nwenddocs{}\nwbegincode{99}\sublabel{NW2YR5B-ufqpm-8}\nwmargintag{{\nwtagstyle{}\subpageref{NW2YR5B-ufqpm-8}}}\moddef{Tangle Commands~{\nwtagstyle{}\subpageref{NW2YR5B-ufqpm-1}}}\plusendmoddef\nwstartdeflinemarkup\nwusesondefline{\\{NW2YR5B-42EjwV-1}}\nwprevnextdefs{NW2YR5B-ufqpm-7}{NW2YR5B-ufqpm-9}\nwenddeflinemarkup
echo "Tangling \nwlinkedidentc{src}{NW2YR5B-3C6SQT-1}/\nwlinkedidentc{WEAVE}{NW2YR5B-2CaUHx-1}\nwlinkedidentc{.aplf}{NW2YR5B-ufqpm-8}..."
notangle -R'[[\nwlinkedidentc{WEAVE}{NW2YR5B-2CaUHx-1}]]' \nwlinkedidentc{codfns}{NW2YR5B-1p0Y9w-1}.nw > \nwlinkedidentc{src}{NW2YR5B-3C6SQT-1}/\nwlinkedidentc{WEAVE}{NW2YR5B-2CaUHx-1}\nwlinkedidentc{.aplf}{NW2YR5B-ufqpm-8}
\nwindexdefn{\nwixident{WEAVE.aplf}}{WEAVE.aplf}{NW2YR5B-ufqpm-8}\eatline
\nwused{\\{NW2YR5B-42EjwV-1}}\nwidentdefs{\\{{\nwixident{WEAVE.aplf}}{WEAVE.aplf}}}\nwidentuses{\\{{\nwixident{codfns}}{codfns}}\\{{\nwixident{src}}{src}}\\{{\nwixident{WEAVE}}{WEAVE}}}\nwindexuse{\nwixident{codfns}}{codfns}{NW2YR5B-ufqpm-8}\nwindexuse{\nwixident{src}}{src}{NW2YR5B-ufqpm-8}\nwindexuse{\nwixident{WEAVE}}{WEAVE}{NW2YR5B-ufqpm-8}\nwendcode{}\nwbegindocs{100}\nwdocspar
\subsection{Building the Runtime}

One of our goals with the Co-dfns runtime is to write as much of it
as possible in APL.
This means that we want to have at minimum a very small kernel that
has been written in C,
while most of the rest of the code is implemented in some APL files.
This leads to a three part breakdown of the process to
build the runtime.

\nwenddocs{}\nwbegincode{101}\sublabel{NW2YR5B-8A0Gv-1}\nwmargintag{{\nwtagstyle{}\subpageref{NW2YR5B-8A0Gv-1}}}\moddef{Build the runtime~{\nwtagstyle{}\subpageref{NW2YR5B-8A0Gv-1}}}\endmoddef\nwstartdeflinemarkup\nwusesondefline{\\{NW2YR5B-66tOQ-1}}\nwenddeflinemarkup
\LA{}Compile the primitives in \code{}prim.apln\edoc{}~{\nwtagstyle{}\subpageref{NW2YR5B-3C6SQT-1}}\RA{}
\LA{}Build \code{}codfns.dll\edoc{} DLL~{\nwtagstyle{}\subpageref{NW2YR5B-2pNilx-1}}\RA{}
\LA{}Copy the runtime files into \code{}tests{\nwbackslash}\edoc{}~{\nwtagstyle{}\subpageref{NW2YR5B-1PDY4d-1}}\RA{}
\nwused{\\{NW2YR5B-66tOQ-1}}\nwendcode{}\nwbegindocs{102}\nwdocspar

We define the command {\Tt{}\nwlinkedidentq{MK∆RTM}{NW2YR5B-66tOQ-1}\nwendquote} to build the runtime.
This command takes a path to the root directory of the Co-dfns
repository; this is to allow us to rebuild the runtime from anywhere
in the system if we so choose.

\nwenddocs{}\nwbegincode{103}\sublabel{NW2YR5B-66tOQ-1}\nwmargintag{{\nwtagstyle{}\subpageref{NW2YR5B-66tOQ-1}}}\moddef{\code{}MK∆RTM\edoc{}~{\nwtagstyle{}\subpageref{NW2YR5B-66tOQ-1}}}\endmoddef\nwstartdeflinemarkup\nwenddeflinemarkup
\nwlinkedidentc{MK∆RTM}{NW2YR5B-66tOQ-1} path;\nwlinkedidentc{put}{NW2YR5B-XAz19-1};\nwlinkedidentc{tie}{NW2YR5B-XAz19-1};\nwlinkedidentc{src}{NW2YR5B-3C6SQT-1};\nwlinkedidentc{vsbat}{NW2YR5B-2pNilx-1};\nwlinkedidentc{vsc}{NW2YR5B-2pNilx-1};\nwlinkedidentc{wsd}{NW2YR5B-2pNilx-1}

\LA{}Basic \code{}tie\edoc{} and \code{}put\edoc{} utilities~{\nwtagstyle{}\subpageref{NW2YR5B-XAz19-1}}\RA{}
\LA{}Build the runtime~{\nwtagstyle{}\subpageref{NW2YR5B-8A0Gv-1}}\RA{}
\nwindexdefn{\nwixident{MK∆RTM}}{MK∆RTM}{NW2YR5B-66tOQ-1}\eatline
\nwnotused{[[MK∆RTM]]}\nwidentdefs{\\{{\nwixident{MK∆RTM}}{MK∆RTM}}}\nwidentuses{\\{{\nwixident{put}}{put}}\\{{\nwixident{src}}{src}}\\{{\nwixident{tie}}{tie}}\\{{\nwixident{vsbat}}{vsbat}}\\{{\nwixident{vsc}}{vsc}}\\{{\nwixident{wsd}}{wsd}}}\nwindexuse{\nwixident{put}}{put}{NW2YR5B-66tOQ-1}\nwindexuse{\nwixident{src}}{src}{NW2YR5B-66tOQ-1}\nwindexuse{\nwixident{tie}}{tie}{NW2YR5B-66tOQ-1}\nwindexuse{\nwixident{vsbat}}{vsbat}{NW2YR5B-66tOQ-1}\nwindexuse{\nwixident{vsc}}{vsc}{NW2YR5B-66tOQ-1}\nwindexuse{\nwixident{wsd}}{wsd}{NW2YR5B-66tOQ-1}\nwendcode{}\nwbegindocs{104}\nwdocspar
This file is another of our external utilities that exists outside 
of the {\Tt{}\nwlinkedidentq{codfns}{NW2YR5B-1p0Y9w-1}\nwendquote} namespace, so it gets its own file in {\Tt{}\nwlinkedidentq{src}{NW2YR5B-3C6SQT-1}{\nwbackslash}\nwendquote}.

\nwenddocs{}\nwbegincode{105}\sublabel{NW2YR5B-ufqpm-9}\nwmargintag{{\nwtagstyle{}\subpageref{NW2YR5B-ufqpm-9}}}\moddef{Tangle Commands~{\nwtagstyle{}\subpageref{NW2YR5B-ufqpm-1}}}\plusendmoddef\nwstartdeflinemarkup\nwusesondefline{\\{NW2YR5B-42EjwV-1}}\nwprevnextdefs{NW2YR5B-ufqpm-8}{\relax}\nwenddeflinemarkup
echo "Tangling \nwlinkedidentc{src}{NW2YR5B-3C6SQT-1}/\nwlinkedidentc{MK∆RTM}{NW2YR5B-66tOQ-1}\nwlinkedidentc{.aplf}{NW2YR5B-ufqpm-9}..."
notangle -R'[[\nwlinkedidentc{MK∆RTM}{NW2YR5B-66tOQ-1}]]' \nwlinkedidentc{codfns}{NW2YR5B-1p0Y9w-1}.nw > \nwlinkedidentc{src}{NW2YR5B-3C6SQT-1}/\nwlinkedidentc{MK∆RTM}{NW2YR5B-66tOQ-1}\nwlinkedidentc{.aplf}{NW2YR5B-ufqpm-9}
\nwindexdefn{\nwixident{MK∆RTM.aplf}}{MK∆RTM.aplf}{NW2YR5B-ufqpm-9}\eatline
\nwused{\\{NW2YR5B-42EjwV-1}}\nwidentdefs{\\{{\nwixident{MK∆RTM.aplf}}{MK∆RTM.aplf}}}\nwidentuses{\\{{\nwixident{codfns}}{codfns}}\\{{\nwixident{MK∆RTM}}{MK∆RTM}}\\{{\nwixident{src}}{src}}}\nwindexuse{\nwixident{codfns}}{codfns}{NW2YR5B-ufqpm-9}\nwindexuse{\nwixident{MK∆RTM}}{MK∆RTM}{NW2YR5B-ufqpm-9}\nwindexuse{\nwixident{src}}{src}{NW2YR5B-ufqpm-9}\nwendcode{}\nwbegindocs{106}\nwdocspar
The first step we must take is producing an appropriate C file that
contains the primitives that we have defined in {\Tt{}prim.apln\nwendquote}.
This means that we want to only compile the code in {\Tt{}prim.apln\nwendquote}
as far as producing the C code.
Since we do not have a full blown runtime yet,
we will be compiling the {\Tt{}prim.c\nwendquote} file along with the rest of the
runtime code,
instead of the normal build process,
which assumes that we already have a working runtime.
This means that we only invoke the {\Tt{}GC\ TT\ \nwlinkedidentq{PS}{NW2YR5B-38DvvD-1}\nwendquote} passes of the
compiler pipeline, while avoiding the {\Tt{}CC\nwendquote} pass.
We use the SALT system to load the source from {\Tt{}prim.apln\nwendquote} and then
run the compiler passes that we want before storing the resulting
code in the {\Tt{}rtm{\nwbackslash}prim.c\nwendquote} file.

\nwenddocs{}\nwbegincode{107}\sublabel{NW2YR5B-3C6SQT-1}\nwmargintag{{\nwtagstyle{}\subpageref{NW2YR5B-3C6SQT-1}}}\moddef{Compile the primitives in \code{}prim.apln\edoc{}~{\nwtagstyle{}\subpageref{NW2YR5B-3C6SQT-1}}}\endmoddef\nwstartdeflinemarkup\nwusesondefline{\\{NW2YR5B-8A0Gv-1}}\nwenddeflinemarkup
\nwlinkedidentc{src}{NW2YR5B-3C6SQT-1}←⎕SRC ⎕SE.SALT.Load path,'\\rtm\\prim.apln'
(path,'\\rtm\\prim.c')\nwlinkedidentc{put}{NW2YR5B-XAz19-1} \nwlinkedidentc{codfns}{NW2YR5B-1p0Y9w-1}.\{GC TT \nwlinkedidentc{PS}{NW2YR5B-38DvvD-1} ⍵\}\nwlinkedidentc{src}{NW2YR5B-3C6SQT-1}
\nwindexdefn{\nwixident{src}}{src}{NW2YR5B-3C6SQT-1}\eatline
\nwused{\\{NW2YR5B-8A0Gv-1}}\nwidentdefs{\\{{\nwixident{src}}{src}}}\nwidentuses{\\{{\nwixident{codfns}}{codfns}}\\{{\nwixident{PS}}{PS}}\\{{\nwixident{put}}{put}}}\nwindexuse{\nwixident{codfns}}{codfns}{NW2YR5B-3C6SQT-1}\nwindexuse{\nwixident{PS}}{PS}{NW2YR5B-3C6SQT-1}\nwindexuse{\nwixident{put}}{put}{NW2YR5B-3C6SQT-1}\nwendcode{}\nwbegindocs{108}\nwdocspar
Once we have the {\Tt{}rtm{\nwbackslash}prim.c\nwendquote} file written appropriately,
we can run the main compiler process.
For simplicity, we just compile all of the {\Tt{}.c\nwendquote} files that
are found in the {\Tt{}rtm{\nwbackslash}\nwendquote} subdirectory.
We must ensure that we are appropriatelly invoking our ArrayFire
dependencies as well as producing the appropriate debugging symbols
most of the time.

\nwenddocs{}\nwbegincode{109}\sublabel{NW2YR5B-2pNilx-1}\nwmargintag{{\nwtagstyle{}\subpageref{NW2YR5B-2pNilx-1}}}\moddef{Build \code{}codfns.dll\edoc{} DLL~{\nwtagstyle{}\subpageref{NW2YR5B-2pNilx-1}}}\endmoddef\nwstartdeflinemarkup\nwusesondefline{\\{NW2YR5B-8A0Gv-1}}\nwenddeflinemarkup
\nwlinkedidentc{vsbat}{NW2YR5B-2pNilx-1}←#.\nwlinkedidentc{codfns}{NW2YR5B-1p0Y9w-1}.\nwlinkedidentc{VS∆PATH}{NW2YR5B-2z4lmm-4}
\nwlinkedidentc{vsbat}{NW2YR5B-2pNilx-1},'\\VC\\Auxiliary\\Build\\vcvarsall.bat'
\nwlinkedidentc{wsd}{NW2YR5B-2pNilx-1}←path,'\\'

\nwlinkedidentc{vsc}{NW2YR5B-2pNilx-1}←'%comspec% /C ""',\nwlinkedidentc{vsbat}{NW2YR5B-2pNilx-1},'" amd64'
\nwlinkedidentc{vsc}{NW2YR5B-2pNilx-1},←'  && cd "',\nwlinkedidentc{wsd}{NW2YR5B-2pNilx-1},'\\rtm"'
\nwlinkedidentc{vsc}{NW2YR5B-2pNilx-1},←'  && cl /MP /W3 /wd4102 /wd4275'
\nwlinkedidentc{vsc}{NW2YR5B-2pNilx-1},←'    /Od /Zc:inline /Zi /FS'
\nwlinkedidentc{vsc}{NW2YR5B-2pNilx-1},←'    /Fo".\\\\" /Fd"\nwlinkedidentc{codfns}{NW2YR5B-1p0Y9w-1}.pdb"'
\nwlinkedidentc{vsc}{NW2YR5B-2pNilx-1},←'    /WX /MD /EHsc /nologo'
\nwlinkedidentc{vsc}{NW2YR5B-2pNilx-1},←'    /I"%AF_PATH%\\include"'
\nwlinkedidentc{vsc}{NW2YR5B-2pNilx-1},←'    /D"NOMINMAX" /D"AF_DEBUG" /D"EXPORTING"'
\nwlinkedidentc{vsc}{NW2YR5B-2pNilx-1},←'    "*.c" /link /DLL /OPT:REF'
\nwlinkedidentc{vsc}{NW2YR5B-2pNilx-1},←'    /INCREMENTAL:NO /SUBSYSTEM:WINDOWS'
\nwlinkedidentc{vsc}{NW2YR5B-2pNilx-1},←'    /LIBPATH:"%AF_PATH%\\lib"'
\nwlinkedidentc{vsc}{NW2YR5B-2pNilx-1},←'    /DYNAMICBASE "af',\nwlinkedidentc{codfns}{NW2YR5B-1p0Y9w-1}.\nwlinkedidentc{AF∆LIB}{NW2YR5B-2z4lmm-3},'.lib"'
\nwlinkedidentc{vsc}{NW2YR5B-2pNilx-1},←'    /OPT:ICF /ERRORREPORT:PROMPT'
\nwlinkedidentc{vsc}{NW2YR5B-2pNilx-1},←'    /TLBID:1 /OUT:"\nwlinkedidentc{codfns}{NW2YR5B-1p0Y9w-1}.dll""'
\nwindexdefn{\nwixident{vsbat}}{vsbat}{NW2YR5B-2pNilx-1}\nwindexdefn{\nwixident{wsd}}{wsd}{NW2YR5B-2pNilx-1}\nwindexdefn{\nwixident{vsc}}{vsc}{NW2YR5B-2pNilx-1}\eatline
\nwused{\\{NW2YR5B-8A0Gv-1}}\nwidentdefs{\\{{\nwixident{vsbat}}{vsbat}}\\{{\nwixident{vsc}}{vsc}}\\{{\nwixident{wsd}}{wsd}}}\nwidentuses{\\{{\nwixident{AF∆LIB}}{AF∆LIB}}\\{{\nwixident{codfns}}{codfns}}\\{{\nwixident{VS∆PATH}}{VS∆PATH}}}\nwindexuse{\nwixident{AF∆LIB}}{AF∆LIB}{NW2YR5B-2pNilx-1}\nwindexuse{\nwixident{codfns}}{codfns}{NW2YR5B-2pNilx-1}\nwindexuse{\nwixident{VS∆PATH}}{VS∆PATH}{NW2YR5B-2pNilx-1}\nwendcode{}\nwbegindocs{110}\nwdocspar
Finally, in order to write up the test harness to work right,
we must copy the appropriate runtime files into the {\Tt{}tests{\nwbackslash}\nwendquote}
directory so that we can find them when we finally start running
our code there.

\nwenddocs{}\nwbegincode{111}\sublabel{NW2YR5B-1PDY4d-1}\nwmargintag{{\nwtagstyle{}\subpageref{NW2YR5B-1PDY4d-1}}}\moddef{Copy the runtime files into \code{}tests{\nwbackslash}\edoc{}~{\nwtagstyle{}\subpageref{NW2YR5B-1PDY4d-1}}}\endmoddef\nwstartdeflinemarkup\nwusesondefline{\\{NW2YR5B-8A0Gv-1}}\nwenddeflinemarkup
⎕CMD ⎕←\nwlinkedidentc{vsc}{NW2YR5B-2pNilx-1}
⎕CMD ⎕←'copy "',\nwlinkedidentc{wsd}{NW2YR5B-2pNilx-1},'rtm\\\nwlinkedidentc{codfns}{NW2YR5B-1p0Y9w-1}.h" "',\nwlinkedidentc{wsd}{NW2YR5B-2pNilx-1},'tests\\"'
⎕CMD ⎕←'copy "',\nwlinkedidentc{wsd}{NW2YR5B-2pNilx-1},'rtm\\\nwlinkedidentc{codfns}{NW2YR5B-1p0Y9w-1}.exp" "',\nwlinkedidentc{wsd}{NW2YR5B-2pNilx-1},'tests\\"'
⎕CMD ⎕←'copy "',\nwlinkedidentc{wsd}{NW2YR5B-2pNilx-1},'rtm\\\nwlinkedidentc{codfns}{NW2YR5B-1p0Y9w-1}.lib" "',\nwlinkedidentc{wsd}{NW2YR5B-2pNilx-1},'tests\\"'
⎕CMD ⎕←'copy "',\nwlinkedidentc{wsd}{NW2YR5B-2pNilx-1},'rtm\\\nwlinkedidentc{codfns}{NW2YR5B-1p0Y9w-1}.pdb" "',\nwlinkedidentc{wsd}{NW2YR5B-2pNilx-1},'tests\\"'
⎕CMD ⎕←'copy "',\nwlinkedidentc{wsd}{NW2YR5B-2pNilx-1},'rtm\\\nwlinkedidentc{codfns}{NW2YR5B-1p0Y9w-1}.dll" "',\nwlinkedidentc{wsd}{NW2YR5B-2pNilx-1},'tests\\"'
\nwused{\\{NW2YR5B-8A0Gv-1}}\nwidentuses{\\{{\nwixident{codfns}}{codfns}}\\{{\nwixident{vsc}}{vsc}}\\{{\nwixident{wsd}}{wsd}}}\nwindexuse{\nwixident{codfns}}{codfns}{NW2YR5B-1PDY4d-1}\nwindexuse{\nwixident{vsc}}{vsc}{NW2YR5B-1PDY4d-1}\nwindexuse{\nwixident{wsd}}{wsd}{NW2YR5B-1PDY4d-1}\nwendcode{}

\nwixlogsorted{c}{{*}{NW2YR5B-1p0Y9w-1}{\nwixd{NW2YR5B-1p0Y9w-1}}}%
\nwixlogsorted{c}{{\code{}DISPLAY\edoc{} Utility}{NW2YR5B-2qLMFZ-1}{\nwixd{NW2YR5B-2qLMFZ-1}}}%
\nwixlogsorted{c}{{\code{}MK∆RTM\edoc{}}{NW2YR5B-66tOQ-1}{\nwixd{NW2YR5B-66tOQ-1}}}%
\nwixlogsorted{c}{{\code{}PP\edoc{} Utility}{NW2YR5B-39J7A7-1}{\nwixd{NW2YR5B-39J7A7-1}}}%
\nwixlogsorted{c}{{\code{}TANGLE.bat\edoc{}}{NW2YR5B-YYfto-1}{\nwixd{NW2YR5B-YYfto-1}}}%
\nwixlogsorted{c}{{\code{}TANGLE.sh\edoc{}}{NW2YR5B-42EjwV-1}{\nwixd{NW2YR5B-42EjwV-1}}}%
\nwixlogsorted{c}{{\code{}TANGLE\edoc{}}{NW2YR5B-4OPRT1-1}{\nwixd{NW2YR5B-4OPRT1-1}}}%
\nwixlogsorted{c}{{\code{}TEST\edoc{}}{NW2YR5B-4gH67U-1}{\nwixd{NW2YR5B-4gH67U-1}}}%
\nwixlogsorted{c}{{\code{}WEAVE.bat\edoc{}}{NW2YR5B-3fkVpu-1}{\nwixd{NW2YR5B-3fkVpu-1}}}%
\nwixlogsorted{c}{{\code{}WEAVE.sh\edoc{}}{NW2YR5B-40bMLy-1}{\nwixd{NW2YR5B-40bMLy-1}}}%
\nwixlogsorted{c}{{\code{}WEAVE\edoc{}}{NW2YR5B-2CaUHx-1}{\nwixd{NW2YR5B-2CaUHx-1}}}%
\nwixlogsorted{c}{{Adjust AST for output}{NW2YR5B-1K2n9O-1}{\nwixu{NW2YR5B-38DvvD-1}\nwixd{NW2YR5B-1K2n9O-1}\nwixd{NW2YR5B-1K2n9O-2}\nwixd{NW2YR5B-1K2n9O-3}\nwixd{NW2YR5B-1K2n9O-4}}}%
\nwixlogsorted{c}{{AST Record Structure}{NW2YR5B-1gMT0G-1}{\nwixu{NW2YR5B-1p0Y9w-1}\nwixd{NW2YR5B-1gMT0G-1}}}%
\nwixlogsorted{c}{{Basic \code{}tie\edoc{} and \code{}put\edoc{} utilities}{NW2YR5B-XAz19-1}{\nwixd{NW2YR5B-XAz19-1}\nwixu{NW2YR5B-66tOQ-1}}}%
\nwixlogsorted{c}{{Build \code{}codfns.dll\edoc{} DLL}{NW2YR5B-2pNilx-1}{\nwixu{NW2YR5B-8A0Gv-1}\nwixd{NW2YR5B-2pNilx-1}}}%
\nwixlogsorted{c}{{Build the runtime}{NW2YR5B-8A0Gv-1}{\nwixd{NW2YR5B-8A0Gv-1}\nwixu{NW2YR5B-66tOQ-1}}}%
\nwixlogsorted{c}{{C Runtime Header}{NW2YR5B-2mMS19-1}{\nwixd{NW2YR5B-2mMS19-1}}}%
\nwixlogsorted{c}{{C Runtime Support}{NW2YR5B-3ZoAJL-1}{\nwixd{NW2YR5B-3ZoAJL-1}}}%
\nwixlogsorted{c}{{Code Generator}{NW2YR5B-HCURD-1}{\nwixu{NW2YR5B-1p0Y9w-1}\nwixd{NW2YR5B-HCURD-1}}}%
\nwixlogsorted{c}{{Compile the primitives in \code{}prim.apln\edoc{}}{NW2YR5B-3C6SQT-1}{\nwixu{NW2YR5B-8A0Gv-1}\nwixd{NW2YR5B-3C6SQT-1}}}%
\nwixlogsorted{c}{{Compiler}{NW2YR5B-1jC2QX-1}{\nwixu{NW2YR5B-1p0Y9w-1}\nwixd{NW2YR5B-1jC2QX-1}}}%
\nwixlogsorted{c}{{Converters between parent and depth vectors}{NW2YR5B-3dpNce-1}{\nwixu{NW2YR5B-1p0Y9w-1}\nwixd{NW2YR5B-3dpNce-1}}}%
\nwixlogsorted{c}{{Copy the runtime files into \code{}tests{\nwbackslash}\edoc{}}{NW2YR5B-1PDY4d-1}{\nwixu{NW2YR5B-8A0Gv-1}\nwixd{NW2YR5B-1PDY4d-1}}}%
\nwixlogsorted{c}{{Global Settings}{NW2YR5B-2z4lmm-1}{\nwixu{NW2YR5B-1p0Y9w-1}\nwixd{NW2YR5B-2z4lmm-1}\nwixd{NW2YR5B-2z4lmm-2}\nwixd{NW2YR5B-2z4lmm-3}\nwixd{NW2YR5B-2z4lmm-4}}}%
\nwixlogsorted{c}{{Implementation of APL Primitives}{NW2YR5B-941K7-1}{\nwixd{NW2YR5B-941K7-1}}}%
\nwixlogsorted{c}{{Interface to the backend C compiler}{NW2YR5B-2LnsgP-1}{\nwixu{NW2YR5B-1p0Y9w-1}\nwixd{NW2YR5B-2LnsgP-1}}}%
\nwixlogsorted{c}{{Line and error reporting utilities}{NW2YR5B-aELRs-1}{\nwixu{NW2YR5B-38DvvD-1}\nwixd{NW2YR5B-aELRs-1}}}%
\nwixlogsorted{c}{{Linking with Dyalog}{NW2YR5B-zH457-1}{\nwixu{NW2YR5B-1p0Y9w-1}\nwixd{NW2YR5B-zH457-1}}}%
\nwixlogsorted{c}{{Must Have APL Utilities}{NW2YR5B-AF6fz-1}{\nwixu{NW2YR5B-1p0Y9w-1}\nwixd{NW2YR5B-AF6fz-1}}}%
\nwixlogsorted{c}{{Normalize the input formatting}{NW2YR5B-32ArUy-1}{\nwixu{NW2YR5B-13WClx-1}\nwixd{NW2YR5B-32ArUy-1}}}%
\nwixlogsorted{c}{{Parse token stream}{NW2YR5B-4CIf4o-1}{\nwixu{NW2YR5B-38DvvD-1}\nwixd{NW2YR5B-4CIf4o-1}}}%
\nwixlogsorted{c}{{Parser}{NW2YR5B-38DvvD-1}{\nwixu{NW2YR5B-1p0Y9w-1}\nwixd{NW2YR5B-38DvvD-1}}}%
\nwixlogsorted{c}{{Parsing Constants}{NW2YR5B-3563XI-1}{\nwixu{NW2YR5B-38DvvD-1}\nwixd{NW2YR5B-3563XI-1}}}%
\nwixlogsorted{c}{{Pretty-printing AST trees}{NW2YR5B-1VExi3-1}{\nwixu{NW2YR5B-1p0Y9w-1}\nwixd{NW2YR5B-1VExi3-1}}}%
\nwixlogsorted{c}{{Tangle Commands}{NW2YR5B-ufqpm-1}{\nwixd{NW2YR5B-ufqpm-1}\nwixd{NW2YR5B-ufqpm-2}\nwixd{NW2YR5B-ufqpm-3}\nwixu{NW2YR5B-42EjwV-1}\nwixd{NW2YR5B-ufqpm-4}\nwixd{NW2YR5B-ufqpm-5}\nwixd{NW2YR5B-ufqpm-6}\nwixd{NW2YR5B-ufqpm-7}\nwixd{NW2YR5B-ufqpm-8}\nwixd{NW2YR5B-ufqpm-9}}}%
\nwixlogsorted{c}{{The \code{}opsys\edoc{} utility}{NW2YR5B-1ejEd9-1}{\nwixu{NW2YR5B-2LnsgP-1}\nwixd{NW2YR5B-1ejEd9-1}\nwixu{NW2YR5B-4OPRT1-1}\nwixu{NW2YR5B-2CaUHx-1}}}%
\nwixlogsorted{c}{{The Fix API}{NW2YR5B-2o6hoR-1}{\nwixu{NW2YR5B-1p0Y9w-1}\nwixd{NW2YR5B-2o6hoR-1}}}%
\nwixlogsorted{c}{{Tokenize input}{NW2YR5B-1192i8-1}{\nwixu{NW2YR5B-38DvvD-1}\nwixd{NW2YR5B-1192i8-1}}}%
\nwixlogsorted{c}{{User-command API}{NW2YR5B-2YFL86-1}{\nwixu{NW2YR5B-1p0Y9w-1}\nwixd{NW2YR5B-2YFL86-1}}}%
\nwixlogsorted{c}{{Verify source input \code{}⍵\edoc{}, set \code{}IN\edoc{}}{NW2YR5B-13WClx-1}{\nwixd{NW2YR5B-13WClx-1}\nwixu{NW2YR5B-38DvvD-1}}}%
\nwixlogsorted{c}{{XML Rendering}{NW2YR5B-1evvdc-1}{\nwixu{NW2YR5B-1p0Y9w-1}\nwixd{NW2YR5B-1evvdc-1}}}%
\nwixlogsorted{i}{{\nwixident{AF∆LIB}}{AF∆LIB}}%
\nwixlogsorted{i}{{\nwixident{AF∆PREFIX}}{AF∆PREFIX}}%
\nwixlogsorted{i}{{\nwixident{assert}}{assert}}%
\nwixlogsorted{i}{{\nwixident{codfns}}{codfns}}%
\nwixlogsorted{i}{{\nwixident{codfns.apln}}{codfns.apln}}%
\nwixlogsorted{i}{{\nwixident{dct}}{dct}}%
\nwixlogsorted{i}{{\nwixident{DISPLAY}}{DISPLAY}}%
\nwixlogsorted{i}{{\nwixident{DISPLAY.aplf}}{DISPLAY.aplf}}%
\nwixlogsorted{i}{{\nwixident{dlk}}{dlk}}%
\nwixlogsorted{i}{{\nwixident{dwh}}{dwh}}%
\nwixlogsorted{i}{{\nwixident{dwv}}{dwv}}%
\nwixlogsorted{i}{{\nwixident{Fix}}{Fix}}%
\nwixlogsorted{i}{{\nwixident{lb3}}{lb3}}%
\nwixlogsorted{i}{{\nwixident{linestarts}}{linestarts}}%
\nwixlogsorted{i}{{\nwixident{mkdm}}{mkdm}}%
\nwixlogsorted{i}{{\nwixident{MK∆RTM}}{MK∆RTM}}%
\nwixlogsorted{i}{{\nwixident{MK∆RTM.aplf}}{MK∆RTM.aplf}}%
\nwixlogsorted{i}{{\nwixident{opsys}}{opsys}}%
\nwixlogsorted{i}{{\nwixident{PP}}{PP}}%
\nwixlogsorted{i}{{\nwixident{PP.aplf}}{PP.aplf}}%
\nwixlogsorted{i}{{\nwixident{pp3}}{pp3}}%
\nwixlogsorted{i}{{\nwixident{PS}}{PS}}%
\nwixlogsorted{i}{{\nwixident{put}}{put}}%
\nwixlogsorted{i}{{\nwixident{quotelines}}{quotelines}}%
\nwixlogsorted{i}{{\nwixident{SIGNAL}}{SIGNAL}}%
\nwixlogsorted{i}{{\nwixident{src}}{src}}%
\nwixlogsorted{i}{{\nwixident{TANGLE}}{TANGLE}}%
\nwixlogsorted{i}{{\nwixident{TANGLE.aplf}}{TANGLE.aplf}}%
\nwixlogsorted{i}{{\nwixident{TANGLE.bat}}{TANGLE.bat}}%
\nwixlogsorted{i}{{\nwixident{TANGLE.sh}}{TANGLE.sh}}%
\nwixlogsorted{i}{{\nwixident{TEST}}{TEST}}%
\nwixlogsorted{i}{{\nwixident{TEST.aplf}}{TEST.aplf}}%
\nwixlogsorted{i}{{\nwixident{tie}}{tie}}%
\nwixlogsorted{i}{{\nwixident{VERSION}}{VERSION}}%
\nwixlogsorted{i}{{\nwixident{vsbat}}{vsbat}}%
\nwixlogsorted{i}{{\nwixident{vsc}}{vsc}}%
\nwixlogsorted{i}{{\nwixident{VS∆PATH}}{VS∆PATH}}%
\nwixlogsorted{i}{{\nwixident{WEAVE}}{WEAVE}}%
\nwixlogsorted{i}{{\nwixident{WEAVE.aplf}}{WEAVE.aplf}}%
\nwixlogsorted{i}{{\nwixident{WEAVE.bat}}{WEAVE.bat}}%
\nwixlogsorted{i}{{\nwixident{WEAVE.sh}}{WEAVE.sh}}%
\nwixlogsorted{i}{{\nwixident{wsd}}{wsd}}%
\nwixlogsorted{i}{{\nwixident{Xml}}{Xml}}%
\nwixlogsorted{i}{{\nwixident{xn}}{xn}}%
\nwixlogsorted{i}{{\nwixident{xt}}{xt}}%
\nwixlogsorted{i}{{\nwixident{⎕IO}}{⎕IO}}%
\nwixlogsorted{i}{{\nwixident{⎕ML}}{⎕ML}}%
\nwixlogsorted{i}{{\nwixident{⎕WX}}{⎕WX}}%
\nwbegindocs{112}\nwdocspar

\subsection{Loading the Compiler}

In order to load the compiler into an APL session as well as all the
development utilities,
we assume that you have first managed to either load up a session
with a bootstrapped version of the {\Tt{}\nwlinkedidentq{TANGLE}{NW2YR5B-4OPRT1-1}\nwendquote} command or that you
already have a tangled {\Tt{}\nwlinkedidentq{src}{NW2YR5B-3C6SQT-1}{\nwbackslash}\nwendquote} directory.
If the {\Tt{}\nwlinkedidentq{src}{NW2YR5B-3C6SQT-1}{\nwbackslash}\nwendquote} directory has not yet been created by running the
{\Tt{}\nwlinkedidentq{TANGLE}{NW2YR5B-4OPRT1-1}\nwendquote} command,
then this must be done before loading the compiler system.
After tangling,
the compiler can be loaded using the provided {\Tt{}LOAD\nwendquote} shortcut.
This shortcut is meant to use the
\href{https://github.com/Dyalog/link}{Dyalog Link}
system for hot-loading the files in {\Tt{}\nwlinkedidentq{src}{NW2YR5B-3C6SQT-1}{\nwbackslash}\nwendquote} into the root namespace.
We do so through the following link command:

\begin{verbatim}
Link.Create # src -source=dir -watch=dir
\end{verbatim}

\noindent
This means that we want to link the {\Tt{}\nwlinkedidentq{src}{NW2YR5B-3C6SQT-1}{\nwbackslash}\nwendquote} directory into the {\Tt{}{\#}\nwendquote}
namespace,
but we also want to make sure that we only pull changes that come
from the filesystem.
This is because we are editing the code via the WEB document,
and we do not want to risk having some intermediate representation
that isn't accurate and that doesn't flow the right way;
we want all appropriate changes to begin in the WEB document
and then, and only then, flow into the session.
This also allows us to make some modifications to the code for testing
and experimentation inside of the session without consideration
for the code outside of the session,
and such changes will be removed or forgotten on the next {\Tt{}\nwlinkedidentq{TANGLE}{NW2YR5B-4OPRT1-1}\nwendquote}
command.

To set this up, we also ensure that we begin our work within the
root Co-dfns repository directory, as this is where we expect to run
the {\Tt{}\nwlinkedidentq{TANGLE}{NW2YR5B-4OPRT1-1}\nwendquote} and {\Tt{}\nwlinkedidentq{WEAVE}{NW2YR5B-2CaUHx-1}\nwendquote} commands.

There is unfortunately only a limited range of possibilities for
linking in a new directory as we wish to do.
The method we choose to use is launching a fresh Dyalog APL session
and then using an {\Tt{}LX\nwendquote} expression from the command line
to do the actual linking using the {\Tt{}⎕SE.UCMD\nwendquote} functionality.
I personally find this to be rather hackish, and I hope that an
alternative approach to doing this will show up in the near future.
Nonetheless, the arguments that we pass to {\Tt{}dyalog.exe\nwendquote}
look something like this:

\begin{verbatim}
LX="⎕SE.UCMD'Link.Create # src -source=dir -watch=dir'"
\end{verbatim}

If you do not use the {\Tt{}LOAD\nwendquote} shortcut, you can use the above
command to do the linking manually.

\section{Index}

\subsection{Chunks}

\nowebchunks

\subsection{Identifiers}

\nowebindex

\clearpage
\section{GNU AFFERO GENERAL PUBLIC LICENSE}

\begin{center}
{\parindent 0in

Version 3, 19 November 2007

Copyright \copyright\    2007 Free Software Foundation, Inc. \texttt{https://fsf.org/}

\bigskip
Everyone is permitted to copy and distribute verbatim copies of this
license document, but changing it is not allowed.}

\end{center}

\begin{center}
{\Large \sc Preamble}
\end{center}

The GNU Affero General Public License is a free, copyleft license
for software and other kinds of works, specifically designed to ensure
cooperation with the community in the case of network server software.

The licenses for most software and other practical works are
designed to take away your freedom to share and change the works.        By
contrast, our General Public Licenses are intended to guarantee your
freedom to share and change all versions of a program--to make sure it
remains free software for all its users.

When we speak of free software, we are referring to freedom, not
price.  Our General Public Licenses are designed to make sure that you
have the freedom to distribute copies of free software (and charge for
them if you wish), that you receive source code or can get it if you
want it, that you can change the software or use pieces of it in new
free programs, and that you know you can do these things.

Developers that use our General Public Licenses protect your rights
with two steps: (1) assert copyright on the software, and (2) offer
you this License which gives you legal permission to copy, distribute
and/or modify the software.

A secondary benefit of defending all users' freedom is that
improvements made in alternate versions of the program, if they
receive widespread use, become available for other developers to
incorporate.    Many developers of free software are heartened and
encouraged by the resulting cooperation.        However, in the case of
software used on network servers, this result may fail to come about.
The GNU General Public License permits making a modified version and
letting the public access it on a server without ever releasing its
source code to the public.

The GNU Affero General Public License is designed specifically to
ensure that, in such cases, the modified source code becomes available
to the community.        It requires the operator of a network server to
provide the source code of the modified version running there to the
users of that server.    Therefore, public use of a modified version, on
a publicly accessible server, gives the public access to the source
code of the modified version.

An older license, called the Affero General Public License and
published by Affero, was designed to accomplish similar goals.  This is
a different license, not a version of the Affero GPL, but Affero has
released a new version of the Affero GPL which permits relicensing under
this license.

The precise terms and conditions for copying, distribution and
modification follow.

\begin{center}
{\Large \sc Terms and Conditions}
\end{center}

\begin{enumerate}

\addtocounter{enumi}{-1}

\item Definitions.

``This License'' refers to version 3 of the GNU Affero General Public License.

``Copyright'' also means copyright-like laws that apply to other kinds of
works, such as semiconductor masks.

``The Program'' refers to any copyrightable work licensed under this
License.        Each licensee is addressed as ``you''.  ``Licensees'' and
``recipients'' may be individuals or organizations.

To ``modify'' a work means to copy from or adapt all or part of the work
in a fashion requiring copyright permission, other than the making of an
exact copy.      The resulting work is called a ``modified version'' of the
earlier work or a work ``based on'' the earlier work.

A ``covered work'' means either the unmodified Program or a work based
on the Program.

To ``propagate'' a work means to do anything with it that, without
permission, would make you directly or secondarily liable for
infringement under applicable copyright law, except executing it on a
computer or modifying a private copy.    Propagation includes copying,
distribution (with or without modification), making available to the
public, and in some countries other activities as well.

To ``convey'' a work means any kind of propagation that enables other
parties to make or receive copies.      Mere interaction with a user through
a computer network, with no transfer of a copy, is not conveying.

An interactive user interface displays ``Appropriate Legal Notices''
to the extent that it includes a convenient and prominently visible
feature that (1) displays an appropriate copyright notice, and (2)
tells the user that there is no warranty for the work (except to the
extent that warranties are provided), that licensees may convey the
work under this License, and how to view a copy of this License.        If
the interface presents a list of user commands or options, such as a
menu, a prominent item in the list meets this criterion.

\item Source Code.

The ``source code'' for a work means the preferred form of the work
for making modifications to it.  ``Object code'' means any non-source
form of a work.

A ``Standard Interface'' means an interface that either is an official
standard defined by a recognized standards body, or, in the case of
interfaces specified for a particular programming language, one that
is widely used among developers working in that language.

The ``System Libraries'' of an executable work include anything, other
than the work as a whole, that (a) is included in the normal form of
packaging a Major Component, but which is not part of that Major
Component, and (b) serves only to enable use of the work with that
Major Component, or to implement a Standard Interface for which an
implementation is available to the public in source code form.  A
``Major Component'', in this context, means a major essential component
(kernel, window system, and so on) of the specific operating system
(if any) on which the executable work runs, or a compiler used to
produce the work, or an object code interpreter used to run it.

The ``Corresponding Source'' for a work in object code form means all
the source code needed to generate, install, and (for an executable
work) run the object code and to modify the work, including scripts to
control those activities.        However, it does not include the work's
System Libraries, or general-purpose tools or generally available free
programs which are used unmodified in performing those activities but
which are not part of the work.  For example, Corresponding Source
includes interface definition files associated with source files for
the work, and the source code for shared libraries and dynamically
linked subprograms that the work is specifically designed to require,
such as by intimate data communication or control flow between those
subprograms and other parts of the work.

The Corresponding Source need not include anything that users
can regenerate automatically from other parts of the Corresponding
Source.

The Corresponding Source for a work in source code form is that
same work.

\item Basic Permissions.

All rights granted under this License are granted for the term of
copyright on the Program, and are irrevocable provided the stated
conditions are met.      This License explicitly affirms your unlimited
permission to run the unmodified Program.        The output from running a
covered work is covered by this License only if the output, given its
content, constitutes a covered work.    This License acknowledges your
rights of fair use or other equivalent, as provided by copyright law.

You may make, run and propagate covered works that you do not
convey, without conditions so long as your license otherwise remains
in force.        You may convey covered works to others for the sole purpose
of having them make modifications exclusively for you, or provide you
with facilities for running those works, provided that you comply with
the terms of this License in conveying all material for which you do
not control copyright.  Those thus making or running the covered works
for you must do so exclusively on your behalf, under your direction
and control, on terms that prohibit them from making any copies of
your copyrighted material outside their relationship with you.

Conveying under any other circumstances is permitted solely under
the conditions stated below.    Sublicensing is not allowed; section 10
makes it unnecessary.

\item Protecting Users' Legal Rights From Anti-Circumvention Law.

No covered work shall be deemed part of an effective technological
measure under any applicable law fulfilling obligations under article
11 of the WIPO copyright treaty adopted on 20 December 1996, or
similar laws prohibiting or restricting circumvention of such
measures.

When you convey a covered work, you waive any legal power to forbid
circumvention of technological measures to the extent such circumvention
is effected by exercising rights under this License with respect to
the covered work, and you disclaim any intention to limit operation or
modification of the work as a means of enforcing, against the work's
users, your or third parties' legal rights to forbid circumvention of
technological measures.

\item Conveying Verbatim Copies.

You may convey verbatim copies of the Program's source code as you
receive it, in any medium, provided that you conspicuously and
appropriately publish on each copy an appropriate copyright notice;
keep intact all notices stating that this License and any
non-permissive terms added in accord with section 7 apply to the code;
keep intact all notices of the absence of any warranty; and give all
recipients a copy of this License along with the Program.

You may charge any price or no price for each copy that you convey,
and you may offer support or warranty protection for a fee.

\item Conveying Modified Source Versions.

You may convey a work based on the Program, or the modifications to
produce it from the Program, in the form of source code under the
terms of section 4, provided that you also meet all of these conditions:
        \begin{enumerate}
        \item The work must carry prominent notices stating that you modified
        it, and giving a relevant date.

        \item The work must carry prominent notices stating that it is
        released under this License and any conditions added under section
        7.      This requirement modifies the requirement in section 4 to
        ``keep intact all notices''.

        \item You must license the entire work, as a whole, under this
        License to anyone who comes into possession of a copy.  This
        License will therefore apply, along with any applicable section 7
        additional terms, to the whole of the work, and all its parts,
        regardless of how they are packaged.    This License gives no
        permission to license the work in any other way, but it does not
        invalidate such permission if you have separately received it.

        \item If the work has interactive user interfaces, each must display
        Appropriate Legal Notices; however, if the Program has interactive
        interfaces that do not display Appropriate Legal Notices, your
        work need not make them do so.
\end{enumerate}
A compilation of a covered work with other separate and independent
works, which are not by their nature extensions of the covered work,
and which are not combined with it such as to form a larger program,
in or on a volume of a storage or distribution medium, is called an
``aggregate'' if the compilation and its resulting copyright are not
used to limit the access or legal rights of the compilation's users
beyond what the individual works permit.        Inclusion of a covered work
in an aggregate does not cause this License to apply to the other
parts of the aggregate.

\item Conveying Non-Source Forms.

You may convey a covered work in object code form under the terms
of sections 4 and 5, provided that you also convey the
machine-readable Corresponding Source under the terms of this License,
in one of these ways:
        \begin{enumerate}
        \item Convey the object code in, or embodied in, a physical product
        (including a physical distribution medium), accompanied by the
        Corresponding Source fixed on a durable physical medium
        customarily used for software interchange.

        \item Convey the object code in, or embodied in, a physical product
        (including a physical distribution medium), accompanied by a
        written offer, valid for at least three years and valid for as
        long as you offer spare parts or customer support for that product
        model, to give anyone who possesses the object code either (1) a
        copy of the Corresponding Source for all the software in the
        product that is covered by this License, on a durable physical
        medium customarily used for software interchange, for a price no
        more than your reasonable cost of physically performing this
        conveying of source, or (2) access to copy the
        Corresponding Source from a network server at no charge.

        \item Convey individual copies of the object code with a copy of the
        written offer to provide the Corresponding Source.      This
        alternative is allowed only occasionally and noncommercially, and
        only if you received the object code with such an offer, in accord
        with subsection 6b.

        \item Convey the object code by offering access from a designated
        place (gratis or for a charge), and offer equivalent access to the
        Corresponding Source in the same way through the same place at no
        further charge.  You need not require recipients to copy the
        Corresponding Source along with the object code.        If the place to
        copy the object code is a network server, the Corresponding Source
        may be on a different server (operated by you or a third party)
        that supports equivalent copying facilities, provided you maintain
        clear directions next to the object code saying where to find the
        Corresponding Source.    Regardless of what server hosts the
        Corresponding Source, you remain obligated to ensure that it is
        available for as long as needed to satisfy these requirements.

        \item Convey the object code using peer-to-peer transmission, provided
        you inform other peers where the object code and Corresponding
        Source of the work are being offered to the general public at no
        charge under subsection 6d.
        \end{enumerate}

A separable portion of the object code, whose source code is excluded
from the Corresponding Source as a System Library, need not be
included in conveying the object code work.

A ``User Product'' is either (1) a ``consumer product'', which means any
tangible personal property which is normally used for personal, family,
or household purposes, or (2) anything designed or sold for incorporation
into a dwelling.        In determining whether a product is a consumer product,
doubtful cases shall be resolved in favor of coverage.  For a particular
product received by a particular user, ``normally used'' refers to a
typical or common use of that class of product, regardless of the status
of the particular user or of the way in which the particular user
actually uses, or expects or is expected to use, the product.    A product
is a consumer product regardless of whether the product has substantial
commercial, industrial or non-consumer uses, unless such uses represent
the only significant mode of use of the product.

``Installation Information'' for a User Product means any methods,
procedures, authorization keys, or other information required to install
and execute modified versions of a covered work in that User Product from
a modified version of its Corresponding Source.  The information must
suffice to ensure that the continued functioning of the modified object
code is in no case prevented or interfered with solely because
modification has been made.

If you convey an object code work under this section in, or with, or
specifically for use in, a User Product, and the conveying occurs as
part of a transaction in which the right of possession and use of the
User Product is transferred to the recipient in perpetuity or for a
fixed term (regardless of how the transaction is characterized), the
Corresponding Source conveyed under this section must be accompanied
by the Installation Information.        But this requirement does not apply
if neither you nor any third party retains the ability to install
modified object code on the User Product (for example, the work has
been installed in ROM).

The requirement to provide Installation Information does not include a
requirement to continue to provide support service, warranty, or updates
for a work that has been modified or installed by the recipient, or for
the User Product in which it has been modified or installed.    Access to a
network may be denied when the modification itself materially and
adversely affects the operation of the network or violates the rules and
protocols for communication across the network.

Corresponding Source conveyed, and Installation Information provided,
in accord with this section must be in a format that is publicly
documented (and with an implementation available to the public in
source code form), and must require no special password or key for
unpacking, reading or copying.

\item Additional Terms.

``Additional permissions'' are terms that supplement the terms of this
License by making exceptions from one or more of its conditions.
Additional permissions that are applicable to the entire Program shall
be treated as though they were included in this License, to the extent
that they are valid under applicable law.        If additional permissions
apply only to part of the Program, that part may be used separately
under those permissions, but the entire Program remains governed by
this License without regard to the additional permissions.

When you convey a copy of a covered work, you may at your option
remove any additional permissions from that copy, or from any part of
it.      (Additional permissions may be written to require their own
removal in certain cases when you modify the work.)      You may place
additional permissions on material, added by you to a covered work,
for which you have or can give appropriate copyright permission.

Notwithstanding any other provision of this License, for material you
add to a covered work, you may (if authorized by the copyright holders of
that material) supplement the terms of this License with terms:
        \begin{enumerate}
        \item Disclaiming warranty or limiting liability differently from the
        terms of sections 15 and 16 of this License; or

        \item Requiring preservation of specified reasonable legal notices or
        author attributions in that material or in the Appropriate Legal
        Notices displayed by works containing it; or

        \item Prohibiting misrepresentation of the origin of that material, or
        requiring that modified versions of such material be marked in
        reasonable ways as different from the original version; or

        \item Limiting the use for publicity purposes of names of licensors or
        authors of the material; or

        \item Declining to grant rights under trademark law for use of some
        trade names, trademarks, or service marks; or

        \item Requiring indemnification of licensors and authors of that
        material by anyone who conveys the material (or modified versions of
        it) with contractual assumptions of liability to the recipient, for
        any liability that these contractual assumptions directly impose on
        those licensors and authors.
        \end{enumerate}

All other non-permissive additional terms are considered ``further
restrictions'' within the meaning of section 10.        If the Program as you
received it, or any part of it, contains a notice stating that it is
governed by this License along with a term that is a further
restriction, you may remove that term.  If a license document contains
a further restriction but permits relicensing or conveying under this
License, you may add to a covered work material governed by the terms
of that license document, provided that the further restriction does
not survive such relicensing or conveying.

If you add terms to a covered work in accord with this section, you
must place, in the relevant source files, a statement of the
additional terms that apply to those files, or a notice indicating
where to find the applicable terms.

Additional terms, permissive or non-permissive, may be stated in the
form of a separately written license, or stated as exceptions;
the above requirements apply either way.

\item Termination.

You may not propagate or modify a covered work except as expressly
provided under this License.    Any attempt otherwise to propagate or
modify it is void, and will automatically terminate your rights under
this License (including any patent licenses granted under the third
paragraph of section 11).

However, if you cease all violation of this License, then your
license from a particular copyright holder is reinstated (a)
provisionally, unless and until the copyright holder explicitly and
finally terminates your license, and (b) permanently, if the copyright
holder fails to notify you of the violation by some reasonable means
prior to 60 days after the cessation.

Moreover, your license from a particular copyright holder is
reinstated permanently if the copyright holder notifies you of the
violation by some reasonable means, this is the first time you have
received notice of violation of this License (for any work) from that
copyright holder, and you cure the violation prior to 30 days after
your receipt of the notice.

Termination of your rights under this section does not terminate the
licenses of parties who have received copies or rights from you under
this License.    If your rights have been terminated and not permanently
reinstated, you do not qualify to receive new licenses for the same
material under section 10.

\item Acceptance Not Required for Having Copies.

You are not required to accept this License in order to receive or
run a copy of the Program.      Ancillary propagation of a covered work
occurring solely as a consequence of using peer-to-peer transmission
to receive a copy likewise does not require acceptance.  However,
nothing other than this License grants you permission to propagate or
modify any covered work.        These actions infringe copyright if you do
not accept this License.        Therefore, by modifying or propagating a
covered work, you indicate your acceptance of this License to do so.

\item Automatic Licensing of Downstream Recipients.

Each time you convey a covered work, the recipient automatically
receives a license from the original licensors, to run, modify and
propagate that work, subject to this License.    You are not responsible
for enforcing compliance by third parties with this License.

An ``entity transaction'' is a transaction transferring control of an
organization, or substantially all assets of one, or subdividing an
organization, or merging organizations.  If propagation of a covered
work results from an entity transaction, each party to that
transaction who receives a copy of the work also receives whatever
licenses to the work the party's predecessor in interest had or could
give under the previous paragraph, plus a right to possession of the
Corresponding Source of the work from the predecessor in interest, if
the predecessor has it or can get it with reasonable efforts.

You may not impose any further restrictions on the exercise of the
rights granted or affirmed under this License.  For example, you may
not impose a license fee, royalty, or other charge for exercise of
rights granted under this License, and you may not initiate litigation
(including a cross-claim or counterclaim in a lawsuit) alleging that
any patent claim is infringed by making, using, selling, offering for
sale, or importing the Program or any portion of it.

\item Patents.

A ``contributor'' is a copyright holder who authorizes use under this
License of the Program or a work on which the Program is based.  The
work thus licensed is called the contributor's ``contributor version''.

A contributor's ``essential patent claims'' are all patent claims
owned or controlled by the contributor, whether already acquired or
hereafter acquired, that would be infringed by some manner, permitted
by this License, of making, using, or selling its contributor version,
but do not include claims that would be infringed only as a
consequence of further modification of the contributor version.  For
purposes of this definition, ``control'' includes the right to grant
patent sublicenses in a manner consistent with the requirements of
this License.

Each contributor grants you a non-exclusive, worldwide, royalty-free
patent license under the contributor's essential patent claims, to
make, use, sell, offer for sale, import and otherwise run, modify and
propagate the contents of its contributor version.

In the following three paragraphs, a ``patent license'' is any express
agreement or commitment, however denominated, not to enforce a patent
(such as an express permission to practice a patent or covenant not to
sue for patent infringement).    To ``grant'' such a patent license to a
party means to make such an agreement or commitment not to enforce a
patent against the party.

If you convey a covered work, knowingly relying on a patent license,
and the Corresponding Source of the work is not available for anyone
to copy, free of charge and under the terms of this License, through a
publicly available network server or other readily accessible means,
then you must either (1) cause the Corresponding Source to be so
available, or (2) arrange to deprive yourself of the benefit of the
patent license for this particular work, or (3) arrange, in a manner
consistent with the requirements of this License, to extend the patent
license to downstream recipients.        ``Knowingly relying'' means you have
actual knowledge that, but for the patent license, your conveying the
covered work in a country, or your recipient's use of the covered work
in a country, would infringe one or more identifiable patents in that
country that you have reason to believe are valid.

If, pursuant to or in connection with a single transaction or
arrangement, you convey, or propagate by procuring conveyance of, a
covered work, and grant a patent license to some of the parties
receiving the covered work authorizing them to use, propagate, modify
or convey a specific copy of the covered work, then the patent license
you grant is automatically extended to all recipients of the covered
work and works based on it.

A patent license is ``discriminatory'' if it does not include within
the scope of its coverage, prohibits the exercise of, or is
conditioned on the non-exercise of one or more of the rights that are
specifically granted under this License.        You may not convey a covered
work if you are a party to an arrangement with a third party that is
in the business of distributing software, under which you make payment
to the third party based on the extent of your activity of conveying
the work, and under which the third party grants, to any of the
parties who would receive the covered work from you, a discriminatory
patent license (a) in connection with copies of the covered work
conveyed by you (or copies made from those copies), or (b) primarily
for and in connection with specific products or compilations that
contain the covered work, unless you entered into that arrangement,
or that patent license was granted, prior to 28 March 2007.

Nothing in this License shall be construed as excluding or limiting
any implied license or other defenses to infringement that may
otherwise be available to you under applicable patent law.

\item No Surrender of Others' Freedom.

If conditions are imposed on you (whether by court order, agreement or
otherwise) that contradict the conditions of this License, they do not
excuse you from the conditions of this License.  If you cannot convey a
covered work so as to satisfy simultaneously your obligations under this
License and any other pertinent obligations, then as a consequence you may
not convey it at all.    For example, if you agree to terms that obligate you
to collect a royalty for further conveying from those to whom you convey
the Program, the only way you could satisfy both those terms and this
License would be to refrain entirely from conveying the Program.

\item Remote Network Interaction; Use with the GNU General Public License.

Notwithstanding any other provision of this License, if you modify the
Program, your modified version must prominently offer all users interacting
with it remotely through a computer network (if your version supports such
interaction) an opportunity to receive the Corresponding Source of your
version by providing access to the Corresponding Source from a network
server at no charge, through some standard or customary means of
facilitating copying of software.        This Corresponding Source shall include
the Corresponding Source for any work covered by version 3 of the GNU
General Public License that is incorporated pursuant to the following
paragraph.

Notwithstanding any other provision of this License, you have permission to
link or combine any covered work with a work licensed under version 3 of
the GNU General Public License into a single combined work, and to convey
the resulting work.      The terms of this License will continue to apply to
the part which is the covered work, but the work with which it is combined
will remain governed by version 3 of the GNU General Public License.

\item Revised Versions of this License.

The Free Software Foundation may publish revised and/or new versions of
the GNU Affero General Public License from time to time.        Such new versions will
be similar in spirit to the present version, but may differ in detail to
address new problems or concerns.

Each version is given a distinguishing version number.  If the
Program specifies that a certain numbered version of the GNU Affero General
Public License ``or any later version'' applies to it, you have the
option of following the terms and conditions either of that numbered
version or of any later version published by the Free Software
Foundation.      If the Program does not specify a version number of the
GNU Affero General Public License, you may choose any version ever published
by the Free Software Foundation.

If the Program specifies that a proxy can decide which future
versions of the GNU Affero General Public License can be used, that proxy's
public statement of acceptance of a version permanently authorizes you
to choose that version for the Program.

Later license versions may give you additional or different
permissions.    However, no additional obligations are imposed on any
author or copyright holder as a result of your choosing to follow a
later version.

\item Disclaimer of Warranty.

\begin{sloppypar}
 THERE IS NO WARRANTY FOR THE PROGRAM, TO THE EXTENT PERMITTED BY
 APPLICABLE LAW.        EXCEPT WHEN OTHERWISE STATED IN WRITING THE
 COPYRIGHT HOLDERS AND/OR OTHER PARTIES PROVIDE THE PROGRAM ``AS IS''
 WITHOUT WARRANTY OF ANY KIND, EITHER EXPRESSED OR IMPLIED,
 INCLUDING, BUT NOT LIMITED TO, THE IMPLIED WARRANTIES OF
 MERCHANTABILITY AND FITNESS FOR A PARTICULAR PURPOSE.  THE ENTIRE
 RISK AS TO THE QUALITY AND PERFORMANCE OF THE PROGRAM IS WITH YOU.
 SHOULD THE PROGRAM PROVE DEFECTIVE, YOU ASSUME THE COST OF ALL
 NECESSARY SERVICING, REPAIR OR CORRECTION.
\end{sloppypar}

\item Limitation of Liability.

 IN NO EVENT UNLESS REQUIRED BY APPLICABLE LAW OR AGREED TO IN
 WRITING WILL ANY COPYRIGHT HOLDER, OR ANY OTHER PARTY WHO MODIFIES
 AND/OR CONVEYS THE PROGRAM AS PERMITTED ABOVE, BE LIABLE TO YOU FOR
 DAMAGES, INCLUDING ANY GENERAL, SPECIAL, INCIDENTAL OR CONSEQUENTIAL
 DAMAGES ARISING OUT OF THE USE OR INABILITY TO USE THE PROGRAM
 (INCLUDING BUT NOT LIMITED TO LOSS OF DATA OR DATA BEING RENDERED
 INACCURATE OR LOSSES SUSTAINED BY YOU OR THIRD PARTIES OR A FAILURE
 OF THE PROGRAM TO OPERATE WITH ANY OTHER PROGRAMS), EVEN IF SUCH
 HOLDER OR OTHER PARTY HAS BEEN ADVISED OF THE POSSIBILITY OF SUCH
 DAMAGES.

\item Interpretation of Sections 15 and 16.

If the disclaimer of warranty and limitation of liability provided
above cannot be given local legal effect according to their terms,
reviewing courts shall apply local law that most closely approximates
an absolute waiver of all civil liability in connection with the
Program, unless a warranty or assumption of liability accompanies a
copy of the Program in return for a fee.

\begin{center}
{\Large\sc End of Terms and Conditions}

\bigskip
How to Apply These Terms to Your New Programs
\end{center}

If you develop a new program, and you want it to be of the greatest
possible use to the public, the best way to achieve this is to make it
free software which everyone can redistribute and change under these terms.

To do so, attach the following notices to the program.  It is safest
to attach them to the start of each source file to most effectively
state the exclusion of warranty; and each file should have at least
the ``copyright'' line and a pointer to where the full notice is found.

{\footnotesize
\begin{verbatim}
<one line to give the program's name and a brief idea of what it does.>

Copyright (C) <textyear>        <name of author>

This program is free software: you can redistribute it and/or modify
it under the terms of the GNU Affero General Public License as published by
the Free Software Foundation, either version 3 of the License, or
(at your option) any later version.

This program is distributed in the hope that it will be useful,
but WITHOUT ANY WARRANTY; without even the implied warranty of
MERCHANTABILITY or FITNESS FOR A PARTICULAR PURPOSE.    See the
GNU Affero General Public License for more details.

You should have received a copy of the GNU Affero General Public License
along with this program.        If not, see <https://www.gnu.org/licenses/>.
\end{verbatim}
}

Also add information on how to contact you by electronic and paper mail.

If your software can interact with users remotely through a computer
network, you should also make sure that it provides a way for users to
get its source.  For example, if your program is a web application, its
interface could display a ``Source'' link that leads users to an archive
of the code.    There are many ways you could offer source, and different
solutions will be better for different programs; see section 13 for the
specific requirements.

You should also get your employer (if you work as a programmer) or
school, if any, to sign a ``copyright disclaimer'' for the program, if
necessary.      For more information on this, and how to apply and follow
the GNU AGPL, see \texttt{https://www.gnu.org/licenses/}.

\end{enumerate}

\end{document}
\nwenddocs{}
