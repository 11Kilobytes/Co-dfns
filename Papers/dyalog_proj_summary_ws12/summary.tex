\setuppapersize[letter]
\setupbodyfont[xits,12pt]
\setupindenting[yes,medium,first]
\setupwhitespace[small]
\definefont [tt] [Apl385]

\starttext
\title{Summary of the HPAPL Project}

Aaron W. Hsu {\tt <awhsu@indiana.edu>}

\subject{Introduction}

This summary provides a brief, concise overview of the HPAPL project 
and its status both as it currently stands and its planned future 
direction.  It particularly details the plans for this summer under the 
research grant provided by Dyalog.  HPAPL stands for “High Performance 
\{A,Axiomatic\} \{Parallel,Programming\} Language.”  It is an implementation 
of APL that has the twin goals of mathematical rigor and high 
performance on parallel architectures.  It is meant to facilitate the 
rapid development of high performance parallel programs.  Preliminary 
language design and compiler development should be complete this 
semester.  This includes preliminary formalisms and an infrastructure 
for defining unchecked compiler optimizations.  The Summer research 
goals encompass the necessay targets to move the existing prototype to 
producing high performance code in both parallel and serial contexts. 
The intended end result is a high speed, formal language lacking only a 
proper set of end-user and development tools.

\subject{Project Status and Pre-Summer Goals/Results}

This semester’s work has been driven by three primary goals.  Firstly,
we strove to explore the possibilities of such a system as HPAPL, 
including how it might be designed and implemented.  Secondly, we wanted 
a software artifact with basic functionality by the end and not just a 
semantics.  We did not intend to be feature complete, but we did want 
an adequate proof of concept.  With these goals in mind, we are in the 
process of designing a prototype compiler and a draft paper semantics 
that are capable of compiling and describing a set of parallel benchmarks. 
These are the NAS benchmarks commonly used in HPC comparisons.  We have 
settled on a design for the naïve compiler and are in the process of 
implementing it.  Little forward progress has been made on the paper 
semantics, but this is likely to change as we move forward on the 
primary compiler.  A start on the benchmarks is progressing by 
implementing the benchmarks first in Dyalog APL.

Based on our current progress, we anticipate having the naïve compiler 
core finished and a basic set of formal semantics complete.  We also 
expect to have several, but not all, of the NAS benchmarks complete.

\subject{Project Over the Summer}

At the conclusion of the current semester, our compiler will lack a 
number of important features.  First among these are the optimization
passes of the compiler.  It will also lack the complete set of APL 
functions and operators, and it will lack any sort of development tools. 
The paper semantics, meant to drive the optimizations, will not yet have been 
encoded into the compiler proper.

The above list of shortcomings cannot be addressed adequately in one 
summer.  Instead, the primary focus of the Summer research will be to 
make the prototype compiler produce the fastest code possible. 
We will leave the completeness of other features to a later date. 

\subject{Benefits of Working with Dyalog}

Current plans account for two months time in the UK over the Summer 
to enable more direct feedback from the Dyalog team.  The primary benefit 
of this is to leverage knowledge of the existing Dyalog code base 
and developer wisdom to expedite the optimization of serial elements 
of the HPAPL language.  This avoids wasting time on strategies that are 
slower than existing practices.  We would like to maximize our use of 
prior art to give HPAPL the best chance of succeeding.  It is also 
likely that discoveries in our compiler can lead to improvements in 
Dyalog's current products.

\subject{Delivery Goals and Conclusion}

By Summer's end, the goal is to have a compiler for HPAPL that generates 
fast code, whose optimizations are driven by proofs and proof obligations. 
The proofs are based on a formal axiomatic semantics of the language. 
Beyond the compiler, a set of well respected benchmarks should be 
encoded into HPAPL as a testament to the strengths and weaknesses of the 
then current version of the compiler.  We intend that this system will 
have several aspects that can be converted into papers and published 
in academic circles. 


\stoptext
