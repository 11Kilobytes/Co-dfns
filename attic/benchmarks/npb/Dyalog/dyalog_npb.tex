\input ConTeXtLPMacros

\definefontsynonym[APL][Apl385]
\setupbodyfont[xits]
\definefont[tt][APL sa 1]

\setuppapersize[letter][letter]
\setupwhitespace[medium]

\starttext
\startfrontmatter
\title{HPAPL: The Compiler}

\vfill
Copyright $\copyright$ 2012 \hfill\break
Aaron W. Hsu $⟨${\tt arcfide@sacrideo.us}$⟩$,\hfill\break
Brett Schrank $⟨${\tt bschrank@indiana.edu}$⟩$,\hfill\break
Timothy Downey $⟨${\tt tcdowney@indiana.edu}$⟩$.

Permission to use, copy, modify, and distribute this software for any
purpose with or without fee is hereby granted, provided that the above
copyright notice and this permission notice appear in all copies.

THE SOFTWARE IS PROVIDED "AS IS" AND THE AUTHOR DISCLAIMS ALL WARRANTIES
WITH REGARD TO THIS SOFTWARE INCLUDING ALL IMPLIED WARRANTIES OF
MERCHANTABILITY AND FITNESS. IN NO EVENT SHALL THE AUTHOR BE LIABLE FOR
ANY SPECIAL, DIRECT, INDIRECT, OR CONSEQUENTIAL DAMAGES OR ANY DAMAGES
WHATSOEVER RESULTING FROM LOSS OF USE, DATA OR PROFITS, WHETHER IN AN
ACTION OF CONTRACT, NEGLIGENCE OR OTHER TORTIOUS ACTION, ARISING OUT OF
OR IN CONNECTION WITH THE USE OR PERFORMANCE OF THIS SOFTWARE.

\completecontent
\stopfrontmatter

\startbodymatter

\chapter{Introduction}

This is an implementation of the NAS Parallel Benchmarks for Dyalog APL.
At the moment, it includes implementations of the IS, EP, and CG 
micro-benchmarks.  Our intention in creating these benchmarks is to 
establish a baseline of performance for current industrial strength 
implementations of APL, against which new research in APL optimization 
may be judged.  The NPB benchmark suite is a straigthforward, but 
well accepted set of benchmarks for testing distributed and parallel 
computation, which we anticipate will be the focus of modern APL 
optimization efforts.

Each benchmark is defined in its own namespace named according to its 
two-letter identifier. The following namespace can be used to load in 
individual benchmarks into an APL session.

\defchunk{Benchmark Loader}
:Namespace Loader
  benchfile←'./dyalog_npb.tex'
  ∇ Load BNM;CN;FN
    #.⎕SE.SALT.Load './ConTeXtLP -Target=#'
    CN←BNM,' Namespace'
    FN←'./',BNM,'.dyalog'
    CN #.ConTeXtLP.Tangle benchfile FN
    #.⎕SE.SALT.Load FN,' -Target=#'
  ∇
:EndNamespace
\stopchunk

To initially bootstrap the system, you can use the provided 
“boostrap.dyalog” file.  Otherwise, you can manually tangle the loader 
namespace using the following set of session commands.

\starttyping
]LOAD ./ConTeXtLP
'Benchmark Loader' ConTeXtLP.Tangle './dyalog_npb.tex' './loader.dyalog'
]LOAD ./loader
\stoptyping

\chapter{Random Number Generation}

Most of the NPB benchmarks require a specific style and implemenation 
of a random number generator in order to get deterministic results. 
Specifically, it is a linear congruential generator that obeys the following 
sequence for producing the next seeds.

\startformula
x_1 = ax_0\ (\mod 2^{46})
\stopformula

And the random number is generated on the range $(0,1)$ by computing 
$2^{-46}x$. We implement this as a class to make it easier to encapsulate 
the state data. You create a random generator by passing either a seed 
or a seed and an $a$ multiplier, which by default is $5^{13}$. 

\defchunk{Random Namespace}
:Class Random
   ⎕IO ⎕ML←0 0
   Seed A V←0 0,2*24
   
   ∇Make S
    :Access Public
    :Implements Constructor
    ⎕SIGNAL (1≡⊃⍴S)/11
    Seed A←S,5*13
   ∇
   
   ∇MakeWithA(s a)
    :Access Public
    :Implements Constructor
    Seed A←s a
   ∇
   
   Mul←{(2*46)|V⊥(V|A+.×B),[¯0.5](0⌷A←⌽V V⊤⍺)×1⌷B←V V⊤⍵}
	
   ∇R←RandI N
    :Access Public
    R←Seed
	0 0⍴({⍵ Mul ⍵⊣R,←⍵ Mul R}⍣(⌈2⍟N))A
	Seed←⊢/R←N↑R⊣R,←A Mul⊢/R←1↓R
	R←R×2*¯46
   ∇
:EndClass
\stopchunk

\chapter{Timing Benchmark Results}

To unify the timing mechanism for all of these benchmarks, we provide 
a class for timing.

\defchunk{Timer Namespace}
:Class Timer
    ⎕IO ⎕ML←0 0
    
    start←0
    end←0
    
    ∇ T←CurrentTime
      :Access Public
      T←(÷1000)×24 60 60 1000⊥3↓⎕TS
    ∇
    
    ∇ Start
      :Access Public
      start←CurrentTime
    ∇
    
    ∇ End
      :Access Public
      end←CurrentTime
    ∇
    
    ∇ Extend
      :Access Public
      End
    ∇
    
    ∇ R←Spent
      :Access Public
      R←end-start
    ∇
:EndClass
\stopchunk

\chapter{IS: Integer Sort}

\defchunk{IS Namespace}
:Namespace IS

    ClassA←(2*23)(2*19)314159265 10

    ∇ Run class;N;Bmax;startSeed;Imax;keys;Seed;Ts;i;R;Te;partialKeyIndex;partialKeyRank;j
      ⍝ Parameter values used by various classes,
      ⍝ See Figure 2.9 of NPB IS benchmark document.
      N Bmax startSeed Imax←class
      :If class≡ClassA
          partialKeyIndex←211237 662041 5336171 3642833 4250760  ⍝ Values to be used for partial
          partialKeyRank←104 17523 123928 8288932 8388264        ⍝ verification as specified in Table 2.8
      :EndIf
      Seed keys←startSeed #.UTIL.RandI N                         ⍝ Generate N keys
      Ts←24 60 60 1000⊥3↓⎕TS                                     ⍝ Begin timing
      :For i :In ⍳Imax
          keys[i Imax+i]←i Bmax-i                                ⍝ Modify sequence of keys
          R←⍋keys                                                ⍝ Compute the ranks of keys
          :For j :In ⍳5                                          ⍝ Partial verification test
              :If j<3
                  :If R[partialKeyIndex[j]]≠partialKeyRank[j]+i
                      ⎕←'Iteration',i,': Partial verification failed.'
                  :EndIf
              :Else
                  :If R[partialKeyIndex[j]]≠partialKeyRank[j]-i
                      ⎕←'Iteration',i,': Partial verification failed.'
                  :EndIf
              :EndIf
          :EndFor
      :EndFor
      Te←24 60 60 1000⊥3↓⎕TS                                     ⍝ End timing
      ⎕←'Time taken: ',(⍕(Te-Ts)÷1000),' seconds.'
      :If ∧/2≤/keys[R]                                           ⍝ Full verification test
          ⎕←'Full verification suceeded.'
      :Else
          ⎕←'Full verification failed.'
      :EndIf
    ∇

:EndNamespace 
\stopchunk

\chapter{EP: Embarassingly Parallel}

\defchunk{EP Namespace}
:Namespace EP
    ⎕IO ⎕ML←1 0
    
    Gen←⍬
    
    DoBlock←{
        n←⍵
        r←Gen.RandI 2×n
        x y←⊂[2]¯1+2×r[¯1 0+[1]2 n⍴2×⍳n]
        tf←1≥t←⊃+/x y*2
        x y t←(⊂tf)/¨x y t
        X Y←x y×⊂((¯2×⍟t)÷t)*÷2
        Q←+/(l∘.≤M)∧(l+1)∘.>M←(|X)⌈|Y⊣l←¯1+⍳10
        (+/¨X Y),Q
    }
    
    ∇R←BS Run N;Time;C
       Gen←⎕NEW #.Random(271828183,5*13)
       Time←⎕NEW #.Timer
       C←⌊N÷BS
       Time.Start
       R←BS({⍵+DoBlock ⍺}⍣C)12⍴0
       R+←{0≠⍵: DoBlock ⍵ ⋄ 12⍴0} N-BS×C
       Time.End
       ⎕←'Running took ',(⍕Time.Spent),' seconds.'
    ∇

:EndNamespace
\stopchunk

\chapter{CG: Conjugate Gradient}

\defchunk{CG Namespace}
:Namespace CG
⍝ === VARIABLES ===

    ClassA←14000 15 11 20

    ClassSample←1400 15 7 10


⍝ === End of variables definition ===

    ⎕IO ⎕ML ⎕WX ⎕RL←0 0 3 2090219546

    ∇ ZRN←X CgSparse A;R;Rho;P;I;Q;Alpha;RhoZ;Beta;Z
      Z←0 ⋄ R←X ⋄ Rho←R+.×R ⋄ P←R
      :For I :In ⍳25
          Q←A SMVP P
          Alpha←Rho÷P+.×Q
          Z←Z+Alpha×P
          RhoZ←Rho
          R←R-Alpha×Q
          Rho←R+.×R
          Beta←Rho÷RhoZ
          P←R+Beta×P
      :EndFor
      R←X-A SMVP Z
      ZRN←Z((R+.×R)*0.5)
    ∇

    ∇ A←MakeSA(N NZ L);W;R;Xi;Xv;I;IV;RN;CI;DV;Ts;Te;S;Av;Ai;At;F
     ⍝ Generate a test matrix of order N as the weighted sum of N outer products of
     ⍝ random sparse vectors.
      ⎕IO←0 ⋄ Ts←⎕TS
      R←0.1*÷N-1                    ⍝ Geometric ratio for W
      Ai Av←⍬ ⍬                     ⍝ Initialize empty sparse matrix A
      :For I :In ⍳N
          Xi←(I≠Xi)/Xi←NZ?N         ⍝ NZ nonzeros maybe with no X_i
          Xv←(2*¯30)×?(⍴Xi)⍴2*30    ⍝ Random values (0..1)
          Xi,←I ⋄ Xv,←0.5           ⍝ X_i is 0.5
          Av,←(R*I)×,Xv∘.×Xv        ⍝ Catenate weighted outer product
          At←((⍴Xi)*2)⍴Xi           ⍝ Nonzero row/column pattern
          Ai,←At+N×,⍉(2⍴⍴Xi)⍴At     ⍝ Catenate respective ravel indices
      :EndFor
      S←⍋Ai ⋄ Ai←Ai[S] ⋄ Av←Av[S]
      S←(1⌽Ai)≠Ai
      IV←S/Ai
      DV←S/+\Av
      DV←(⊃DV),1↓DV-¯1⌽DV
      CI←N|IV                              ⍝ Extract the Column indices
      RN←N⍴0 ⋄ RN[(IV-CI)÷N]+←1            ⍝ Build the NZ row vector
      DV[((RN/⍳⍴RN)=CI)/⍳⍴DV]+←0.1         ⍝ Diagonal gets additional 0.1
      CI←N|CI-L                            ⍝ Shift the diagonal
      A←RN CI DV                           ⍝ A with goodies
      Te←⎕TS
      ⎕←'Made array in ',(⍕Ts #.UTIL.TimerSecs Te),' seconds.'
    ∇

    ∇ R←M SMVP V;S;NZ;CI;DV;B;⎕IO
      ⎕IO←0
     ⍝ Sparse Matrix, Dense vector product, inspired by
     ⍝  Hendriks, Ferdinand and Wai-Mee Ching.
     ⍝  "Sparse Matrix Technology Tools in APL."
     ⍝  1990 ACM.
     ⍝ NZ CI DV←M                ⍝ Explode the matrix
     ⍝ R←DV×V[CI]
     ⍝ R←0,+\R            ⍝ Do the product
     ⍝ R←R[(0≠NZ)×+\NZ]          ⍝ Get desired sums
     ⍝ S←(0≠R)/⍳⍴R               ⍝ Select the nonzero results
     ⍝ R[S]←R[⊃S],1↓R[S]-¯1⌽R[S] ⍝ Subtract excess
      ⎕IO←0
      NZ CI DV←M
      B←(+/NZ)⍴0
      B[¯1++\NZ]←1
      R←(×NZ)\-2-/0,B/+\DV×V[CI]
    ∇

    ∇ (Cg Solve)(N NI NZ L);A;X;I;Z;R;Rn;Zeta;Ts;Te;⎕IO
      ⎕IO←0
      A←MakeSA N NZ L
      X←N⍴1
      ⎕←'Iteration        ∥R∥                 Zeta'
      ⎕←'─────────────────────────────────────────────────────'
      Ts←⎕TS
      :For I :In 1+⍳NI
          Z Rn←X Cg A
          Zeta←L+÷X+.×Z
          ⎕←9 0 20 ¯7 24 13⍕I Rn Zeta
          X←Z÷(Z+.×Z)*0.5
      :EndFor
      Te←⎕TS
      ⎕←''
      ⎕←'Time taken: ',(⍕Ts #.UTIL.TimerSecs Te),' seconds.'
    ∇

:EndNamespace 
\stopchunk


\stopbodymatter

\startappendices

\stopappendices

\startbackmatter
\completeindex
\stopbackmatter

\stoptext
